%
% Copyright � 2012 Peeter Joot.  All Rights Reserved.
% Licenced as described in the file LICENSE under the root directory of this GIT repository.
%

%
%
%\documentclass{article}

%\input{../peeters_macros.tex}
%\input{../peeters_macros2.tex}

%\usepackage[bookmarks=true]{hyperref}

%\usepackage{color,cite,graphicx}
   % use colour in the document, put your citations as [1-4]
   % rather than [1,2,3,4] (it looks nicer, and the extended LaTeX2e
   % graphics package.
%\usepackage{latexsym,amssymb,epsf} % do not remember if these are
   % needed, but their inclusion can not do any damage


\chapter{Some rapidity angle notes}
\label{chap:rapidity}
%\author{Peeter Joot \quad peeterjoot@protonmail.com}
\date{ Dec 18, 2008.  rapidity.tex }

%\begin{document}

%\maketitle{}
%\tableofcontents

\section{Motivation}

Lut writes, "setting up a little calculation, I am writing a 4-velocity as"

\begin{equation}\label{eqn:rapidity:20}
\begin{aligned}
( \gamma, \gamma \beta_x, \gamma \beta_y, \gamma \beta_z ),
\end{aligned}
\end{equation}

which has length \(-1\) if \(\gamma^{-2} = 1 - (\beta_x)^2+(\beta_y)^2+(\beta_z)^2\).

Can you write this in terms of the 3 rapidities \(a_1, a_2, a_3\) ?

I was not able to answer this right away so it is worth an examination
of rapidity angles to ensure that I understand the ideas.

\section{Stuff}

Putting back in the \(c\) factors, and switching to the \(+---\) signature I am used to, the position
vector is

\begin{equation}\label{eqn:rapidity:40}
\begin{aligned}
x &= x^\mu \gamma_\mu = ct \gamma_0 + x^i \gamma_i,
\end{aligned}
\end{equation}

for which the corresponding proper velocity is
\begin{equation}\label{eqn:rapidity:60}
\begin{aligned}
v &= \frac{dx}{d\tau} = c \frac{dt}{d\tau} \gamma_0 + \frac{dx^i}{dt} \frac{dt}{d\tau} \gamma_i
\end{aligned}
\end{equation}

Writing \(\gamma = dt/d\tau\), and squaring the proper velocity we have

\begin{equation}\label{eqn:rapidity:80}
\begin{aligned}
\frac{v^2}{c^2}
&= 1 \\
&= \gamma^2 \left(\gamma_0 + \inv{c}\frac{dx^i}{dt} \gamma_i\right)^2 \\
&= \gamma^2 \left(1 - \sum_i \inv{c^2} \left(\frac{dx^i}{dt}\right)^2 \right) \\
\end{aligned}
\end{equation}

So we have

\begin{equation}\label{eqn:rapidity:100}
\begin{aligned}
\gamma
&= \inv{\sqrt{1 - \sum_i \inv{c^2} \left(\frac{dx^i}{dt}\right)^2 }} \\
\end{aligned}
\end{equation}

Observe that \(\gamma\) ranges from \(1\) to infinity, and can thus be described by the \([0,\infty]\) range of the hyperbolic cosine function.  With
the relative velocity \(\Bv = \sum_i (dx^i/dt) \sigma_i\), this is

\begin{equation}\label{eqn:rapidity:120}
\begin{aligned}
\frac{dt}{d\tau} &= \gamma  \\
&= \cosh\alpha \\
&= \inv{\sqrt{1 - (\Bv/c)^2}}
\end{aligned}
\end{equation}

In terms of the hyperbolic cosine for \(\gamma\) our proper velocity then becomes

\begin{equation}\label{eqn:rapidity:140}
\begin{aligned}
v/c &= \cosh\alpha \left(1 + \frac{\Bv}{c} \right) \gamma_0
\end{aligned}
\end{equation}

Taking the hint from the Lorentz transform where we have both \(\sinh\) and \(\cosh\) factors can one write

\begin{equation}\label{eqn:rapidity:160}
\begin{aligned}
\gamma \frac{\Bv}{c} = \sinh\alpha
\end{aligned}
\end{equation}

This gives

\begin{equation}\label{eqn:rapidity:180}
\begin{aligned}
\frac{\Bv}{c} = \tanh\alpha
\end{aligned}
\end{equation}

so we need \(\alpha\) to be a spacetime relative vector.  With \(\cosh\) being an even function \(\cosh{\alpha} = \cosh{\Abs{\alpha}}\), so this
is still a
scalar as desired.  Inverting the relationship for \(\alpha\) we have

\begin{equation}\label{eqn:rapidity:200}
\begin{aligned}
\alpha = \tanh^{-1} (\Bv/c) = \vcap \tanh^{-1} (\Abs{\Bv/c})
\end{aligned}
\end{equation}

The unit vector \(\vcap\) can be factored out of the inverse hyperbolic tangent function since it is odd (consider the Taylor series expansion of \(\tanh^{-1}\) to see why one can do this).

Finally, we have by
dotting with the spatial basis vectors \(\sigma_i\) three quantities in terms of spacetime vector rapidity angle

\begin{equation}\label{eqn:rapidity:220}
\begin{aligned}
\alpha_i = (\vcap \cdot \sigma_i) \tanh^{-1} (\Abs{\Bv/c}).
\end{aligned}
\end{equation}

The \(\vcap \cdot \sigma_i\) parts are direction cosines, so the three rapidities Lut was asking about all appear to be weighted direction cosines.

%\bibliographystyle{plainnat}
%\bibliography{myrefs}

%\end{document}
