%
% Copyright � 2012 Peeter Joot.  All Rights Reserved.
% Licenced as described in the file LICENSE under the root directory of this GIT repository.
%

%
%
%\documentclass{article}

%\input{../peeters_macros.tex}
%\input{../peeters_macros2.tex}

%\usepackage[bookmarks=true]{hyperref}

%\usepackage{color,cite,graphicx}
   % use colour in the document, put your citations as [1-4]
   % rather than [1,2,3,4] (it looks nicer, and the extended LaTeX2e
   % graphics package.
%\usepackage{latexsym,amssymb,epsf} % do not remember if these are
   % needed, but their inclusion can not do any damage


\chapter{Bohm Chapter 9 problems}
\label{chap:bohmCh9}
%\author{Peeter Joot \quad peeterjoot@protonmail.com }
\date{ March 6, 2009.  bohmCh9.tex }

%\begin{document}

%\maketitle{}

%\tableofcontents

\section{Bohm Chapter 9 problems}

Problems and additional details from reading of \citep{bohm1989qt}, chapter 9.

\subsection{P1. Momentum wave function normalization}

Given a normalized wave function

\begin{equation}\label{eqn:bohmCh9:20}
\begin{aligned}
\IIinf \psi^\conj(x) \psi(x) dx = 1
\end{aligned}
\end{equation}

Show that the wave function \(\phi(k)\) is also normalized, and find the normalization factor for \(\Phi(p)\).

\begin{equation}\label{eqn:bohmCh9:40}
\begin{aligned}
\IIinf \phi^\conj(k) \phi(k) dk
&=
\IIinf \phi^\conj(k) \left( \inv{\sqrt{2\pi}} \IIinf \psi(x) e^{-i k x} dx \right) dk  \\
&=
\IIinf \left( \inv{\sqrt{2\pi}} \IIinf \phi^\conj(k) e^{-i k x} dk \right) \psi(x) dx  \\
&=
\IIinf {\left( \inv{\sqrt{2\pi}} \IIinf \phi(k) e^{i k x} dk \right)}^\conj \psi(x) dx  \\
&=
\IIinf \psi^\conj(x) \psi(x) dx  \\
&= 1
\qedmarker
\end{aligned}
\end{equation}

Bohm defines \(\Phi(p) \propto \phi(k)\) with the normalization constant determined by \(\int \Abs{\Phi(p)} dp = 1\).  Suppose we
let \(\Phi(p) = \alpha \phi(k)\), then we have

\begin{equation}\label{eqn:bohmCh9:60}
\begin{aligned}
1
&= \int \Phi^\conj(p) \Phi(p) dp \\
&= \int \alpha^2 \phi^\conj(k) \phi(k) \Hbar d k
\end{aligned}
\end{equation}

So we want \(\alpha^2 \Hbar = 1\), and therefore \(\Phi(p) = \inv{\sqrt{\Hbar}} \phi(k)\).

In \citep{mcmahon2005qmd}, with followup in \citep{gabookI:PJqmFourier} we have seen that an alternate Fourier transform pair can be used in terms of
momentum variables.  That is

\begin{equation}\label{eqn:bohmCh9:80}
\begin{aligned}
\Phi(p) &= \FM \IIinf \psi(x) e^{-ipx/\Hbar} dx \\
\psi(x) &= \FM \IIinf \Phi(p) e^{ipx/\Hbar} dp \\
\end{aligned}
\end{equation}

Observe that this is consistent with Bohm's notation, since one can read off
\(\Phi(p)\) in terms of \(\phi(k)\).
by inspection

\begin{equation}\label{eqn:bohmCh9:100}
\begin{aligned}
\Phi(p) &= \FM \IIinf \psi(x) e^{-ipx/\Hbar} dx = \inv{\sqrt{\Hbar}} \phi(k)
\end{aligned}
\end{equation}

\subsection{P2. Expectation of polynomial momentum function}

Given a function of momentum

\begin{equation}\label{eqn:bohmCh9:120}
\begin{aligned}
f(p) &= \sum C_n p^n
\end{aligned}
\end{equation}

Express the average, or expectation value of \(f(p)\).  It is sufficient to consider one of the monomial terms, say \(p^n\).  A translation
to position basis via Fourier transformation produces the desired result

\begin{equation}\label{eqn:bohmCh9:140}
\begin{aligned}
\expectation{p^n}
&= \int \Phi^\conj(p) p^n \Phi(p) dp \\
&= \inv{2\pi \Hbar} \iiint \left( \psi^\conj(x') e^{ipx'/\Hbar} dx' \right) (\Hbar k)^n \left( \psi(x) e^{-ipx/\Hbar} dx \right) (\Hbar dk) \\
&= \frac{\Hbar^n}{2\pi } \iiint \psi^\conj(x') e^{ikx'} dx' k^n e^{-i k x} \psi(x) dx dk \\
\end{aligned}
\end{equation}

The \(k^n\) can be reduced to differential form as Bohm did for the \(\expectation{p}\) case

\begin{equation}\label{eqn:bohmCh9:160}
\begin{aligned}
k^n e^{-i k x}
&= k^{n-1} k e^{-i k x} \\
&= k^{n-1} i \PD{x}{}e^{-i k x} \\
&= k^{n-m} i^m \PDN{x}{}{m}e^{-i k x} \\
&= i^n \PDN{x}{}{n}e^{-i k x} \\
\end{aligned}
\end{equation}

This leaves something that is in shape for integration by parts

\begin{equation}\label{eqn:bohmCh9:180}
\begin{aligned}
\expectation{p^n}
&= \frac{(i\Hbar)^n}{2\pi} \iiint \psi^\conj(x') e^{ikx'} dx' \left( \PDN{x}{}{n}e^{-i k x} \right) \psi(x) dx dk \\
&= \frac{(-i\Hbar)^n}{2\pi} \iiint \psi^\conj(x') e^{ikx'} dx' \PDN{x}{\psi(x)}{n} e^{-i k x} dx dk \\
&= \frac{(-i\Hbar)^n}{2\pi} \iiint \psi^\conj(x') e^{ik(x'-x)} \PDN{x}{\psi(x)}{n} dx' dx dk \\
&= {(-i\Hbar)^n}{} \iint \psi^\conj(x') \PDN{x}{\psi(x)}{n} dx' dx \inv{2\pi}\int e^{ik(x'-x)} dk \\
\end{aligned}
\end{equation}

This last integral is really a distribution, and can be identified with the delta function \(\delta(x'-x)\) operating on, in this case, the preceding integral.
%, and operates on a test function.  Suppose we designate such a test function as \(a(u)\), then we have
%
%\int \inv{2\pi}\int e^{iku} dk a(u) du
%&= \int \inv{2\pi}\int e^{iku} a(u) du dk \\

%Now, we can apply the distribution theory as covered in \citep{osgoodFourier} to do a delta function reduction of the exponentials in
%this integral.  Specifically pick a function \(a(k)\)

So we have
\begin{equation}\label{eqn:bohmCh9:200}
\begin{aligned}
\expectation{p^n}
&= {(-i\Hbar)^n}{} \iint \psi^\conj(x') \PDN{x}{\psi(x)}{n} dx' dx \delta(x'-x) \\
&= {(-i\Hbar)^n}{} \int \psi^\conj(x) \PDN{x}{\psi(x)}{n} dx \\
\end{aligned}
\end{equation}

We can put this into explicit operator form, nicely motivating the identification of \(-i\Hbar \PDi{x}{}\) with the momentum by virtue
of the definition of the average or expectation value.

\begin{equation}\label{eqn:bohmCh9:220}
\begin{aligned}
\expectation{p^n}
&= \int \psi^\conj(x) {\left( -i \Hbar \PD{x}{} \right)}^n {\psi(x)} dx \\
\end{aligned}
\end{equation}

\subsection{P3.  Expectation of position in momentum space}

\begin{equation}\label{eqn:bohmCh9:240}
\begin{aligned}
\expectation{x}
&= \int \psi^\conj(x) x \psi(x) dx \\
&=
\inv{2\pi\Hbar} \iiint \Phi^\conj(p) e^{-ipx/\Hbar} dp x \Phi(p') e^{ip'x/\Hbar} dp' dx \\
&=
\inv{2\pi\Hbar} \iiint \Phi^\conj(p) e^{-ipx/\Hbar} dp \left( -i \PD{p'}{} e^{ip'x/\Hbar} \right) \Phi(p') dp' dx \\
&=
\inv{2\pi\Hbar} \iiint \Phi^\conj(p) e^{-ipx/\Hbar} dp \left( i \PD{p'}{\Phi(p')} \right) e^{ip'x/\Hbar} dp' dx \\
&=
\iint \Phi^\conj(p) \left( i \PD{p'}{} \right) {\Phi(p')} dp dp' \inv{2\pi\Hbar} \int e^{i(p'-p)x/\Hbar} dx  \\
&=
\iint \Phi^\conj(p) \left( i \PD{p'}{} \right) {\Phi(p')} dp dp' \delta(p'-p)  \\
\end{aligned}
\end{equation}

This is

\begin{equation}\label{eqn:bohmCh9:260}
\begin{aligned}
\expectation{x} &= \int \Phi^\conj(p) \left( i \PD{p}{} \right) {\Phi(p)} dp
\end{aligned}
\end{equation}

We see that expressing momentum in position space and position in momentum space both result in differential
operator forms in calculations of expected values

\begin{equation}\label{eqn:bohm_ch9:operatorCorrespondance}
\begin{aligned}
p &\sim -i \Hbar \PD{x}{} \\
x &\sim i \Hbar \PD{p}{}
\end{aligned}
\end{equation}

Observe the Hamiltonian and Poisson equation structure in these two sets of operators.

\subsection{P4. Expectation of polynomial position function}

This problem follows just as P2, and I am not going to bother typing it up for myself.  For validity, we require
\(x^n \phi(x) \rightarrow 0\) as \(x \rightarrow \pm \infty\), or equivalently that \(\PDN{p}{\Phi}{n} \rightarrow 0\).

\subsection{P5. Some commutator calculations}

\subsubsection{P5. Position momentum moment commutators}

Evaluate

\begin{equation}\label{eqn:bohmCh9:280}
\begin{aligned}
f(x,p) = x^n p^m - p^m x^n
\end{aligned}
\end{equation}

Up to this point we have only seen operators in expectation values.  Let us look the simplest case with \(n = m = 1\) in that
context

\begin{equation}\label{eqn:bohmCh9:300}
\begin{aligned}
\expectation{f}
&= \frac{\Hbar}{i} \int \psi^\conj(x) \left(x \PD{x}{} - \PD{x}{} x \right) \psi(x) dx \\
&= \frac{\Hbar}{i} \int \psi^\conj(x) \left(x \PD{x}{\psi(x)} - \psi(x) - x \PD{x}{\psi(x)} \right) dx \\
&= -\frac{\Hbar}{i} \int \psi^\conj(x) \psi(x) dx \\
&= {i\Hbar}
\end{aligned}
\end{equation}

So in the same way that the operator correspondence between momentum and the derivative as summarized in
\eqnref{eqn:bohm_ch9:operatorCorrespondance}, one can associate the commutator operator with its action in the expectation value and
say

\begin{equation}\label{eqn:bohm_ch9:commutator}
\begin{aligned}
x p - p x \sim  i\Hbar
\end{aligned}
\end{equation}

The higher order commutator expansions could also be evaluated this way, but exploiting the operator nature directly
makes this easier.  For the first order moment commutator above one can write

\begin{equation}\label{eqn:bohmCh9:320}
\begin{aligned}
f(x,p) \psi(x)
&= (x p - p x) \psi(x) \\
&= -i \Hbar \left(x \PD{x}{} - \PD{x}{} x\right) \psi(x) \\
&= -i \Hbar \left(x \PD{x}{\psi}(x) - \PD{x}{x \psi(x)} \right) \\
&= -i \Hbar \left(x \PD{x}{\psi}(x) - \PD{x}{\psi(x)} -\psi(x) \right) \\
&= i \Hbar \psi(x) \\
\end{aligned}
\end{equation}

So again we see that as a right acting operator the net effect on any wave function is the following action

\begin{equation}\label{eqn:bohmCh9:340}
\begin{aligned}
(x p - p x) \psi = i \Hbar \psi \\
\end{aligned}
\end{equation}

If one starts from this point and then calculates the expectation value the result will still be \(i \Hbar\), but working
with the probability integrals from the get go is just additional complication.

Building on this result we can then calculate the higher order moment differences of the problem by using the commutator
to change the order of operations

\begin{equation}\label{eqn:bohmCh9:360}
\begin{aligned}
p x \sim -i \Hbar + x p
\end{aligned}
\end{equation}

Let us use this for a couple simple examples to start
\begin{equation}\label{eqn:bohmCh9:380}
\begin{aligned}
x^2 p - p x^2
&=
x^2 p - ( -i \Hbar + x p) x \\
&=
x^2 p + i \Hbar x - x ( -i \Hbar + x p) \\
&=
x^2 p + 2 i \Hbar x - x^2 p \\
&=
2 i \Hbar x \\
\end{aligned}
\end{equation}

\begin{equation}\label{eqn:bohmCh9:400}
\begin{aligned}
x p^2 - p^2 x
&=
x p^2 - p ( -i \Hbar + x p) \\
&=
x p^2 + i \Hbar p - p x p \\
&=
x p^2 + i \Hbar p + ( +i \Hbar - x p) p \\
&=
2 i \Hbar p \\
\end{aligned}
\end{equation}

Calculation of third powers shows a pattern, and one can guess at an induction hypothesis

\begin{equation}\label{eqn:bohmCh9:420}
\begin{aligned}
x p^n &= p^n x + n i \Hbar p^{n-1} \\
-p x^n &= -x^n p + n i \Hbar x^{n-1} \\
\end{aligned}
\end{equation}

The \(n=1\) cases follow from \(xp - px = i\Hbar\), leaving only the induction on \(n\).  For the momentum powers we have

\begin{equation}\label{eqn:bohmCh9:440}
\begin{aligned}
x p^n p
&= p^n x p + n i \Hbar p^{n} \\
&= p^n (p x + i \Hbar) + n i \Hbar p^{n} \\
&= p^{n+1} x + (n+1) i \Hbar p^{n}
\qedmarker
\end{aligned}
\end{equation}

For the position powers we have
\begin{equation}\label{eqn:bohmCh9:460}
\begin{aligned}
-p x^n x
&= -x^n p x + n i \Hbar x^{n} \\
&= x^n (-x p + i \Hbar) + n i \Hbar x^{n} \\
&= -x^{n+1} p + (n+1) i \Hbar x^{n}
\qedmarker
\end{aligned}
\end{equation}

This completes the proof for a first order version of the problem

\begin{equation}\label{eqn:bohmCh9:480}
\begin{aligned}
x p^n - p^n x &= n i \Hbar p^{n-1} \\
x^n p -p x^n &=  n i \Hbar x^{n-1} \\
\end{aligned}
\end{equation}

Observe that working with the operator form changes the calculation of derivatives problem in the original
commutator evaluation to nothing more than an algebraic exercise.
% (but one where there is a requirement to not accidentally invert the product order).

The general case still remains.  Building up to that let us do a couple examples

%x p^n =
%p^n x +  n i \Hbar p^{n-1} \\
%
%x^n p =
%p x^n  +  n i \Hbar x^{n-1} \\
%
%x^{n-1} p =
%(p x^{n-1}  +  (n-1) i \Hbar x^{n-2})
%x^{n-2} p =
%(p x^{n-2}  +  (n-2) i \Hbar x^{n-3})

\begin{equation}\label{eqn:bohmCh9:500}
\begin{aligned}
x^n p^2
&= (x^n p) p \\
&= (p x^n  +  n i \Hbar x^{n-1} ) p \\
&= p (x^n p) +  n i \Hbar (x^{n-1} p ) \\
&= p^2 x^n  +  2 n i \Hbar p x^{n-1} +  n (n-1) (i \Hbar)^2 x^{n-2} \\
\end{aligned}
\end{equation}

\begin{equation}\label{eqn:bohmCh9:520}
\begin{aligned}
x^n p^3
&=
(x^n p^2) p \\
&=
(p^2 x^n  +  2 n i \Hbar p x^{n-1} +  n (n-1) (i \Hbar)^2 x^{n-2} ) p \\
&=
  p^2 ( p x^n  +  n i \Hbar x^{n-1} )
+ 2 n i \Hbar p ( p x^{n-1}  +  (n-1) i \Hbar x^{n-2} )
+ n (n-1) (i \Hbar)^2 ( p x^{n-2}  +  (n-2) i \Hbar x^{n-3})
\\
&=
  p^3 x^n
+ 3 n (i \Hbar) p^2 x^{n-1}
+ 3 n(n-1) (i \Hbar)^2 p x^{n-2}
+ n (n-1)(n-2) (i \Hbar)^3 x^{n-3}
\\
\end{aligned}
\end{equation}

We see what looks like binomial coefficients, so a reasonable inductive hypothesis, for \(m \le n\)

\begin{equation}\label{eqn:bohm_ch9:commutatorMomentMlessThanN}
\begin{aligned}
x^n p^m
&= \sum_{j=0}^m \binom{m}{j} (i \Hbar)^j p^{m-j} x^{n-j} (n)(n-1)\cdots(n-j+1)
\end{aligned}
\end{equation}

And in particular, for \(m \le n\)
\begin{equation}\label{eqn:bohmCh9:540}
\begin{aligned}
x^n p^m - p^m x^n
&= \sum_{j=1}^m \binom{m}{j} (i \Hbar)^j p^{m-j} x^{n-j} (n)(n-1)\cdots(n-j+1)
\end{aligned}
\end{equation}

For \(m \ge n\), let us start with

\begin{equation}\label{eqn:bohmCh9:560}
\begin{aligned}
p^m x = x p^m -  m i \Hbar p^{m-1}
\end{aligned}
\end{equation}
%p^m x = x p^m -  m (i \Hbar) p^{m-1}
%p^{m-1} x = x p^{m-1} -  (m-1) (i \Hbar) p^{m-2}
%p^{m-2} x = x p^{m-2} -  (m-2) (i \Hbar) p^{m-3}

First do the \(x^2\)
\begin{equation}\label{eqn:bohmCh9:580}
\begin{aligned}
p^m x^2
&= x p^m x -  m (i \Hbar) p^{m-1} x \\
&= x (p^m x) -  m (i \Hbar) (p^{m-1} x) \\
&= x^2 p^m - 2 m (i \Hbar) x p^{m-1} +  m (m-1)(i \Hbar)^2 p^{m-2} \\
\end{aligned}
\end{equation}

And for the cube \(x^3\)
\begin{equation}\label{eqn:bohmCh9:600}
\begin{aligned}
p^m x^3
&=
( p^m x^2 ) x \\
&=
( x^2 p^m - 2 m (i \Hbar) x p^{m-1} +  m (m-1)(i \Hbar)^2 p^{m-2} ) x \\
&=
x^2 (p^m x )
- 2 m (i \Hbar) x (p^{m-1} x )
+ m (m-1)(i \Hbar)^2 (p^{m-2} x) \\
&=
x^2 ( x p^m -  m (i \Hbar) p^{m-1} ) \\
&\quad- 2 m (i \Hbar) x ( x p^{m-1} -  (m-1) (i \Hbar) p^{m-2} ) \\
&\quad+ m (m-1)(i \Hbar)^2 ( x p^{m-2} -  (m-2) (i \Hbar) p^{m-3} ) \\
&=
x^3 p^m
- 3 m (i \Hbar) x^2 p^{m-1}
+ 3 m (m-1) (i \Hbar)^2 x p^{m-2}
- m (m-1)(i \Hbar)^2 (m-2) (i \Hbar) p^{m-3} \\
\end{aligned}
\end{equation}

It appears that in this case with \(m \ge n\), like \eqnref{eqn:bohm_ch9:commutatorMomentMlessThanN}, we want as the induction statement

\begin{equation}\label{eqn:bohmCh9:620}
\begin{aligned}
p^m x^n &= \sum_{j=0}^n \binom{n}{j} (-i \Hbar)^j x^{n-j} p^{m-j} (m)(m-1)\cdots(m-j+1)
\end{aligned}
\end{equation}

And for the commutator moment the expected result, pending induction on the above, is
\begin{equation}\label{eqn:bohmCh9:640}
\begin{aligned}
x^n p^m - p^m x^n &= -\sum_{j=1}^n \binom{n}{j} (-i \Hbar)^j x^{n-j} p^{m-j} (m)(m-1)\cdots(m-j+1)
\end{aligned}
\end{equation}

Summarizing, this is
\begin{equation}\label{eqn:bohmCh9:660}
\begin{aligned}
&x^n p^m - p^m x^n
= \\
&\left\{
\begin{array}{l l}
\sum_{j=1}^m \binom{m}{j} (i \Hbar)^j p^{m-j} x^{n-j} (n)(n-1)\cdots(n-j+1) & \quad \mbox{if \(m \le n\)} \\
-\sum_{j=1}^n \binom{n}{j} (-i \Hbar)^j x^{n-j} p^{m-j} (m)(m-1)\cdots(m-j+1) & \quad \mbox{if \(m \ge n\)}
\end{array}
\right.
\end{aligned}
\end{equation}

\subsubsection{P5.b}

\begin{equation}\label{eqn:bohmCh9:680}
\begin{aligned}
e^{i k x} p -
p e^{ i k x}
\end{aligned}
\end{equation}

Reversing the second term via power series expansion we have

\begin{equation}\label{eqn:bohmCh9:700}
\begin{aligned}
p e^{ i k x}
&=
p \sum_{n=0}^\infty \frac{( i k x )^n}{n!} \\
&=
\sum_{n=0}^\infty \frac{( i k )^n}{n!} (x^n p - n i \Hbar x^{n-1} )
\\
&=
e^{ i k x} p
-\sum_{n=1}^\infty \frac{( i k )^n}{n!} (n i \Hbar x^{n-1} )
\\
&=
e^{ i k x} p
-(i k)(i\Hbar) \sum_{n=1}^\infty \frac{( i k x)^{n-1}}{(n-1)!}
\\
&=
e^{ i k x} p
+ (k\Hbar) e^{i k x}
\\
\end{aligned}
\end{equation}

So we have

\begin{equation}\label{eqn:bohmCh9:720}
\begin{aligned}
e^{ i k x} p - p e^{ i k x} &= - (k\Hbar) e^{i k x}
\end{aligned}
\end{equation}

\subsection{P6. Hermitian operators. Powers of momentum operators}

Show that \(p^n\) is Hermitian

\begin{equation}\label{eqn:bohmCh9:740}
\begin{aligned}
\expectation{p^n}^\conj
&=
\left( \int \psi^\conj (-i \Hbar)^n \frac{d^n}{dx^n} \psi \right)^\conj \\
&=
\int \psi (i \Hbar)^n \frac{d^n}{dx^n} \psi^\conj \\
&=
\int \left( (-1)^n \frac{d^n}{dx^n} \psi \right) (i \Hbar)^n \psi^\conj \\
&=
\int \left( p^n \psi \right) \psi^\conj \\
&=
\expectation{p^n}
\end{aligned}
\end{equation}

Thus, by the definition of equation (13) in the text, this operator is
Hermitian.

Next is to show that

\begin{equation}\label{eqn:bohmCh9:760}
\begin{aligned}
f(p) = \sum_k A_k p^k
\end{aligned}
\end{equation}

is Hermitian, provided \(A_k\) are all real.  This part is clear by inspection.

\subsection{P7. Hermitian operators. Powers of position operators}

Want to show that the following is Hermitian

\begin{equation}\label{eqn:bohmCh9:780}
\begin{aligned}
f(x) &= \sum A_k x^k
\end{aligned}
\end{equation}

If the conjugate of the expectation equals itself we required only \(A_k = A_k^\conj\), so \(A_k\) must be strictly real, and we are done.

\subsection{P8. Non Hermitian momentum power operators if derivative does not vanish}

Show that if \(\partial^n \psi/\partial x^n\) does not vanish then \((-i\Hbar)^{n+1} \partial^{n+1}/\partial x^{n+1}\) is not Hermitian.

We want to evaluate the following and compare it to its conjugate

\begin{equation}\label{eqn:bohmCh9:800}
\begin{aligned}
\expectation{p^{n+1}}
&= (-i\Hbar)^{n+1} \IIinf \psi^\conj \PDN{x}{\psi}{n+1} dx \\
&=
(-i\Hbar)^{n+1} \left. \psi^\conj \PDN{x}{\psi}{n} \right\vert_{-\infty}^{\infty}
+(-1)^{1}(-i\Hbar)^{n+1} \IIinf \PDN{x}{\psi^\conj}{1} \PDN{x}{\psi}{n} dx
\\
&=
(-i\Hbar)^{n+1} \left. \psi^\conj \PDN{x}{\psi}{n} \right\vert_{-\infty}^{\infty}
+(-1)^{1}(-i\Hbar)^{n+1} \left. \PDN{x}{\psi^\conj}{1} \PDN{x}{\psi}{n-1} \right\vert_{-\infty}^{\infty}
+(-1)^{2}(-i\Hbar)^{n+1} \IIinf \PDN{x}{\psi^\conj}{2} \PDN{x}{\psi}{n-1} dx
\\
&=
(-i\Hbar)^{n+1} \sum_{k=0}^1
(-1)^{k}
\left. \PDN{x}{\psi^\conj}{k} \PDN{x}{\psi}{n-k} \right\vert_{-\infty}^{\infty}
+(-1)^{2}(-i\Hbar)^{n+1} \IIinf \PDN{x}{\psi^\conj}{2} \PDN{x}{\psi}{n-1} dx
\\
&=
(-i\Hbar)^{n+1} \sum_{k=0}^m
(-1)^{k}
\left. \PDN{x}{\psi^\conj}{k} \PDN{x}{\psi}{n-k} \right\vert_{-\infty}^{\infty}
+(-1)^{m+1}(-i\Hbar)^{n+1} \IIinf \PDN{x}{\psi^\conj}{m+1} \PDN{x}{\psi}{n-m)} dx
\\
&=
(-i\Hbar)^{n+1} \sum_{k=0}^n
(-1)^{k}
\left. \PDN{x}{\psi^\conj}{k} \PDN{x}{\psi}{n-k} \right\vert_{-\infty}^{\infty}
+(i\Hbar)^{n+1} \IIinf \PDN{x}{\psi^\conj}{n+1} \psi dx
\\
&=
(-i\Hbar)^{n+1} \sum_{k=0}^n
(-1)^{k}
\left. \PDN{x}{\psi^\conj}{k} \PDN{x}{\psi}{n-k} \right\vert_{-\infty}^{\infty}
+
\expectation{p^{n+1}}^\conj
\\
\end{aligned}
\end{equation}
%\newcommand{\PDN}[3]{\frac{\partial^{#3} {#2}}{\partial {#1}^{#3}}}

So we have
\begin{equation}\label{eqn:bohmCh9:820}
\begin{aligned}
\expectation{p^{n+1}} - \expectation{p^{n+1}}^\conj
&=
(-i\Hbar)^{n+1} \sum_{k=0}^n
(-1)^{k}
\left. \PDN{x}{\psi^\conj}{k} \PDN{x}{\psi}{n-k} \right\vert_{-\infty}^{\infty}
\\
\end{aligned}
\end{equation}

If \(p^{n+1}\) is Hermitian, then this difference should be zero, but if the indicated partial does not vanish this
remainder bit can be non-zero.

\subsection{P9. m, n'th moment is Hermitian}

Consider the operator

\begin{equation}\label{eqn:bohmCh9:840}
\begin{aligned}
\sum A_{nm} \left(\frac{p^n x^m + x^m p^n}{2}\right)
\end{aligned}
\end{equation}

Show that this is Hermitian if all \(A_{nm}\) are real.

Consider first one specific term with \(A_{nm}\), calculate the conjugate of the expectation value, and integrate
by parts

\begin{equation}\label{eqn:bohmCh9:860}
\begin{aligned}
\left(
\int \psi^\conj \inv{2}(p^n x^m + x^m p^n) \psi
\right)^\conj
&=
\inv{2} (-1)^n
\int \psi (p^n x^m + x^m p^n) \psi^\conj \\
&=
\inv{2} (i\Hbar)^n
\int \psi \left(\frac{d^n (x^m \psi^\conj)}{dx^n} + x^m \frac{d^n \psi^\conj}{dx^n} \right) \\
&=
\inv{2} (-i\Hbar)^n
\int x^m \psi^\conj \left(\frac{d^n \psi }{dx^n} + \psi^\conj \frac{d^n (x^m \psi)}{dx^n} \right) \\
&=
\inv{2}
\int \psi^\conj ( x^m p^n + p^n x^m ) \psi
\end{aligned}
\end{equation}

This shows that \((p^n x^m + x^m p^n)/2\) is Hermitian, and the conjugation requires \(A_{nm}\) to be real for the
product of the two to be Hermitian.

\subsection{P10. Hermitizing Classical operator \texorpdfstring{\((px)^2\)}{p x p x}}

Show that
\begin{equation}\label{eqn:bohmCh9:880}
\begin{aligned}
\inv{2}\left(x^2 p^2 + p^2 x^2 \right)
\end{aligned}
\end{equation}

and
\begin{equation}\label{eqn:bohmCh9:900}
\begin{aligned}
\inv{4}\left(x p + p x \right)^2
\end{aligned}
\end{equation}

lead to results that differ by a factor of \(\Hbar^2\).

To do so consider the difference of the expectation of this operator, first calculating this difference.  We will
want to use the commutator relation, in a few equivalent forms

\begin{equation}\label{eqn:bohmCh9:920}
\begin{aligned}
xp - p x &= i \Hbar \\
xp &= p x + i \Hbar \\
px &= x p - i \Hbar \\
x p + p x &= 2 p x + i \Hbar \\
          &= 2 x p - i \Hbar \\
\end{aligned}
\end{equation}

This gives us

\begin{equation}\label{eqn:bohmCh9:940}
\begin{aligned}
\inv{4}\left(x p + p x \right)^2
&= \inv{4}(2 p x + i \Hbar)( 2 x p - i \Hbar ) \\
&= p x^2 p + i\Hbar \inv{2}(x p - p x) + \inv{4} \Hbar^2 \\
&= p x^2 p - \Hbar^2 \inv{2} + \inv{4} \Hbar^2 \\
&= p x^2 p - \Hbar^2 \inv{4} \\
\end{aligned}
\end{equation}

For the other operator, reduction to a form that also contains \(p x^2 p\), we have

\begin{equation}\label{eqn:bohmCh9:960}
\begin{aligned}
\inv{2}\left(x^2 p^2 + p^2 x^2 \right)
&=
\inv{2}\left(x (x p) p + p (p x) x \right) \\
&=
\inv{2}\left(x (p x + i \Hbar) p + p (x p -i\Hbar) x \right) \\
&=
\inv{2}\left((x p) x p + p x (p x) + i \Hbar ( x p - p x ) \right) \\
&=
\inv{2}\left((p x + i\Hbar) x p + p x ( x p - i\Hbar) + (i \Hbar)^2 \right) \\
&=
\inv{2}\left( 2 p x^2 p + i \Hbar ( x p - p x) + -\Hbar^2 \right) \\
&=
p x^2 p + -\Hbar^2
\end{aligned}
\end{equation}

So now, if we take the difference
\begin{equation}\label{eqn:bohmCh9:980}
\begin{aligned}
\inv{2}\left(x^2 p^2 + p^2 x^2 \right) - \inv{4}\left(x p + p x \right)^2
&= (p x^2 p + -\Hbar^2 ) - (p x^2 p - \Hbar^2 \inv{4})
 \\
&= -\frac{3}{4}\Hbar^2
 \\
\end{aligned}
\end{equation}

The difference of the expectation values of these operators is thus of the order \(\Hbar^2\) as was to be calculated.

\subsection{P11.  An explicit calculation of a Hermitian operator}

Show by integration by parts that \((xp)^\dagger = p x\).

The defining relation for the Hermitian conjugation operation is equation 16 in the text.

\begin{equation}\label{eqn:bohmCh9:1000}
\begin{aligned}
\int \psi^\conj (O^\dagger \psi) dx &= \int \psi (O^\conj \psi^\conj) dx
\end{aligned}
\end{equation}

For the operator \(xp\), we have

\begin{equation}\label{eqn:bohmCh9:1020}
\begin{aligned}
\int \psi^\conj (xp)^\dagger \psi dx
&= \int \psi (xp)^\conj \psi^\conj dx \\
&= (-i\Hbar)^\conj \int \psi x \PD{x}{\psi^\conj} dx \\
&= - i\Hbar \int \PD{x}{x \psi} \psi^\conj dx \\
&= \int \psi^\conj (p x) \psi dx \\
\end{aligned}
\end{equation}

So we have, as desired

\begin{equation}\label{eqn:bohmCh9:1040}
\begin{aligned}
(xp)^\dagger = p x
\end{aligned}
\end{equation}

\subsection{P12. Hermitian operator from antisymmetric difference}

Show that \(H = i (O - O^\dagger)\) is a Hermitian operator.

This follows directly from the definition, calculating the expectation

\begin{equation}\label{eqn:bohmCh9:1060}
\begin{aligned}
\expectation{H}
&=\int \psi^\conj (i (O - O^\dagger) \psi ) \\
&=
i \int \psi^\conj (O \psi )
-i \int \psi^\conj (O^\dagger \psi ) \\
&=
i \int \psi^\conj (O \psi )
-i \int \psi (O^\conj \psi^\conj) \\
\end{aligned}
\end{equation}

Taking conjugates we have

\begin{equation}\label{eqn:bohmCh9:1080}
\begin{aligned}
\expectation{H}^\conj
&=
-i \int \psi (O^\conj \psi^\conj )
+i \int \psi^\conj (O \psi ) \\
&=
\expectation{H} \\
\qedmarker
\end{aligned}
\end{equation}


\subsection{P13. When product of operators is Hermitian}

What relation must exist between Hermitian \(B\) and \(A\) must exist for \(AB\) to be Hermitian.

TODO:
Am guessing that this has something to do with the commutator of the operators.  This one I do not have a check mark
besides in my text, so did I ever figure it out?

\subsection{P14. Show directly that \texorpdfstring{\(i(p^2 x - xp^2)\)}{i p p x - i x p p} is Hermitian}

This follows from \((AB)^\dagger = BA\) in the text above.  We have

\begin{equation}\label{eqn:bohmCh9:1100}
\begin{aligned}
(i (p^2 x - x p^2))^\dagger
&=
(x^\dagger p^\dagger p^\dagger - p^\dagger p^\dagger x^\dagger)(-i) \\
&=
-i (x p^2 - p^2 x) \\
&=
i (p^2 x - x p^2 )
\qedmarker
\end{aligned}
\end{equation}

%\bibliographystyle{plainnat}
%\bibliography{myrefs}

%\end{document}
