%
% Copyright � 2012 Peeter Joot.  All Rights Reserved.
% Licenced as described in the file LICENSE under the root directory of this GIT repository.
%

%
%
%\documentclass{article}

%\input{../peeters_macros.tex}
%\input{../peeters_macros2.tex}

%\usepackage{listings}
%\usepackage{txfonts} % for ointctr... (also appears to make "prettier" \int and \sum's)
%\usepackage[bookmarks=true]{hyperref}

%\usepackage{color,cite,graphicx}
   % use colour in the document, put your citations as [1-4]
   % rather than [1,2,3,4] (it looks nicer, and the extended LaTeX2e
   % graphics package.
%\usepackage{latexsym,amssymb,epsf} % do not remember if these are
   % needed, but their inclusion can not do any damage


\chapter{Quantum Harmonic Oscillator}
\label{chap:harmonicOsc}
%\author{Peeter Joot \quad peeterjoot@protonmail.com }
\date{ April 19, 2009.  harmonicOsc.tex }

%\begin{document}

%\maketitle{}
%\tableofcontents
\section{Motivation}

In \citep{byron1992mca} (chapter II), an operator solution to the
(one dimensional) quantum
harmonic oscillator problem is presented.  Try this in a more old fashioned way,
as a comparison.

We want to solve the Schr\"{o}dinger equation for a quadratic potential

\begin{equation}\label{eqn:harmonic_osc:toSolve}
\begin{aligned}
-\frac{\Hbar^2}{2m} \psi + \inv{2} m \omega^2 x^2 \psi = i \Hbar \PD{t}{\psi}
\end{aligned}
\end{equation}

\section{Setup}

\subsection{Separation of variables}

\Eqnref{eqn:harmonic_osc:toSolve} is separable, and to do so we can write

\begin{equation}\label{eqn:harmonicOsc:20}
\begin{aligned}
\psi(x,t) = \phi(x) T(t)
\end{aligned}
\end{equation}

and proceed to form the separated equation

\begin{equation}\label{eqn:harmonicOsc:40}
\begin{aligned}
-\frac{\Hbar^2}{2m} \frac{\phi''}{\phi} + \inv{2} m \omega^2 x^2 = i \Hbar \frac{T'}{T} = \text{constant}
\end{aligned}
\end{equation}

Writing \(E\) for the constant, the and solving for the time function we have

\begin{equation}\label{eqn:harmonicOsc:60}
\begin{aligned}
(\ln(T))' &= -i \frac{E}{\Hbar} \\
\implies \\
\ln(T) &= -i \frac{Et}{\Hbar} + \ln(A)
\end{aligned}
\end{equation}

where \(A\) is some constant.  This yields

\begin{equation}\label{eqn:harmonicOsc:80}
\begin{aligned}
T(t) = A e^{ -i E t/\Hbar }
\end{aligned}
\end{equation}

What remains is now to solve the spatial wave equation

\begin{equation}\label{eqn:harmonic_osc:spatial}
\begin{aligned}
\phi'' - \left( \frac{m^2 \omega^2}{\Hbar^2} x^2 - \frac{2m E}{\Hbar^2}\right) \phi = 0
\end{aligned}
\end{equation}

This does not really look too much like the harmonic oscillator problem of classical physics.  Let us remind ourself
what that was like before continuing.

\subsection{Mass on a spring}

The harmonic oscillator problem from classical physics
shows up many times, and is usually first seen when examining motion
of a mass on a spring.  There we have a restoring force that accelerates the
mass in the opposite direction from its equilibrium position

\begin{equation}\label{eqn:harmonicOsc:100}
\begin{aligned}
m \xddot = -k x
\end{aligned}
\end{equation}

This has two complex exponential solutions.  With a substitution of the test function

\begin{equation}\label{eqn:harmonicOsc:120}
\begin{aligned}
x = e^{i \omega t}
\end{aligned}
\end{equation}

we have

\begin{equation}\label{eqn:harmonicOsc:140}
\begin{aligned}
(-m \omega^2 + k) e^{i \omega t} = 0
\end{aligned}
\end{equation}

So the test function is a solution provided

\begin{equation}\label{eqn:harmonicOsc:160}
\begin{aligned}
\omega^2 = \frac{k}{m}
\end{aligned}
\end{equation}

That does not really help understanding why \eqnref{eqn:harmonic_osc:spatial} is labeled the Harmonic oscillator.  Let us instead put the equation into an energy form.  The work done against the spring
(potential energy to be returned when the mass is released) is

\begin{equation}\label{eqn:harmonicOsc:180}
\begin{aligned}
W
&= - \int F \cdot dx \\
&= - \int -kx dx \\
&= \inv{2} k (x^2 - x_0^2)
\end{aligned}
\end{equation}

So, our Lagrangian is

\begin{equation}\label{eqn:harmonicOsc:200}
\begin{aligned}
\LL = \inv{2}m \xdot^2 - \inv{2} k (x^2 - x_0^2)
\end{aligned}
\end{equation}

The constant term in the potential can be dropped, since it will not contribute to the equations of motion.  Our conjugate momentum \(\PDi{\xdot}{\LL}\) is just \(m \xdot\), and the Hamiltonian
is therefore

\begin{equation}\label{eqn:harmonicOsc:220}
\begin{aligned}
H
&= \xdot \PD{\xdot}{\LL} - \LL \\
&= m \xdot^2  - \left( \inv{2}m \xdot^2 - \inv{2} k x^2 \right) \\
&= \inv{2} m \xdot^2  + \inv{2} k x^2 \\
\end{aligned}
\end{equation}

Or

\begin{equation}\label{eqn:harmonicOsc:240}
\begin{aligned}
H &= \frac{p^2}{2 m } + \inv{2} m \omega^2 x^2 \\
\end{aligned}
\end{equation}

Ah.  In this form we see the structure of the QM Harmonic oscillator equation.  With the position space representation of the momentum operator \(p \sim -i \Hbar \PDi{x}{}\) we have
something now similar to \eqnref{eqn:harmonic_osc:toSolve}.

\section{Series solution}

\subsection{Assuming Gaussian solutions}

Having seen the operator solution of the QM harmonic oscillator problem, we will cheat, and use that as a starting point.  Assume that
the solution can be expressed as a scaled Gaussian as in

\begin{equation}\label{eqn:harmonicOsc:260}
\begin{aligned}
\phi(x) &= f(x) e^{ - \alpha x^2/2 }
\end{aligned}
\end{equation}

\begin{equation}\label{eqn:harmonicOsc:280}
\begin{aligned}
\phi'(x) &= \left( f'(x) - \alpha x f(x) \right) e^{ - \alpha x^2/2 } \\
\phi''(x)
&=
\left( f''(x) - \alpha f(x) -\alpha x f'(x) \right) e^{ - \alpha x^2/2 }
+\left( f'(x) - \alpha x f(x) \right) (-\alpha x) e^{ - \alpha x^2/2 } \\
&=
\left( f''(x) - \alpha f(x) - 2 \alpha x f'(x) + \alpha^2 x^2 f(x) \right) e^{ - \alpha x^2/2 }
\end{aligned}
\end{equation}

Substitution of the scaled Gaussian test solution, and its derivatives, gives us

\begin{equation}\label{eqn:harmonicOsc:300}
\begin{aligned}
\left( f'' - \alpha f - 2 \alpha x f' + \alpha^2 x^2 f - \left( \frac{m^2 \omega^2}{\Hbar^2} x^2 - \frac{2 m E}{\Hbar^2} \right) f \right) e^{ -\alpha x^2 /2} = 0
\end{aligned}
\end{equation}

Since the exponential is never zero, this requires a zero for the differential equation

\begin{equation}\label{eqn:harmonicOsc:320}
\begin{aligned}
f'' - 2 \alpha x f' + \left( \left(\alpha^2 - \frac{m^2 \omega^2}{\Hbar^2} \right) x^2 + \left(\frac{2 m E}{\Hbar^2} - \alpha \right) \right) f = 0
\end{aligned}
\end{equation}

Compared to \eqnref{eqn:harmonic_osc:spatial}, this does not really appear to be much of an improvement, but let us work with it, looking for a term by term power series solution.

Before doing so, a couple helper variable substitutions appear to be in order.  Let

\begin{equation}\label{eqn:harmonicOsc:340}
\begin{aligned}
\beta^2 &= \alpha^2 - \frac{m^2 \omega^2}{\Hbar^2}  \\
\sigma &= \frac{2 m E}{\Hbar^2} - \alpha
\end{aligned}
\end{equation}

So the new differential equation to solve is

\begin{equation}\label{eqn:harmonicOsc:360}
\begin{aligned}
f'' - 2 \alpha x f' + \beta^2 x^2 f - \sigma f = 0
\end{aligned}
\end{equation}

Assuming various power series solutions of the form

\begin{equation}\label{eqn:harmonicOsc:380}
\begin{aligned}
f_n(x) &= \sum_{r=0}^n a_r x^r
\end{aligned}
\end{equation}

derivatives are
\begin{equation}\label{eqn:harmonicOsc:400}
\begin{aligned}
f_n' &= \sum_{r=0}^n r a_r x^{r-1} \\
f_n'' &= \sum_{r=0}^n r(r-1) a_r x^{r-2}
\end{aligned}
\end{equation}

We have
\begin{equation}\label{eqn:harmonicOsc:420}
\begin{aligned}
0 &= \sum_{r=0}^{n-2} (r+2)(r+1) a_{r+2} x^{r}
 - 2 \alpha \sum_{r=1}^n r a_r x^{r}
+ \beta^2
\sum_{r=0}^n a_r x^{r+2}
- \sigma \sum_{r=0}^n a_r x^r
\end{aligned}
\end{equation}

This should be enough to figure out recurrence relations for the various constants in the polynomials.

However, before trying to acquire the recurrence relations in their most general form, an attempt at a few simple cases, looking for the lowest order
polynomial solutions explicitly, gets into trouble.

Specifically, if I try \(n=0\), \(n=1\), \(n=2\) equating each of the polynomial coefficients to zero keeps killing all the coefficients in sequence.  What is
gone wrong?  FIXME: try again.  Had sign errors above.

\subsection{Try zeroth order polynomial scaled Gaussian explicitly}

Somewhere above things went wrong.  How about a plain old Gaussian?  Let us substitute

\begin{equation}\label{eqn:harmonicOsc:440}
\begin{aligned}
\psi = A e^{\alpha x^2/2}
\end{aligned}
\end{equation}

into

\begin{equation}\label{eqn:harmonicOsc:460}
\begin{aligned}
-\frac{\Hbar^2}{2m} \psi'' + \left( \inv{2}m \omega^2 x^2 - E \right) \psi = 0
\end{aligned}
\end{equation}

Dropping the constant \(A\) temporarily the derivatives are
\begin{equation}\label{eqn:harmonicOsc:480}
\begin{aligned}
\psi' &= \alpha x e^{\alpha x^2/2} \\
\psi'' &= ( \alpha^2 x^2 + \alpha ) e^{\alpha x^2/2} \\
\end{aligned}
\end{equation}

This gives
\begin{equation}\label{eqn:harmonicOsc:500}
\begin{aligned}
\left( -( \alpha^2 x^2 + \alpha )\frac{\Hbar^2}{2 m} + \left( \inv{2}m \omega^2 x^2 - E \right) \right) e^{\alpha x^2/2} = 0
\end{aligned}
\end{equation}

Equating \(x^2\) and \(x^0\) terms we have

\begin{equation}\label{eqn:harmonicOsc:520}
\begin{aligned}
\frac{\alpha^2 \Hbar^2}{2 m} &= \inv{2}m \omega^2  \\
\frac{\alpha \Hbar^2}{2 m} &= -E
\end{aligned}
\end{equation}

Or
\begin{equation}\label{eqn:harmonicOsc:540}
\begin{aligned}
\alpha &= \pm \frac{ m \omega }{ \Hbar }  \\
{\alpha } &= - \frac{2 m E }{\Hbar^2}
\end{aligned}
\end{equation}

Since \(\omega = \sqrt{k/m}\) is a given, we want \(E\) in terms of \(\omega\).  Picking \(\alpha\) negative for positive energy, this is

\begin{equation}\label{eqn:harmonicOsc:560}
\begin{aligned}
E = \frac{ \omega \Hbar }{ 2 }
\end{aligned}
\end{equation}

The solution to the \(T'/T\) equation is thus

\begin{equation}\label{eqn:harmonicOsc:580}
\begin{aligned}
T \propto e^{-i \omega t/2}
\end{aligned}
\end{equation}

So, except for the undetermined constant normalization factor, we have one full solution of the wave equation,

\begin{equation}\label{eqn:harmonicOsc:600}
\begin{aligned}
\psi(x,t) = A \exp\left( - \frac{m \omega x^2 }{2 \Hbar} - i \frac{\omega t}{2} \right)
\end{aligned}
\end{equation}

The normalization

\begin{equation}\label{eqn:harmonicOsc:620}
\begin{aligned}
1
&= \int \psi \psi^\conj \\
&= A^2 \int \exp\left( - \frac{m \omega x^2 }{\Hbar} \right) \\
&= A^2 \sqrt{ \frac{\Hbar \pi}{ m \omega } }
\end{aligned}
\end{equation}

For a normalized solution
\begin{equation}\label{eqn:harmonicOsc:640}
\begin{aligned}
\psi(x,t) = \left(
\frac{ m \omega }{\Hbar \pi}
\right)^{1/4} \exp\left( - \frac{m \omega x^2 }{2 \Hbar} - i \frac{\omega t}{2} \right)
\end{aligned}
\end{equation}

\subsection{Try first order polynomial scaled Gaussian}

Next, let us try

\begin{equation}\label{eqn:harmonicOsc:660}
\begin{aligned}
\psi = (x + A) e^{-\alpha x^2/2}
\end{aligned}
\end{equation}

No coefficient for the first order nomial has been used since we will need a scale factor in the end for normalization anyways.  Taking derivatives

\begin{equation}\label{eqn:harmonicOsc:680}
\begin{aligned}
\psi'
&= (1 + (x + A)(-\alpha x)) e^{-\alpha x^2/2} \\
&= (1 - \alpha A x - \alpha x^2 ) e^{-\alpha x^2/2} \\
\end{aligned}
\end{equation}

\begin{equation}\label{eqn:harmonicOsc:700}
\begin{aligned}
\psi''
&=
\left( - 3 \alpha x - \alpha A + \alpha^2 A x^2 + \alpha^2 x^3 \right) e^{-\alpha x^2/2} \\
\end{aligned}
\end{equation}

So we have

\begin{equation}\label{eqn:harmonicOsc:720}
\begin{aligned}
-\frac{\Hbar^2}{2m}\left( - 3 \alpha x - \alpha A + \alpha^2 A x^2 + \alpha^2 x^3 \right)
+ \left( \inv{2}m \omega^2 x^2 - E \right) (x + A) = 0
\end{aligned}
\end{equation}

Equating either cubic or squared terms provides \(\alpha\)

\begin{equation}\label{eqn:harmonicOsc:740}
\begin{aligned}
\alpha = \pm \frac{m \omega}{\Hbar}
\end{aligned}
\end{equation}

This is what we had for the plain old Gaussian as well.

Equating either the first and scalar terms gives us the energy

\begin{equation}\label{eqn:harmonicOsc:760}
\begin{aligned}
E = \frac{3 \alpha \Hbar^2}{2m} = \frac{3 \omega \Hbar}{2}
\end{aligned}
\end{equation}

which differs from the zeroth order case by a factor of three.

It is curious that the coefficient \(A\) cannot be determined.  I did not expect it to be a free parameter.  Is the normalization enough to
fix this and any other leading factor?

With
\begin{equation}\label{eqn:harmonicOsc:780}
\begin{aligned}
\psi = (B x + A) \exp( - m \omega x^2/2 \Hbar - 3 i \omega t / 2 )
\end{aligned}
\end{equation}

\begin{equation}\label{eqn:harmonicOsc:800}
\begin{aligned}
1
&= \int \psi\psi^\conj \\
&= \int (B^2 x^2 + 2 A B x + A^2 ) e^{- m \omega x^2/ \Hbar } \\
&= \int (B^2 x^2 + A^2 ) e^{- m \omega x^2/ \Hbar } \\
&= A^2 \sqrt{\frac{\pi \Hbar}{m \omega}} + B^2 \inv{2 \pi} \left( \frac{\pi \Hbar}{ m \omega} \right)^{3/2} \\
&= \sqrt{\frac{\pi \Hbar}{m \omega}} \left( A^2 + B^2 \frac{ \Hbar}{ 2 m \omega} \right) \\
\end{aligned}
\end{equation}

In terms of an arbitrary constant \(B\), this gives

\begin{equation}\label{eqn:harmonicOsc:820}
\begin{aligned}
\end{aligned}
\end{equation}

It is perhaps reasonable to pick \(B=1\).  Regardless, the general solution for this first order polynomial scaled Gaussian is

\begin{equation}\label{eqn:harmonicOsc:840}
\begin{aligned}
\psi(x,t) =
\left(B x \pm \sqrt{\sqrt{\frac{m \omega}{\pi \Hbar}} - B^2 \frac{ \Hbar}{ 2 m \omega} }\right) \exp\left( - \frac{m \omega x^2}{2 \Hbar} - \frac{3 i \omega t }{ 2} \right)
\end{aligned}
\end{equation}

I expected something a bit more simple, without this extra degree of freedom.  That probably has to come from the orthonormality conditions on the
Hermite polynomials.

There is also the matter of the error above when trying to tackle the general case.  Now, there is not anything here that is particularly special in these two
cases, so I had expect the error has to be a plain old algebra problem
hiding in there somewhere.

\subsection{Orthonormality}

What is the inner product for this Hermite polynomial space.  I have seen statements that it was something like

\begin{equation}\label{eqn:harmonicOsc:860}
\begin{aligned}
(f,g) = \int f g e^{-u^2} du
\end{aligned}
\end{equation}

From the problem at hand it is perhaps reasonable to assume that this
mathematical orthogonality relation follows from a plain old integral over
the real line for position of the (phaseless) wave functions.  Let us see
if that is the case for the first two that have been calculated so far.

Let

\begin{equation}\label{eqn:harmonicOsc:880}
\begin{aligned}
f &=
\left(
\frac{ m \omega }{\Hbar \pi}
\right)^{1/4} \exp\left( - \frac{m \omega x^2 }{2 \Hbar} \right) \\
g &=
\left(x + \sqrt{\sqrt{\frac{m \omega}{\pi \Hbar}} - \frac{ \Hbar}{ 2 m \omega} }\right) \exp\left( - \frac{m \omega x^2}{2 \Hbar} \right)
\end{aligned}
\end{equation}

Here \(B=1\) has been used for the first order polynomial wave function.  We want a Graham-Shmidt result based on this

\begin{equation}\label{eqn:harmonicOsc:900}
\begin{aligned}
h = g - (f,g) f
\end{aligned}
\end{equation}

\begin{equation}\label{eqn:harmonicOsc:920}
\begin{aligned}
(f,g)
&= \IIinf f(x) g(x) dx \\
&=
\left(
\frac{ m \omega }{\Hbar \pi}
\right)^{1/4}
\IIinf dx
\left(x + \sqrt{\sqrt{\frac{m \omega}{\pi \Hbar}} - \frac{ \Hbar}{ 2 m \omega} }\right)
\exp\left( - \frac{m \omega x^2 }{\Hbar} \right) \\
&=
\left(
\frac{ m \omega }{\Hbar \pi}
\right)^{1/4}
\sqrt{\sqrt{\frac{m \omega}{\pi \Hbar}} - \frac{ \Hbar}{ 2 m \omega} }
\IIinf dx
\exp\left( - \frac{m \omega x^2 }{\Hbar} \right) \\
&=
\left(
\frac{ m \omega }{\Hbar \pi}
\right)^{1/4}
\sqrt{\sqrt{\frac{m \omega}{\pi \Hbar}} - \frac{ \Hbar}{ 2 m \omega} }
\sqrt{ \frac{\pi \Hbar}{ m \omega} }
\end{aligned}
\end{equation}

So we have

\begin{equation}\label{eqn:harmonicOsc:940}
\begin{aligned}
(f,g) f
&=
\sqrt{\sqrt{\frac{m \omega}{\pi \Hbar}} - \frac{ \Hbar}{ 2 m \omega} }
\exp\left( - \frac{m \omega x^2 }{2 \Hbar} \right) \\
\end{aligned}
\end{equation}

\begin{equation}\label{eqn:harmonicOsc:960}
\begin{aligned}
g - (f,g) f
&= x \exp\left( - \frac{m \omega x^2}{2 \Hbar} \right)
\end{aligned}
\end{equation}

This leaves us with two wave function solutions to the Harmonic oscillator equation, where the spatial parts are orthonormal with respect to an unweighted inner product
over all (linear position) space.  Designating these \(\phi_0\), and \(\phi_1\), we have

\begin{equation}\label{eqn:harmonicOsc:980}
\begin{aligned}
\psi_0 &= \exp\left( - \frac{m \omega x^2}{2 \Hbar} - \frac{i \omega t }{ 2} \right) \\
\psi_1 &= \sqrt{2\pi} \left(\frac{m \omega}{\pi\Hbar}\right)^{3/4} x \exp\left( - \frac{m \omega x^2}{2 \Hbar} - \frac{3 i \omega t }{ 2} \right)
\end{aligned}
\end{equation}

Generalizing, the pattern is clear.  We should have

\begin{equation}\label{eqn:harmonicOsc:1000}
\begin{aligned}
\psi_N &= H_N(x) \exp\left( - \frac{m \omega x^2}{2 \Hbar} - \frac{(2N + 1) i \omega t }{ 2} \right)
\end{aligned}
\end{equation}

Where

\begin{equation}\label{eqn:harmonicOsc:1020}
\begin{aligned}
\delta_{nm} \propto \IIinf H_n(x) H_m(x) \exp\left( - \frac{m \omega x^2}{\Hbar} \right) dx
\end{aligned}
\end{equation}

Where use of the Graham-Schmidt procedure can be used to generate each of these polynomial functions.

FIXME: show that this orthogonality is enough to guarantee a solution.

%\bibliographystyle{plainnat}
%\bibliography{myrefs}

%\end{document}
