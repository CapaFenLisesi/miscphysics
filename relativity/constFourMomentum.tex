%
% Copyright � 2012 Peeter Joot.  All Rights Reserved.
% Licenced as described in the file LICENSE under the root directory of this GIT repository.
%

%
%
%\input{../peeter_prologue.tex}

\chapter{Force free relativistic motion}
\label{chap:constFourMomentum}
%\useCCL
\blogpage{http://sites.google.com/site/peeterjoot/math2009/constFourMomentum.pdf}
\date{Nov 15, 2009}
\revisionInfo{constFourMomentum.tex }

%\beginArtWithToc
\beginArtNoToc

\section{Motivation}

Considering the Euler-Lagrange solutions for the relativistic force free covariant Lagrangian

\begin{equation}\label{eqn:constFourMomentum:qqq1}
\begin{aligned}
\LL &= \inv{2} m \xdot^\mu \xdot_\mu \\
\xdot^\mu &= \frac{d x^\mu}{d \tau},
\end{aligned}
\end{equation}

we get a set of four constant momentum equations

\begin{equation}\label{eqn:constFourMomentum:qqq5}
\begin{aligned}
m \xdot_\mu = m v_\mu(0).
\end{aligned}
\end{equation}

Can we make some sense of this?  While this seems natural enough in comparison to Newtonian physics, we ``just'' add a component when switching to a four vector representation, the \(\gamma\) factors that one may expect are nowhere obvious to be seen.

\section{Guts}

A decomposition into an explicit spacetime split looks like it is the first step along the path to resolves this

\begin{equation}\label{eqn:constFourMomentum:qqq6}
\begin{aligned}
X \equiv (c t, \Bx) = (ct, x^1, x^2, x^3).
\end{aligned}
\end{equation}

Considering first the time component of our equations of motion we have

\begin{equation}\label{eqn:constFourMomentum:qqq7}
\begin{aligned}
m c \frac{dt}{d\tau} = m v_0(0).
\end{aligned}
\end{equation}

Or
\begin{equation}\label{eqn:constFourMomentum:qqq8}
\begin{aligned}
\frac{dt}{d\tau} = \frac{v_0(0)}{c}.
\end{aligned}
\end{equation}

For the spatial components we have

\begin{equation}\label{eqn:constFourMomentum:34}
\begin{aligned}
m \frac{d x_k}{d\tau}
&=
m \frac{d x_k}{dt}  \frac{dt}{d\tau} \\
&=
m \frac{d x_k}{dt}  \frac{v_0(0)}{c}.
\end{aligned}
\end{equation}

With a switch to upper indices, the remaining three equations of motion are then just

\begin{equation}\label{eqn:constFourMomentum:qqq10}
\begin{aligned}
\frac{d x^k}{dt} = c \frac{ v^k(0) }{ v_0(0) }.
\end{aligned}
\end{equation}

Or
\begin{equation}\label{eqn:constFourMomentum:qqq10a}
\begin{aligned}
\frac{d \Bv}{dt} = c \frac{ \Bv(0) }{ v_0(0) }.
\end{aligned}
\end{equation}

The math seems to be saying that relativistically, in the absence of forces, we have constant velocity in our rest frame.  This constant velocity is relative to the initial time component of the four velocity.  This is not what I would have expected from the relativistically corrected Newtons laws in three vector form

\begin{equation}\label{eqn:constFourMomentum:qqq11}
\begin{aligned}
\BF = \frac{d}{dt}\left( \frac{m \Bv}{\sqrt{1 - (\Bv/c)^2}} \right).
\end{aligned}
\end{equation}

In this equation it appears that we should only expect constant velocity in the small speed limit where \(\Bv/c\) can be neglected.  If we, however, take this equation and run with it, where does it lead?  Introducing a vector constant for the spatial momentum \(\Bp(0)\) we have

\begin{equation}\label{eqn:constFourMomentum:qqq12}
\begin{aligned}
\frac{m \Bv}{\sqrt{1 - (\Bv/c)^2}} = \Bp_0.
\end{aligned}
\end{equation}

We can now square and rearrange, yielding

\begin{equation}\label{eqn:constFourMomentum:qqq13}
\begin{aligned}
\frac{\Bv^2}{c^2} = \frac{ {\Bp_0}^2 } { m^2 c^2 + {\Bp_0}^2 }.
\end{aligned}
\end{equation}

With the additional assumption that \(\Bv\) and \(\Bp_0\) are colinear we can take roots (the two could differ by an arbitrary spatial rotation), yielding

\begin{equation}\label{eqn:constFourMomentum:qqq14}
\begin{aligned}
\frac{\Bv}{c} = \frac{ \Bp_0} { \sqrt{m^2 c^2 + {\Bp_0}^2} }.
\end{aligned}
\end{equation}

Just as seen starting from the covariant Lagrangian, we have constant spatial velocity in the absence of external forces.  There was no fundamental inconsistency between the covariant result and the relativistically corrected Newtonian force law.  It was just not initially obvious to me that this was the case.

%\EndArticle
\EndNoBibArticle
