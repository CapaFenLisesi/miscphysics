%
% Copyright � 2012 Peeter Joot.  All Rights Reserved.
% Licenced as described in the file LICENSE under the root directory of this GIT repository.
%

%
%
%\input{../peeter_prologue_print.tex}
%\input{../peeter_prologue_widescreen.tex}

\chapter{Time evolution of some wave functions}
\label{chap:liboff314}
%\useCCL
\blogpage{http://sites.google.com/site/peeterjoot/math2010/liboff314.pdf}
\date{May 23, 2010}
\revisionInfo{liboff314.tex}

%\beginArtWithToc
\beginArtNoToc

\section{Motivation}

In \citep{liboff2003iqm} is problem 3.14, Describe the time evolution of the following wavefunctions

\begin{equation}\label{eqn:liboff314:21}
\begin{aligned}%\label{eqn:liboff314:1}
\psi_1 &= A \sin \omega t \cos k (x + c t) \\
\psi_2 &= A \sin 10^{-5} k x \cos k (x - c t) \\
\psi_3 &= A \cos k ( x - c t ) \sin 10^{-5} k (x - c t)
\end{aligned}
\end{equation}

This is not really a QM problem, but seems worthwhile anyways, because it is not obvious looking at the functions what this is.

\section{\texorpdfstring{\(\psi_3\)}{psi 3}}

Let us start in reverse order with \(\psi_3\), but in a slightly more general form that is less error prone to manipulate.  These wavefunctions can be viewed as superpositions, and expanding out as exponentials temporarily gets us to a form that makes this more obvious.

\begin{equation}\label{eqn:liboff314:41}
\begin{aligned}
\psi
&= A \sin k_1 ( x + v_1 t) \cos k_2 ( x + v_2 t) \\
&= \frac{A}{4i} \left( e^{ i k_1 ( x + v_1 t)} - e^{ -i k_1 ( x + v_1 t)} \right) \left( e^{ i k_2 ( x + v_2 t)} + e^{ -i k_2 ( x + v_2 t)} \right) \\
&= \frac{A}{2} \left(
\sin ((k_1 + k_2) x + (k_1 v_1 + k_2 v_2 ) t)
+ \sin ((k_1 - k_2) x + (k_1 v_1 - k_2 v_2 ) t) \right) \\
&= \frac{A}{2} \left(
\sin \left( (k_1 + k_2) \left(x + \frac{k_1 v_1 + k_2 v_2 }{k_1 + k_2} t\right) \right)
+\sin \left( (k_1 - k_2) \left(x + \frac{k_1 v_1 - k_2 v_2 }{k_1 - k_2} t\right) \right)
\right)
\end{aligned}
\end{equation}

Now the problem is simplified to observing how a wave of the form \(\phi = \sin \kappa (x + v t)\) propagates, or really the interaction of two such waves moving together or against each other, depending on the signs of the constants.  Let us now put in the constants for \(\psi_3\) to get a better feel for it

\begin{equation}\label{eqn:liboff314:61}
\begin{aligned}
\psi_3
&= \frac{A}{2} \left(
\sin \left( 1.00001 k \left(x - c t\right) \right)
-\sin \left( 0.99999 k \left(x - c t\right) \right)
\right)
\end{aligned}
\end{equation}

It was not obvious from the original product of sinusoids form that the question asked about that the resulting wave form stays in phase for its time propagation, but we see that to be the case above.  This really just leaves some thought about the standing wave itself to understand what is happening.  For that, at time 0, we have a destructive interference superposition of two almost identical period standing waves.  That near perfect cancellation will likely leave an envelope, and \href{http://www.wolframalpha.com/input/?i=Plot[Cos[x]+Sin[0.00001+x],+{x,+-317000,+317000}]}{a Mathematica plot} gives a better feel for this waveform.  This in turn will propagate at light speed down the x-axis.  Because \(k_1\) is so small we have a nearly linear, and nearly flat, envelope for the \(\cos k x\), as can be expected near the origin since we have there

\begin{equation}\label{eqn:liboff314:81}
\begin{aligned}
\psi_3(x, 0) \approx A 10^{-5} k x \cos k x
\end{aligned}
\end{equation}

Comparing to \href{http://www.wolframalpha.com/input/?i=Plot[Cos[x]+Sin[0.00001+x],+{x,+-317,+317}]}{a smaller range plot}, one sees that it is necessary to increase the plot range significantly before seeing the oscillatory nature of the envelope.

\section{\texorpdfstring{\(\psi_2\)}{psi 2}}

For \(\psi_2\) we have almost the same wave function, but out sine term has no time variation.  What does this do to the waveform?  Let us see if a sum and difference of angles form sheds some light on that.  We have

\begin{equation}\label{eqn:liboff314:101}
\begin{aligned}
\psi
&= A \sin k_1 x \cos k_2 ( x + v_2 t) \\
&= \frac{A}{4i} \left( e^{ i k_1 x } - e^{ -i k_1 x } \right) \left( e^{ i k_2 ( x + v_2 t)} + e^{ -i k_2 ( x + v_2 t)} \right) \\
&= \frac{A}{2} \left(
\sin ((k_1 + k_2) x + k_2 v_2 t)
+ \sin ((k_1 - k_2) x - k_2 v_2 t) \right) \\
&= \frac{A}{2} \left(
\sin \left( (k_1 + k_2) \left(x + \frac{k_2 v_2 }{k_1 + k_2} t\right) \right)
+\sin \left( (k_1 - k_2) \left(x - \frac{k_2 v_2 }{k_1 - k_2} t\right) \right)
\right)
\end{aligned}
\end{equation}

Specifically for \(k_1 = 10^{-5} k\), and \(k_2 = k\), we have

\begin{equation}\label{eqn:liboff314:121}
\begin{aligned}
\psi_2
&= \frac{A}{2} \left(
\sin \left( 1.00001 k \left(x - \frac{1}{1.00001} c t\right) \right)
-\sin \left( 0.99999 k \left(x + \frac{1}{0.99999 } c t\right) \right)
\right)
\end{aligned}
\end{equation}

Again at \(t=0\) we have a very widely spread envelope with rapid oscillations within it.  There is a very small difference in the rate that the two components waveforms will go out of phase, and each of the component waveforms is moving in the opposite directions.  What does that phase change do to the evolution?  Looking back to the original product of sinusoids form for \(\psi_2\) I believe this will just mean we have the phase shifting with time within the envelope.

\section{\texorpdfstring{\(\psi_1\)}{psi 1}}

Finally for the first wave function we have both of the sinusoid factors with time variation.  What does that expand out to in terms of superposition of fundamental frequencies?

\begin{equation}\label{eqn:liboff314:141}
\begin{aligned}
\psi_1
&= A \sin \omega t \cos k ( x + c t) \\
&= \frac{A}{4i} \left( e^{ i \omega t} - e^{ -i \omega t } \right) \left( e^{ i k ( x + c t)} + e^{ -i k ( x + c t)} \right) \\
&= \frac{A}{2} \left(
\sin ( k (x + c t) + \omega t )
-\sin ( k (x + c t) - \omega t )
\right),
\end{aligned}
\end{equation}

or
\begin{equation}\label{eqn:liboff314:161}
\begin{aligned}
\psi_1
&= \frac{A}{2} \left(
\sin ( k x + ( k c + \omega ) t )
-\sin ( k x + ( k c - \omega) t )
\right)
\end{aligned}
\end{equation}

We have the superposition of two \(\sin k x\) wave forms, destructively interfering with each other, one with phase changing at the angular rate \(k c + \omega\), and the other at the rate \(k c - \omega\).  What is the overall waveform?  It is still not obvious what this is.  I actually have the inclination to try to not treat these analytically, but pull out some graphing software.  Something like the real Mathematica software would be nice since it would allow for the use of sliders to vary parameters and then animate the graphs as time varied.

\EndArticle
