%
% Copyright � 2012 Peeter Joot.  All Rights Reserved.
% Licenced as described in the file LICENSE under the root directory of this GIT repository.
%

%
%
%\documentclass{article}

%\input{../peeters_macros.tex}
%\input{../peeters_macros2.tex}

%\usepackage[bookmarks=true]{hyperref}

%\usepackage{color,cite,graphicx}
   % use colour in the document, put your citations as [1-4]
   % rather than [1,2,3,4] (it looks nicer, and the extended LaTeX2e
   % graphics package.
%\usepackage{latexsym,amssymb,epsf} % do not remember if these are
   % needed, but their inclusion can not do any damage

\chapter{Derive the star distance calculation in the Feynman lectures}
\label{chap:starDistance}
%\author{Peeter Joot \quad peeterjoot@protonmail.com}
\date{ Jan 17, 2000.  starDistance.tex }

%\begin{document}

%\maketitle{}

%\tableofcontents
\section{Motivation}

\imageFigure{../figures/miscphysics/feynman_star}{Triangulating the distance to a star}{fig:feynman_star}{0.4}

Derivation of the "locate Sputnik" formula of Fig 5-5 in \citep{feynman1963flp}, as illustrated in \cref{fig:feynman_star}.

\section{}

Elliptical orbit \(x^2/a^2 + y^2/b^2 = c^2\), with orbital diameter \(2ac = L\).  Angles to the star, measured relative to the Sun are \(\alpha\), and \(\pi - (\alpha+\epsilon)\).  Using the sine rule for triangles,

\begin{displaymath}
  \frac{ l_1}{\sin(\pi - (\alpha + \epsilon))}
= \frac{ l_2 }{\sin{\alpha}}
= \frac{L}{\sin{\epsilon}}
\end{displaymath}

So the lengths to the star from each end of the orbit we have
\begin{eqnarray*}
l_1 & = & L \frac{ \sin(\alpha + \epsilon) }   { \sin{\epsilon}} \\
l_2 & = & L \frac{ \sin{\alpha} }   { \sin{\epsilon}}
\end{eqnarray*}

and so the average length \(\overline{l}\) to the star is
\begin{eqnarray*}
\overline{l} & = & L \frac{ \sin(\alpha + \epsilon) + \sin{\alpha} }{    \sin{\epsilon}} \\
             & = & 2L \frac{ \sin(\alpha + \epsilon/2)\cos(\epsilon/2) }{    \sin{\epsilon}}
\end{eqnarray*}

Since the angular difference \(\epsilon << 0\), the average distance to the star can be approximated as
\begin{displaymath}
\overline{l} = 2L \frac{ \sin{\alpha} }{\epsilon }
\end{displaymath}
%\end{multicols}

%\pagebreak

%\bibliographystyle{plainnat}
%\bibliography{myrefs}

%\end{document}
