%
% Copyright � 2012 Peeter Joot.  All Rights Reserved.
% Licenced as described in the file LICENSE under the root directory of this GIT repository.
%

%
%
%\documentclass{article}      % Specifies the document class

%\input{../peeters_macros.tex}

%\usepackage{color,cite,graphicx}
   % use colour in the document, put your citations as [1-4]
   % rather than [1,2,3,4] (it looks nicer, and the extended LaTeX2e
   % graphics package.
%\usepackage{latexsym,amssymb,epsf} % do not remember if these are
   % needed, but their inclusion can not do any damage


%
% The real thing:
%

                             % The preamble begins here.
\chapter{Pythagoras law}
\label{chap:pythagoras}
%\author{Peeter Joot}         % Declares the author's name.
\date{ March 17, 2008.  pythagoras.tex }

%\begin{document}             % End of preamble and beginning of text.

%\maketitle{}

\section{Length}

To base vector multiplication on length, and examine all the consequences of having done so, it is first necessary to
The geometrical definition of length for vectors generalizes Pythagoras theorem to higher dimensions.

In two dimensions this theorem can be proved with the aid of the following diagram

\imageFigure{../../figures/miscphysics/square_in_square}{Geometrical Proof of Pythagoras Theorem for Right Triangle}{fig:phthagoras}{0.3}

The area of the interior and exterior squares is \(c^2\), and \((a+b)^2\) respectively.  The interior area can also be calculated by subtracting the area of the triangles from the exterior area:

\begin{equation}\label{eqn:pythagoras:20}
(a+b)^2 - 4(ab/2) = a^2 + b^2 + 2ab - 2ab
\end{equation}

Thus proving Pythagoras theorem for the length of the diagonal in a right angle triangle

\begin{equation}\label{eqn:pythagoras:40}
c^2 = a^2 + b^2
\end{equation}

%for a geometrical proof like this one should perhaps show that the inscribed shape is a square ; all the lengths being equal is sufficient IMO.

The length of a vector in three dimensions can be found by repeated application of Pythagoras theorem, as in the following figure

\imageFigure{../../figures/miscphysics/3d_vector_len}{Length of vector in three dimensions}{fig:3dveclen}{0.3}

The vector \(\Be = \Ba + \Bb + \Bc\), where each of the vectors \(\Ba\), \(\Bb\), and \(\Bc\) are mutually perpendicular can be found by first calculating

\begin{equation}\label{eqn:pythagoras:60}
d^2 = a^2 + b^2
\end{equation}

Then

\begin{equation}\label{eqn:pythagoras:80}
e^2 = d^2 + c^2 = a^2 + b^2 + c^2
\end{equation}

This process can be repeated for any number of higher dimensions.  Having calculated the length of a \(N-1\) dimensional vector

\begin{equation}\label{eqn:pythagoras:100}
{L(\Bv)}^2 = \sum_{i=1}^{N-1} {{l_i}^2}
\end{equation}

Once an additional component of length \(l_N\) is added to that vector in a direction mutually perpendicular to all previous components the new length of this vector becomes

\begin{equation}\label{eqn:pythagoras:120}
\sum_{i=1}^{N-1} {{l_i}^2} + l_N^2 = \sum_{i=0}^{N} {{l_i}^2}
\end{equation}

This is what we mean by the geometrical length of a vector.

%\subsection{Pythagoras Law, and the vector product}
%
%We have a rule for vector multiplication when two vectors are collinear.  Comparison to Pythagoras law will provide an additional rule for vector multiplication when the vectors are completely perpendicular.
%

%\end{document}               % End of document.
