%
% Copyright � 2012 Peeter Joot.  All Rights Reserved.
% Licenced as described in the file LICENSE under the root directory of this GIT repository.
%

%
%
%\documentclass{article}

%\input{../peeters_macros.tex}
%\input{../peeters_macros2.tex}

%\usepackage[bookmarks=true]{hyperref}

\chapter{Some notes on DeBroglie paper}
\label{chap:debroglie}
%\author{Peeter Joot \quad peeterjoot@protonmail.com}
\date{ Oct. 25, 2008.  debroglie.tex }

%\begin{document}

%\maketitle{}
%\tableofcontents

\section{Motivation}

The translation of the DeBroglie thesis \citep{AFkracklauerDeBroglie}
appears to have a quite readable introduction to many relativity and
quantum phenomena.  Here I collect additional notes on things that were
not clear to me.

\section{Chapter 2}

\subsection{Equation 2.2.10}

Let

\begin{equation}\label{eqn:debroglie:20}
\begin{aligned}
\LL = k = g_{ij} \qdot^i \qdot^j
\end{aligned}
\end{equation}
\begin{equation}\label{eqn:debroglie:40}
\begin{aligned}
\PD{\qdot^k}{\LL} = 2 g_{ik} \qdot^i \\
\PD{\qdot^k}{\LL} \qdot^k = 2 g_{ik} \qdot^i \qdot^k = 2 K
\end{aligned}
\end{equation}


\subsection{Equation 2.3.3}

A worldline velocity with respect to some parametrization is

\begin{equation}\label{eqn:debroglie:60}
\begin{aligned}
\left(\frac{ds}{d\lambda}\right)^2 = \left( \frac{dx}{d\lambda} \cdot \frac{dx}{d\lambda} \right)^2
&= \left(\frac{dx^4}{d\lambda}\right)^2 - \sum_i \left(\frac{dx^i}{d\lambda}\right)^2
\end{aligned}
\end{equation}

For \(\lambda = s\), we can therefore calculate \(u^4\):

\begin{equation}\label{eqn:debroglie:80}
\begin{aligned}
\left(\frac{ds}{ds}\right)^2 = 1
&= \left(\frac{dx^4}{ds}\right)^2 - \sum_i \left(\frac{dx^i}{ds}\right)^2 \\
&= \left(\frac{dx^4}{ds}\right)^2 - \sum_i \left( \frac{dx^i}{dx^4} \frac{dx^4}{ds} \right)^2 \\
&= \left(\frac{dx^4}{ds}\right)^2 \left( 1 - \sum_i \left( \frac{dx^i}{dx^4} \right)^2 \right) \\
\end{aligned}
\end{equation}

Or
\begin{equation}\label{eqn:debroglie:timearc}
\begin{aligned}
\left(\frac{dx^4}{ds}\right)^2 = \inv{1 - \Bv^2/c^2 } \\
\end{aligned}
\end{equation}

There is a freedom to pick either plus or minus here.  Returning to that later, first
calculate the remainder of this table of derivatives.  Picking \(x^1\) as representative

\begin{equation}\label{eqn:debroglie:100}
\begin{aligned}
1 &= \inv{1 - \Bv^2/c^2 }
- \left(\frac{dx^1}{ds}\right)^2
- \left( \frac{dy}{dx^4} \frac{dx^4}{ds} \right)^2
- \left( \frac{dz}{dx^4} \frac{dx^4}{ds} \right)^2 \\
\left(\frac{dx^1}{ds}\right)^2
&= \frac{\Bv^2/c^2 }{1 - \Bv^2/c^2 }
-\inv{1 - \Bv^2/c^2 } \left( \left( \frac{dy}{dx^4} \right)^2 + \left( \frac{dz}{dx^4} \right)^2 \right) \\
&= \inv{c^2 (1 - \Bv^2/c^2) } \left( \Bv^2 - \left( \frac{dy}{dt} \right)^2 - \left( \frac{dz}{dt} \right)^2 \right) \\
&= \frac{v_x^2/c^2}{1 - \Bv^2/c^2}
\end{aligned}
\end{equation}

Now, express the coordinate vector for the worldline differential in its entirety:

\begin{equation}\label{eqn:debroglie:120}
\begin{aligned}
\frac{dx}{ds} =
\frac{d}{ds}(x^1, x^2, x^3, x^4)
&= \frac{dx^4}{ds} \left( \frac{dx^1}{dx^4}, \frac{dx^2}{dx^4}, \frac{dx^3}{dx^4}, 1 \right) \\
&= \frac{dx^4}{ds} ( v_x/c, v_y/c, v_z/c, 1) \\
\end{aligned}
\end{equation}

This shows that the flexibility to choose a sign for the square roots to obtain \(dx^\mu/ds\) must all match the sign for the \(dx^4/ds\) term.  Considering a particle at rest in the implied frame associated with these coordinates, one has, by \eqnref{eqn:debroglie:timearc}

\begin{equation}\label{eqn:debroglie:140}
\begin{aligned}
\frac{dx}{ds}
&= \pm \left(0, 0, 0, \inv{\sqrt{1 - \Bv^2/c^2}} \right) \\
&= \pm \left(0, 0, 0, 1 \right) \\
\end{aligned}
\end{equation}

If we take positive \(ds\) to measure increase of time in the rest frame, then there is some sense to picking the positive root.  One
would not have to, since there is also a corresponding freedom to bury a sign adjustment in the \(dx_\mu/ds\) derivatives.

%\bibliographystyle{plainnat}
%\bibliography{myrefs}

%\end{document}
