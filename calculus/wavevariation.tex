%
% Copyright � 2012 Peeter Joot.  All Rights Reserved.
% Licenced as described in the file LICENSE under the root directory of this GIT repository.
%

%
%
%\documentclass{article}

%\input{../peeters_macros.tex}
%\input{../peeters_macros2.tex}
%\usepackage{listings}
%\usepackage{txfonts} % for ointctr... (also appears to make "prettier" \int and \sum's)
%\usepackage[bookmarks=true]{hyperref}

%\usepackage{color,cite,graphicx}
   % use colour in the document, put your citations as [1-4]
   % rather than [1,2,3,4] (it looks nicer, and the extended LaTeX2e
   % graphics package.
%\usepackage{latexsym,amssymb,epsf} % do not remember if these are
   % needed, but their inclusion can not do any damage


\chapter{Simple minded variation of one dimensional wave equation Lagrangian}
\label{chap:wavevariation}
%\author{Peeter Joot \quad peeterjoot@protonmail.com }
\date{ April 27, 2009.  wavevariation.tex }

%\begin{document}

%\maketitle{}
%\tableofcontents
%\section{}

From the action

\begin{equation}\label{eqn:wavevariation:20}
\begin{aligned}
S_\eta = \int dx dt \inv{2} \left( \left(\PD{x}{\eta}\right)^2 - \inv{v^2} \left(\PD{t}{\eta}\right)^2 \right)
\end{aligned}
\end{equation}

We can vary the field \(\eta = \psi + \epsilon\), where \(\psi\) is the field variable to be determined, and \(\epsilon\) is the field variable allowed to vary within the volume of integration.

Forming the difference, subtracts off the ``constant'' parts of the action due only to the optimal field variable \(\psi\)

\begin{equation}\label{eqn:wavevariation:40}
\begin{aligned}
S_{\psi + \epsilon} - S_\psi
&=
\int dx dt \inv{2} \left( \left(\PD{x}{(\psi + \epsilon)}\right)^2 - \inv{v^2} \left(\PD{t}{(\psi + \epsilon)}\right)^2 \right)
-\int dx dt \inv{2} \left( \left(\PD{x}{\psi}\right)^2 - \inv{v^2} \left(\PD{t}{\psi}\right)^2 \right) \\
&=
\int dx dt \left( \PD{x}{\psi}\PD{x}{\epsilon} - \PD{t}{\psi}\PD{t}{\epsilon} \right)
+ \int dx dt \inv{2} \left( \left(\PD{x}{\epsilon }\right)^2 - \inv{v^2} \left(\PD{t}{\epsilon }\right)^2 \right) \\
\end{aligned}
\end{equation}

Now integrating by parts

\begin{equation}\label{eqn:wavevariation:60}
\begin{aligned}
S_{\psi + \epsilon} - S_\psi
&=
\int dx dt \left( \inv{v^2} \PDSq{t}{\psi} -\PDSq{x}{\psi} \right) {\epsilon}
+ \int dx dt \inv{2} \left( -\PDSq{x}{\epsilon } + \inv{v^2} \PDSq{t}{\epsilon } \right) \epsilon \\
\end{aligned}
\end{equation}

Roughly speaking the terms that are quadratic in \(\epsilon\) can be discarded as small, and if the remaining differential
is to be zero for all \(\epsilon\), we are left with the wave equation

\begin{equation}\label{eqn:wavevariation:80}
\begin{aligned}
\inv{v^2} \PDSq{t}{\psi} -\PDSq{x}{\psi} = 0
\end{aligned}
\end{equation}

with solutions
%(like your exponential)
of the form

\begin{equation}\label{eqn:wavevariation:100}
\begin{aligned}
\psi = f(x \pm vt)
\end{aligned}
\end{equation}

%\bibliographystyle{plainnat}
%\bibliography{myrefs}

%\end{document}
