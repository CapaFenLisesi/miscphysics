%
% Copyright � 2012 Peeter Joot.  All Rights Reserved.
% Licenced as described in the file LICENSE under the root directory of this GIT repository.
%

%
%
%\documentclass{article}

%\input{../peeters_macros.tex}
%\input{../peeters_macros2.tex}

%\usepackage[bookmarks=true]{hyperref}

%\usepackage{color,cite,graphicx}
   % use colour in the document, put your citations as [1-4]
   % rather than [1,2,3,4] (it looks nicer, and the extended LaTeX2e
   % graphics package.
%\usepackage{latexsym,amssymb,epsf} % do not remember if these are
   % needed, but their inclusion can not do any damage


\chapter{Simple Wave Packet Examples}
\label{chap:wavepacket}
%\author{Peeter Joot \quad peeterjoot@protonmail.com}
\date{ Feb 16, 2009.  wavepacket.tex }

%\begin{document}

%\maketitle{}
%\tableofcontents
\section{Wave packet examples}

In \citep{bohm1989qt}, chapter 3, a couple explicit wave packet examples are given.
Some of this is a little hard to follow in the small set font of the text, and some details are missing.  Here the integrals are performed
in detail.

\subsection{Unweighted example. Equation 1}

Summing plane waves over a small range of frequencies

\begin{equation}\label{eqn:wavepacket:20}
\begin{aligned}
E_z(x)
&= \int_{k_0 -\Delta k}^{k_0 -\Delta k} dk e^{ik(x-x_0)} \\
&= {\left. \frac{e^{ik(x-x_0)}}{i(x-x_0)} \right\vert}_{k_0 -\Delta k}^{k_0 -\Delta k}  \\
&= \frac{e^{i(k_0 + \Delta k)(x-x_0)}}{i(x-x_0)} -\frac{e^{i(k_0 - \Delta k)(x-x_0)}}{i(x-x_0)} \\
&= 2 {e^{i k_0 (x-x_0)}}
 \frac{{e^{i \Delta k(x-x_0)}} -{e^{-i \Delta k(x-x_0)}}}{2 i (x-x_0)} \\
\end{aligned}
\end{equation}

Which is Bohm's equation 1.
\begin{equation}\label{eqn:wavepacket:40}
\begin{aligned}
E_z(x) &= 2 {e^{i k_0 (x-x_0)}} \frac{\sin( \Delta k (x-x_0))}{ x-x_0}
\end{aligned}
\end{equation}

\subsection{Gaussian weighting example. Equation 4}

Next example, also chosen for simplicity, uses a Gaussian weighting function

\begin{equation}\label{eqn:wavepacket:60}
\begin{aligned}
\psi
&= \int_\infty^\infty
\exp\left( - \frac{(k-k_0)^2}{2(\Delta k)^2} \right)
\exp\left( i k(x-x_0) \right) dk \\
\end{aligned}
\end{equation}

Next, is the sneaky/clever step of adding and subtracting \(i k_0 (x-x_0) - (x-x_0)^2 (\Delta k)^2/2\) to the exponentials, which gives

\begin{equation}\label{eqn:wavepacket:80}
\begin{aligned}
\psi
&=
\exp\left( i k_0 (x-x_0) - \frac{(x-x_0)^2}{2} (\Delta k)^2 \right) \\
&\quad \int_\infty^\infty
\exp\left(
- \frac{(k-k_0)^2}{2(\Delta k)^2}
+i (k-k_0)(x-x_0)
+\frac{(x-x_0)^2}{2} (\Delta k)^2
\right)
dk \\
\end{aligned}
\end{equation}

Looking at just the remaining integral part, say \(I\), we have
\begin{equation}\label{eqn:wavepacket:100}
\begin{aligned}
I &=
\int_\infty^\infty
\exp\left( \inv{2} \left(
i^2 \frac{(k-k_0)^2}{(\Delta k)^2}
+2i (k-k_0)(x-x_0)
+{(x-x_0)^2}{} (\Delta k)^2
\right)
\right)
dk \\
&=
\int_\infty^\infty
\exp\left( \inv{2} \left(
i \frac{k-k_0}{\Delta k}
+(x-x_0) \Delta k
\right)^2
\right)
dk \\
&=
\int_\infty^\infty
\exp\left( -\inv{2} \left(
\frac{k-k_0}{\Delta k}
-i(x-x_0) \Delta k
\right)^2
\right)
dk \\
\end{aligned}
\end{equation}

A change of variables \(u = (k-k_0)/{\Delta k} -i(x-x_0) \Delta k\), \(du = dk/{\Delta k}\) gives

\begin{equation}\label{eqn:wavepacket:120}
\begin{aligned}
I
&= \Delta k \int_\infty^\infty e^{ -u^2/2 } du \\
&= \sqrt{2 \pi} \Delta k
\end{aligned}
\end{equation}

Which gives

\begin{equation}\label{eqn:wavepacket:140}
\begin{aligned}
\psi(x)
&= \sqrt{2 \pi} \Delta k
\exp\left( i k_0 (x-x_0) - \frac{(x-x_0)^2}{2} (\Delta k)^2 \right) \\
\end{aligned}
\end{equation}

Off by a factor of \(\sqrt{\Delta k}\) compared to the text?  Typo in the book or a mistake above?

\subsection{Gaussian weighting with angular velocity and acceleration}

Next covered is a wave packet where the angular frequency is a function of the wave number, as in equation 8

\begin{equation}\label{eqn:wavepacket:160}
\begin{aligned}
E(x,t) &= \IIinf f(k - k_0) \exp\left( i k(x-x_0) - i\omega(k) t\right) dk
\end{aligned}
\end{equation}

With a Taylor series expansion of the frequency about \(k_0\)

\begin{equation}\label{eqn:wavepacket:180}
\begin{aligned}
\omega(k)
&= \omega(k_0) + \left(\PD{k}{\omega}\right)_{k=k_0} (k-k_0) + \left(\PDSq{k}{\omega}\right)_{k=k_0} \frac{(k-k_0)^2}{2} + \cdots \\
&= \omega_0 + V_g (k-k_0) + \alpha \frac{(k-k_0)^2}{2} + \cdots \\
\end{aligned}
\end{equation}

Here \(V_g\), and \(\alpha\) are the group velocity and accelerations respectively.  Now, I had never seen the group velocity expressed
this way, which seems a particularly simple way of putting it.
The example of how \(\omega = 2 \pi c/\lambda n(\lambda)\) can vary with index of refraction and wavelength is also nice.  I
imagine a light wave going through a water oil air transition and the angular frequency in each region causing dispersion and
reflection and path alteration effects.

Back to the second order approximation of the frequency, substituting back into the wave packet integral one has

\begin{equation}\label{eqn:wavepacket:200}
\begin{aligned}
E(x,t) &\approx \IIinf f(k - k_0) \exp\left( i k(x-x_0) - i
\left(\omega_0 + V_g (k-k_0) + \alpha \frac{(k-k_0)^2}{2} \right)
t\right) dk
\end{aligned}
\end{equation}

With \(\kappa = k - k_0\), \(\Delta x = x -x_0\) and the Gaussian weighting \(f(\kappa) = e^{-\kappa^2/2(\Delta k)^2}\) this is

\begin{equation}\label{eqn:wavepacket:220}
\begin{aligned}
\exp&( i k_0 (x-x_0) )
\IIinf
\exp\left( -\frac{\kappa^2}{2(\Delta k)^2} +i \kappa (x-x_0)
- i \left(\omega_0 + V_g \kappa + \alpha \frac{\kappa^2}{2} \right) t\right) dk \\
&=
\exp( i k_0 \Delta x -i \omega_0 t)
\IIinf
\exp\left(
i \kappa ( \Delta x - V_g t )
-\frac{\kappa^2}{2} \left(\inv{(\Delta k)^2} + i \alpha t \right)
\right) d\kappa \\
\end{aligned}
\end{equation}

With \(a = \Delta x - V_g t\) and \(b = \inv{(\Delta k)^2} + i \alpha t\), the exponential in the integral takes the form
\begin{equation}\label{eqn:wavepacket:240}
\begin{aligned}
\exp\left( i \kappa a - \kappa^2 \frac{b}{2} \right)
&= \exp\left( - \frac{b}{2}\left(-2 i \kappa \frac{a}{b} + \kappa^2 \right) \right)  \\
&= \exp\left( - \frac{b}{2}\left( \kappa - i \frac{a}{b} \right)^2 + \frac{b}{2}\left(\frac{ia}{b}\right)^2 \right)  \\
&= \exp\left( - \frac{b}{2}\left( \kappa - i \frac{a}{b} \right)^2 - \frac{a^2}{2b} \right)  \\
\end{aligned}
\end{equation}

Our wave packet is now

\begin{equation}\label{eqn:wavepacket:260}
\begin{aligned}
E(x,t)
&= \exp\left( i k_0 \Delta x -i \omega_0 t - \frac{( \Delta x - V_g t)^2 (\Delta k)^2}{2(1 + i \alpha t (\Delta k)^2)} \right)
\IIinf
\exp\left( - \frac{b}{2}\left( \kappa - i \frac{a}{b} \right)^2 \right)
d\kappa \\
\end{aligned}
\end{equation}

A change of vars \(u = \sqrt{b}(\kappa - ia/b)\) gives

\begin{equation}\label{eqn:wavepacket:280}
\begin{aligned}
E(x,t)
&= \exp\left( i k_0 \Delta x -i \omega_0 t - \frac{( \Delta x - V_g t)^2 (\Delta k)^2}{2(1 + i \alpha t (\Delta k)^2)} \right)
\frac{\Delta k}{\sqrt{1 + i \alpha t (\Delta k)^2}}
\IIinf \exp\left( - \frac{u^2}{2} \right) d\kappa \\
&= \exp\left( i k_0 \Delta x -i \omega_0 t - \frac{( \Delta x - V_g t)^2 (\Delta k)^2}{2(1 + i \alpha t (\Delta k)^2)} \right)
\frac{\sqrt{2\pi}\Delta k}{\sqrt{1 + i \alpha t (\Delta k)^2}} \\
&=
\frac{\sqrt{2\pi}\Delta k}{\sqrt{1 + i \alpha t (\Delta k)^2}}
\exp\left( i \left(k_0 \Delta x - \omega_0 t
+
\alpha t (\Delta k)^2 \frac{( \Delta x - V_g t)^2 (\Delta k)^2}{2(1 + \alpha^2 t^2 (\Delta k)^4)}
\right) \right) \times \\
&\exp\left(
-
\frac{( \Delta x - V_g t)^2 (\Delta k)^2}{2(1 + \alpha^2 t^2 (\Delta k)^4)}
\right) \\
\end{aligned}
\end{equation}

This is consistent with the result in the text and with \(\alpha = 0\) confirms that equation 4 did have a typo (irrelevant to the
intensity discussion).

%gnuplot> set xlabel "x-axis"
%gnuplot> set ylabel "t-axis"
%gnuplot> splot [x=75:100] [t=-10:10] exp(-10*(x-10*t)**2/(1+10*t*t*100))
%gnuplot> splot [x=75:100] [t=-10:100] exp(-10*(x-10*t)**2/(1+10*t*t*100))
%gnuplot> splot [x=75:100] [t=-100:100] exp(-10*(x-10*t)**2/(1+10*t*t*100))
%gnuplot> splot [x=75:100] [t=-100:100] exp(-10*(x-1*t)**2/(1+10*t*t*100))
%gnuplot> splot [x=75:100] [t=-100:100] exp(-10*(x-1*t)**2/(1+10*t*t*1000))
%gnuplot> splot [x=75:100] [t=-100:100] exp(-10*(x-1*t)**2/(1+10*t*t*10000))
%gnuplot> splot [x=75:100] [t=-100:100] exp(-10*(x-1*t)**2/(1+10*t*t*2))
%gnuplot> splot [x=-75:100] [t=-100:100] exp(-10*(x-1*t)**2/(1+10*t*t*2))

%\bibliographystyle{plainnat}
%\bibliography{myrefs}

%\end{document}
