%
% Copyright � 2012 Peeter Joot.  All Rights Reserved.
% Licenced as described in the file LICENSE under the root directory of this GIT repository.
%

%
%
\chapter{Integer binomial theorem induction, the easy dumb way}
\label{chap:binomial}
\date{ March 26, 2009.   binomial.tex }

\section{Motivation}

While working a problem with an induction requirement similar to but more
complicated than the binomial theorem, I stepped back and thought I had try
this as an easier first step.  Had some trouble doing it, until I tried it
explicitly with for the power of three case.  Ironically, working it
out for an explicit index takes the abstraction out of the problem, and
generalizing further really only requires a search and replace.

\subsection{Do it}

Want to prove

\begin{equation}\label{eqn:binomial:20}
\begin{aligned}
(t + x)^k = \sum_{m=0}^k \binom{k}{m} t^m x^{k-m}
\end{aligned}
\end{equation}

where

\begin{equation}\label{eqn:binomial:40}
\begin{aligned}
\binom{k}{m} = \frac{k!}{(k-m)!m!}
\end{aligned}
\end{equation}

In particular want to prove this for the \(k+1\) case, given \(k\).

\subsubsection{step with k+1 = 3}

Three is not actually the best way to start since it is almost too trivial, but
one gets the idea easily by doing it.

\begin{equation}\label{eqn:binomial:60}
\begin{aligned}
(t + x)^3
&= (t + x)(t + x)^2  \\
&= (t + x)\sum_{m=0}^2 \binom{2}{m} t^m x^{2-m} \\
&=
\sum_{m=0}^2 \binom{2}{m} t^{m+1} x^{2-m}
+ \sum_{m=0}^2 \binom{2}{m} t^{m} x^{2-m + 1} \\
&=
\sum_{m=1}^{2 + 1} \binom{2}{m-1} t^{m} x^{2 - m + 1}
+ \sum_{m=0}^2 \binom{2}{m} t^{m} x^{2-m + 1} \\
\end{aligned}
\end{equation}

Now pull the lowest and highest order terms out of the sums, and group the
remaining bits.

\begin{equation}\label{eqn:binomial:80}
\begin{aligned}
(t + x)^3
&=
 \binom{2}{0} t^{0} x^{2 - 0 + 1} \\
&+ \sum_{m=1}^{2} \left( \binom{2}{m} + \binom{2}{m-1} \right) t^{m} x^{2 - m + 1} \\
&+ \binom{2}{2 + 1 -1} t^{2 + 1} x^{2 - (2 + 1) + 1}
\\
\end{aligned}
\end{equation}

Now, observe that in all the steps above everywhere if all places that \(2\), a nice easy to think with and concrete number, we could have used some
abstract index.

\subsubsection{step with k+1}

A straight text search and replace on \(2\) with \(k\) gives

\begin{equation}\label{eqn:binomial:100}
\begin{aligned}
(t + x)^{k+1}
&= (t + x)(t + x)^k  \\
&= (t + x)\sum_{m=0}^k \binom{k}{m} t^m x^{k-m} \\
&=
\sum_{m=0}^k \binom{k}{m} t^{m+1} x^{k-m}
+ \sum_{m=0}^k \binom{k}{m} t^{m} x^{k-m + 1} \\
&=
\sum_{m=1}^{k + 1} \binom{k}{m-1} t^{m} x^{k - m + 1}
+ \sum_{m=0}^k \binom{k}{m} t^{m} x^{k-m + 1} \\
&=
 \binom{k}{0} t^{0} x^{k - 0 + 1} \\
&+ \sum_{m=1}^{k} \left( \binom{k}{m} + \binom{k}{m-1} \right) t^{m} x^{k - m + 1} \\
&+ \binom{k}{k + 1 -1} t^{k + 1} x^{k - (k + 1) + 1} \\
\end{aligned}
\end{equation}

This is EXACTLY the same as above with \(k=2\) but it sure looks more complicated with an an abstract index.

To finish off we need a couple observations, that
\(\binom{k}{0} = 1 = \binom{k+1}{0}\), and \(\binom{k}{k} = 1 = \binom{k+1}{k+1}\).  This leaves us with

\begin{equation}\label{eqn:binomial:120}
\begin{aligned}
(t + x)^{k+1}
&= \binom{k+1}{0} t^{0} x^{k + 1} \\
&+ \sum_{m=1}^{k} \left( \binom{k}{m} + \binom{k}{m-1} \right) t^{m} x^{k - m + 1} \\
&+ \binom{k + 1}{k + 1} t^{k + 1} x^{0} \\
\end{aligned}
\end{equation}

So if we can show
\begin{equation}\label{eqn:binomial:140}
\begin{aligned}
\binom{k}{m} + \binom{k}{m-1} = \binom{k+1}{m}
\end{aligned}
\end{equation}

then we would have

\begin{equation}\label{eqn:binomial:160}
\begin{aligned}
(t + x)^{k+1}
&= \sum_{m=0}^{k+1} \binom{k+1}{m} t^{m} x^{k + 1 - m}
\end{aligned}
\end{equation}

which is what was desired.

\subsubsection{That last little piece}

To prove that last little piece, let us do it again the dumb way, and let a regular expression \(s/2/k/g\) in vim do the hard work.

\begin{equation}\label{eqn:binomial:180}
\begin{aligned}
\binom{2}{m} + \binom{2}{m-1}
&=
\frac{2!}{(2-m)!m!}
+ \frac{2!}{(2-m + 1)!(m-1)!} \\
&=
\frac{2!}{(2-m)!m(m-1)!}
+ \frac{2!}{(3-m)!(m-1)!} \\
&=
\frac{2!}{(2-m)!m(m-1)!}
+ \frac{2!}{(3-m)(2-m)!(m-1)!} \\
&=
\frac{2!}{(2-m)!(m-1)!} \left( \inv{m} + \inv{3-m}\right) \\
&=
\frac{2!}{(2-m)!(m-1)!} \frac{3 -m + m }{m(3-m)} \\
&=
\frac{3!}{(3-m)!(m)!} \\
\end{aligned}
\end{equation}

So, generalizing the easy way with \(s/3/k+1/g\), we have

\begin{equation}\label{eqn:binomial:200}
\begin{aligned}
\binom{k}{m} + \binom{k}{m-1}
&=
\frac{k!}{(k-m)!m!}
+ \frac{k!}{(k-m + 1)!(m-1)!} \\
&=
\frac{k!}{(k-m)!m(m-1)!}
+ \frac{k!}{((k+1)-m)!(m-1)!} \\
&=
\frac{k!}{(k-m)!m(m-1)!}
+ \frac{k!}{((k+1)-m)(k-m)!(m-1)!} \\
&=
\frac{k!}{(k-m)!(m-1)!} \left( \inv{m} + \inv{(k+1)-m}\right) \\
&=
\frac{k!}{(k-m)!(m-1)!} \frac{(k+1) -m + m }{m((k+1)-m)} \\
&=
\frac{(k+1)!}{((k+1)-m)!(m)!} \\
\end{aligned}
\end{equation}

Every step is EXACTLY the same as with \(k=2\), the only differences were straight text substitution.  That leaves us with

\begin{equation}\label{eqn:binomial:220}
\begin{aligned}
\binom{k}{m} + \binom{k}{m-1}
&=
\binom{k+1}{m}
\end{aligned}
\end{equation}

That is all we needed to complete the proof.

I think this is a superior way to do inductive proofs.  Just do the absolute easiest case and do it with a number that is easy to think with.  Search and replace in an editor does all the bits that would make you look clever if you were to leave off the fact that you were really only doing the easy version!

\section{The original problem}

The original problem that I was trying to solve was problem (3.FIXME) from \citep{byron1992mca}.  For an n-th degree polynomial \(f(t)\)
show, with \(D = d/dt\), that the operator

\begin{equation}\label{eqn:binomial:240}
\begin{aligned}
T = I + \frac{xD}{1!} + \cdots \frac{(xD)^n}{n!}
\end{aligned}
\end{equation}

serves to shift the function \(T f(t) = f(t + x)\).

\subsection{Inductive setup}

For the lower order cases, this is easy to show that the property holds, and for the zeroth degree and first degree polynomial cases we have
respectively for \(f(t) = \sum a_i t^i\)

\begin{equation}\label{eqn:binomial:260}
\begin{aligned}
T_0 a_0 t^0 = a_0 (t + x)^0
\end{aligned}
\end{equation}

\begin{equation}\label{eqn:binomial:280}
\begin{aligned}
T_1 (a_0 t^0 + a_1 t^1) = (a_0 t^0 + a_1 t^1) + x a_1 = a_0(t + x)^0 + a_1(t + x)^1
\end{aligned}
\end{equation}

\subsection{Induction step}

Now, given this result for

\begin{equation}\label{eqn:binomial:300}
\begin{aligned}
T_k &= \sum_{m=0}^k \frac{(xD)^m}{m!} \\
f_k(t) &= \sum_{m=0}^k a_m t^m
\end{aligned}
\end{equation}

Let us prove it for \(T_{k+1}\).  That is

\begin{equation}\label{eqn:binomial:320}
\begin{aligned}
T_{k+1} f_{k+1}(t)
&=
\left( T_{k} + \frac{(xD)^{k+1}}{(k+1)!} \right) \left( f_{k}(t) + a_{k+1}t^{k+1} \right) \\
&=
T_{k} f_{k}(t)
+T_{k} a_{k+1}t^{k+1}
+\frac{(xD)^{k+1}}{(k+1)!} f_{k}(t)
+\frac{(xD)^{k+1}}{(k+1)!} a_{k+1}t^{k+1}
\\
&=
f_{k}(x + t)
+T_{k} a_{k+1}t^{k+1}
+\frac{(xD)^{k+1}}{(k+1)!} a_{k+1}t^{k+1}
\\
\end{aligned}
\end{equation}

Okay, that was not so bad to setup.  What remains is to compute, for \(m \le k+1\),

\begin{equation}\label{eqn:binomial:340}
\begin{aligned}
\frac{(xD)^{m}}{m!} t^{k+1}
\end{aligned}
\end{equation}

Let us try this with search and replace induction using \(m=3\), and \(k=7\) to start.  This gives

\begin{equation}\label{eqn:binomial:360}
\begin{aligned}
\frac{(xD)^{3}}{3!} t^{7+1}
&=
\frac{(xD)^{3-1}}{3!} xD t^{7+1} \\
&=
\frac{(xD)^{3-1}}{3!} x^1 (7+1) t^{7} \\
&=
\frac{(xD)^{3-2}}{3!} x^2 (7+1)(7) t^{7-1} \\
&=
\hdots \\
&=
\frac{(xD)^{3-3}}{3!} x^3 (7+1)(7) \cdots (7-(3-2)) t^{7-(3-1)} \\
&=
\frac{(xD)^{3-3}}{3!} x^3 \frac{(7+1)!}{(7-(3-2)-1)!} t^{7-(3-1)} \\
\end{aligned}
\end{equation}

Search and replace on three and seven, all steps remain valid and IDENTICAL, but are ``generalized''

\begin{equation}\label{eqn:binomial:380}
\begin{aligned}
\frac{(xD)^{m}}{m!} t^{k+1}
&=
\frac{(xD)^{m-1}}{m!} xD t^{k+1} \\
&=
\frac{(xD)^{m-1}}{m!} x^1 (k+1) t^{k} \\
&=
\frac{(xD)^{m-2}}{m!} x^2 (k+1)(k) t^{k-1} \\
&=
\hdots \\
&=
\frac{(xD)^{m-m}}{m!} x^m (k+1)(k) \cdots (k-(m-2)) t^{k-(m-1)} \\
&=
\frac{(xD)^{m-m}}{m!} x^m \frac{(k+1)!}{(k-(m-2)-1)!} t^{k-(m-1)} \\
\end{aligned}
\end{equation}

And we are left with

\begin{equation}\label{eqn:binomial:400}
\begin{aligned}
\frac{(xD)^{m}}{m!} t^{k+1}
&=
\frac{(k+1)!}{(k + 1 - m)!m!} x^m t^{k + 1 -m} \\
&=
\binom{k+1}{m} x^m t^{k + 1 -m} \\
\end{aligned}
\end{equation}

Wrapping up, we now have

\begin{equation}\label{eqn:binomial:420}
\begin{aligned}
T_{k+1} f_{k+1}(t)
&=
\left( T_{k} + \frac{(xD)^{k+1}}{(k+1)!} \right) \left( f_{k}(t) + a_{k+1}t^{k+1} \right) \\
&=
f_{k}(x + t)
+ a_{k+1}
\left( T_{k} t^{k+1} +\frac{(xD)^{k+1}}{(k+1)!} t^{k+1} \right)
\\
&=
f_{k}(x + t)
+ a_{k+1}
\left(
\sum_{m=0}^k \frac{(xD)^m}{m!} t^{k+1} +\frac{(xD)^{k+1}}{(k+1)!} t^{k+1} \right)
\\
&=
f_{k}(x + t)
+ a_{k+1} \sum_{m=0}^{k+1} \frac{(xD)^m}{m!} t^{k+1}
\\
&=
f_{k}(x + t)
+ a_{k+1} \sum_{m=0}^{k+1} \binom{k+1}{m} x^m t^{k + 1 -m} \\
&=
f_{k}(x + t) + a_{k+1} (x + t)^{k+1} \\
&=
f_{k+1}(x + t)
\end{aligned}
\end{equation}

As desired.
