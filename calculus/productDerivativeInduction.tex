%
% Copyright � 2012 Peeter Joot.  All Rights Reserved.
% Licenced as described in the file LICENSE under the root directory of this GIT repository.
%

%
%
%\input{../peeter_prologue_print.tex}
%\input{../peeter_prologue_widescreen.tex}

\chapter{Induction for Nth derivative of a product}
\label{chap:productDerivativeInduction}
%\useCCL
\blogpage{http://sites.google.com/site/peeterjoot2/math2012/productDerivativeInduction.pdf}
\date{Jan 1, 2012}
\revisionInfo{productDerivativeInduction.tex}

%\beginArtWithToc
\beginArtNoToc

\section{Motivation}

I had use for the result \(D^n( a b)\), where \(D = d/dx\), and felt like doing the induction.

\section{Guts}

Let us look at the first few cases, starting with \(n = 1\).  That is just the chain rule

\begin{equation}\label{eqn:productDerivativeInduction:10}
D( a b) = (D a) b + a (D b).
\end{equation}

For \(n = 2\) we have

\begin{equation}\label{eqn:productDerivativeInduction:70}
\begin{aligned}
D^2( a b)
&= D ((D a) b + a (D b)) \\
&=
(D^2 a) b + (D a) (D b)
(D a) (D b) + a (D^2 b) \\
&=
(D^2 a) b + 2 (D a) (D b) + a (D^2 b) \\
\end{aligned}
\end{equation}

The induction hypothesis is clearly of the form of a binomial product

\begin{equation}\label{eqn:productDerivativeInduction:30}
D^n (a b) =
\sum_{k=0}^n \binom{n}{k} (D^k a)(D^{n-k} b).
\end{equation}

Let us try the induction

\begin{equation}\label{eqn:productDerivativeInduction:90}
\begin{aligned}
D &\sum_{k=0}^n \binom{n}{k} (D^k a)(D^{n-k} b) \\
&=
\sum_{k=0}^n \binom{n}{k}
(D^{k+1} a)(D^{n-k} b)
+ (D^{k} a)(D^{n+1-k} b) \\
&=
(D^{n+1} a) b
+
\sum_{k=0}^{n-1} \binom{n}{k}
(D^{k+1} a)(D^{n-k} b)
+
(D^{0} a)(D^{n+1} b)
+
\sum_{k=1}^n \binom{n}{k}
(D^{k} a)(D^{n+1-k} b)
\\
&=
(D^{n+1} a) b
+
(D^{0} a)(D^{n+1} b)
+
\sum_{j=1}^{n} \binom{n}{j-1}
(D^{j} a)(D^{n+1-j} b)
+
\sum_{k=1}^n \binom{n}{k}
(D^{k} a)(D^{n+1-k} b)
\\
&=
(D^{n+1} a) b
+
(D^{0} a)(D^{n+1} b)
+
\sum_{j=1}^{n} \left(\binom{n}{j-1} + \binom{n}{j} \right)
(D^{j} a)(D^{n+1-j} b) \\
\end{aligned}
\end{equation}

Simplification of the binomial coefficients should finish the job

\begin{equation}\label{eqn:productDerivativeInduction:110}
\begin{aligned}
\binom{n}{j-1} + \binom{n}{j}
&=
\frac{n!}{(j-1)!(n+1-j)!}
+\frac{n!}{j!(n-j)!} \\
&=
\frac{n!}{(j-1)!(n-j)!} \left(
\inv{n+1-j} + \inv{j}
\right) \\
&=
\frac{(n+1)!}{j!(n+1-j)!} \\
&=
\binom{n+1}{j}
\end{aligned}
\end{equation}

Assembling the terms we have
\begin{equation}\label{eqn:productDerivativeInduction:50}
D^{n+1} (a b)
=
\sum_{j=0}^{n+1} \binom{n+1}{j}
(D^{j} a)(D^{n+1-j} b),
\end{equation}

which completes the induction.

%\EndArticle
\EndNoBibArticle
