%
% Copyright � 2012 Peeter Joot.  All Rights Reserved.
% Licenced as described in the file LICENSE under the root directory of this GIT repository.
%

%
%
%\input{../peeter_prologue_print.tex}
%\input{../peeter_prologue_widescreen.tex}

\chapter{Effect of sinusoid operators}
\label{chap:liboff319}
%\useCCL
\blogpage{http://sites.google.com/site/peeterjoot/math2010/liboff319.pdf}
\date{May 23, 2010}
\revisionInfo{liboff319.tex}

%\beginArtWithToc
\beginArtNoToc

\section{Problem 3.19}

\citep{liboff2003iqm} problem 3.19 is

What is the effect of operating on an arbitrary function \(f(x)\) with the following two operators

\begin{subequations}
\label{eqn:liboff319:1}
\begin{equation}\label{eqn:liboff319:22}
\begin{aligned}
\hat{O}_1 &\equiv \partial^2/\partial x^2 - 1
+ sin^2 (\partial^3/\partial x^3)
+ cos^2 (\partial^3/\partial x^3) \\
\hat{O}_2 &\equiv
+ cos (2 \partial/\partial x)
+ sin^2 (\partial/\partial x)
+ \int_a^b dx
\end{aligned}
\end{equation}
\end{subequations}

On the surface with \(\sin^2 y + \cos^2 y = 1\) and \(\cos 2y + 2 \sin^2 y = 1\) it appears that we have just
\begin{subequations}
\label{eqn:liboff319:2}
\begin{equation}\label{eqn:liboff319:42}
\begin{aligned}
\hat{O}_1 &\equiv \partial^2/\partial x^2  \\
\hat{O}_2 &\equiv 1 + \int_a^b dx
\end{aligned}
\end{equation}
\end{subequations}

but it this justified when the sinusoids are functions of operators?  Let us look at the first case.  For some operator \(\hat{f}\) we have

\begin{equation}\label{eqn:liboff319:62}
\begin{aligned}
\sin^2 \hat{f} + \cos^2 \hat{f}
&=
-\inv{4} \left(
e^{i\hat{f}} -e^{-i\hat{f}}
\right)
\left(
e^{i\hat{f}} -e^{-i\hat{f}}
\right)
+\inv{4} \left(
e^{i\hat{f}} +e^{-i\hat{f}}
\right)
\left(
e^{i\hat{f}} +e^{-i\hat{f}}
\right) \\
&=
\inv{2} \left(
e^{i\hat{f}} e^{-i\hat{f}} +e^{-i\hat{f}} e^{i\hat{f}}
\right)
\end{aligned}
\end{equation}

Can we assume that these cancel for general operators?  How about for our specific differential operator \(\hat{f} = \partial^3/\partial x^3\)?  For that one we have

\begin{equation}\label{eqn:liboff319:82}
\begin{aligned}
e^{i \partial^3/\partial x^3} e^{-i \partial^3/\partial x^3} g(x)
&=
\sum_{k=0}^\infty
\inv{k!}
\left(\frac{\partial^3}{\partial x^3}\right)^k
\sum_{m=0}^\infty
\inv{m!}
\left(\frac{\partial^3}{\partial x^3}\right)^m g(x)
\end{aligned}
\end{equation}

Since the differentials commute, so do the exponentials and we can write the slightly simpler

\begin{equation}\label{eqn:liboff319:102}
\begin{aligned}
\sin^2 \hat{f} + \cos^2 \hat{f} = e^{i\hat{f}} e^{-i\hat{f}}
\end{aligned}
\end{equation}

I am pretty sure the commutative property of this differential operator would also allow us to say (in this case at least)

\begin{equation}\label{eqn:liboff319:122}
\begin{aligned}
\sin^2 \hat{f} + \cos^2 \hat{f} = 1
\end{aligned}
\end{equation}

Will have to look up the combinatoric argument that allows one to write, for numbers,

\begin{equation}\label{eqn:liboff319:142}
\begin{aligned}
e^x e^y =
\sum_{k=0}^\infty
\inv{k!} x^k
\sum_{m=0}^\infty
\inv{m!} y^m
=
\sum_{j=0}^\infty
\inv{j!} (x+y)^j
= e^{x+y}
\end{aligned}
\end{equation}

If this only assumes that \(x\) and \(y\) commute, and not any other numeric properties then we have the supposed result \eqnref{eqn:liboff319:2}.  We also know of algebraic objects where this does not hold.  One example is exponentials of non-commuting square matrices, and other is non-commuting bivector exponentials.

\EndArticle
