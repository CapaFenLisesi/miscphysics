%
% Copyright � 2012 Peeter Joot.  All Rights Reserved.
% Licenced as described in the file LICENSE under the root directory of this GIT repository.
%

%
%
%\documentclass{article}

%\input{../peeters_macros.tex}
%\input{../peeters_macros2.tex}

%\usepackage{listings}
%\usepackage{txfonts} % for ointctr... (also appears to make "prettier" \int and \sum's)
%\usepackage[bookmarks=true]{hyperref}

%\usepackage{color,cite,graphicx}
   % use colour in the document, put your citations as [1-4]
   % rather than [1,2,3,4] (it looks nicer, and the extended LaTeX2e
   % graphics package.
%\usepackage{latexsym,amssymb,epsf} % do not remember if these are
   % needed, but their inclusion can not do any damage


\chapter{Commutator and Anti-Commutator Hermitian-ness}
\label{chap:commutatorHerm}
%\author{Peeter Joot \quad peeterjoot@protonmail.com }
\date{ April 13, 2009.  commutatorHerm.tex }

%\begin{document}

%\maketitle{}

%\tableofcontents

\section{Motivation}

Reading of the proof of chapter 3, equation 12.7, in \citep{pauli2000wm},
that the anti-commutator

\begin{equation}\label{eqn:commutatorHerm:20}
\begin{aligned}
\symmetric{F}{G} = {F G + G F}
\end{aligned}
\end{equation}

is Hermitian, is unclear to me.  Fill in the missing details.  Also prove 12.8, that

\begin{equation}\label{eqn:commutatorHerm:40}
\begin{aligned}
i \antisymmetric{F}{G} = i (F G - G F)
\end{aligned}
\end{equation}

is Hermitian.

\section{Hermitian operator examples}

\subsection{Hermitian definition}

Pauli defines Hermitian in terms of the operator expectation value.  An operator \(H\) is Hermitian if

\begin{equation}\label{eqn:commutator_herm:Hermitian}
\begin{aligned}
\Expectation{H} = \int \psi^\conj (H \psi) d^3 x = \int \psi (H \psi)^\conj d^3 x
\end{aligned}
\end{equation}

Or
\begin{equation}\label{eqn:commutator_herm:HermitianInt}
\begin{aligned}
0 &= \Expectation{H} - \Expectation{H}^\conj = \int d^3 x \left( \psi^\conj (H \psi) - \psi (H \psi)^\conj \right)
\end{aligned}
\end{equation}

\subsection{Two operator form}

For completeness, let us derive the two wave function form of the Hermitian operator definition in full detail (omitted from the text).  With \(\psi = \psi_1 + \psi_2\)
\eqnref{eqn:commutator_herm:HermitianInt} becomes

\begin{equation}\label{eqn:commutatorHerm:60}
\begin{aligned}
\Expectation{H} - \Expectation{H}^\conj
&= \int (\psi_1 + \psi_2)^\conj (H (\psi_1 + \psi_2)) d^3 x - \int (\psi_1 + \psi_2) (H (\psi_1 + \psi_2))^\conj d^3 x \\
&= \int d^3 x \\
&\psi_1^\conj (H \psi_1) - \psi_1 (H \psi_1)^\conj
+\psi_1^\conj (H \psi_2) - \psi_1 (H \psi_2)^\conj \\
&+\psi_2^\conj (H \psi_2) - \psi_2 (H \psi_2)^\conj
+\psi_2^\conj (H \psi_1) - \psi_2 (H \psi_1)^\conj \\
&= \int d^3 x
\left( \psi_1^\conj (H \psi_2) - \psi_1 (H \psi_2)^\conj
+\psi_2^\conj (H \psi_1) - \psi_2 (H \psi_1)^\conj \right) \\
\end{aligned}
\end{equation}

Grouping terms, we have

\begin{equation}\label{eqn:commutatorHerm:80}
\begin{aligned}
\int d^3 x \left( \psi_1^\conj (H \psi_2) - \psi_2 (H \psi_1)^\conj \right) = \int d^3 x \left( \psi_1 (H \psi_2)^\conj - \psi_2^\conj (H \psi_1) \right)
\end{aligned}
\end{equation}

This is quite a bit different than both sides being separately zero, and the key to that further statement (as pointed out in 9.17 in \citep{bohm1989qt})
is that this is also true if the two wave function are adjusted by constant phase factors
\(\psi_1 \rightarrow \psi_1 e^{ia}\),
\(\psi_2 \rightarrow \psi_2 e^{ib}\).  Doing so we have

\begin{equation}\label{eqn:commutatorHerm:100}
\begin{aligned}
e^{i(a-b)} \int d^3 x \left( \psi_1^\conj (H \psi_2) - \psi_2 (H \psi_1)^\conj \right) = e^{i(b-a)} \int d^3 x \left( \psi_1 (H \psi_2)^\conj - \psi_2^\conj (H \psi_1) \right)
\end{aligned}
\end{equation}

For this to hold for any \(a\), \(b\) both sides of the equation must separately equal zero, and we have

\begin{equation}\label{eqn:commutator_herm:HermitianTwo}
\begin{aligned}
\int d^3 x \psi_1^\conj (H \psi_2) = \int d^3 x \psi_2 (H \psi_1)^\conj
\end{aligned}
\end{equation}

\subsection{Lemma for repeated operators}

Next, examine the reversion behavior of repeated operators.  This appears to be used in the text (or is proved implicitly via some other operation not explained).

Given an pair of Hermitian operators, \(H_1\), and \(H_2\),

\begin{equation}\label{eqn:commutatorHerm:120}
\begin{aligned}
\int d^3 x \psi_1^\conj (H_1 H_2 \psi_2)
\end{aligned}
\end{equation}

what do we get by reversing the operator action?  With the introduction of a couple of helper wave function variables, \(\epsilon = H_2 \psi_2\), and \(\beta = H_1 \psi_1\),
this becomes straight forward to determine

\begin{equation}\label{eqn:commutatorHerm:140}
\begin{aligned}
\int d^3 x \psi_1^\conj (H_1 \epsilon)
&=
\int d^3 x \epsilon (H_1 \psi_1)^\conj \\
&=
\int d^3 x (H_2 \psi_2) (H_1 \psi_1)^\conj \\
&=
\int d^3 x \beta^\conj (H_2 \psi_2) \\
&=
\int d^3 x \psi_2 (H_2 \beta)^\conj \\
\end{aligned}
\end{equation}

So we have
\begin{equation}\label{eqn:commutator_herm:reverse}
\begin{aligned}
\int d^3 x \psi_1^\conj (H_1 H_2 \psi_2) &= \int d^3 x \psi_2 (H_2 H_1 \psi_1)^\conj
\end{aligned}
\end{equation}

\subsection{Anti-commutator}

The statement that the anti-commutator is Hermitian means that we have

\begin{equation}\label{eqn:commutatorHerm:160}
\begin{aligned}
\int d^3 x \psi_2^\conj \left( \symmetric{F}{G} \psi_1 \right) &= \int d^3 x \psi_1 \left(\symmetric{F}{G} \psi_2 \right)^\conj \\
\text{or} \\
\int d^3 x \psi_2^\conj \left( (F G + G F) \psi_1 \right) &= \int d^3 x \psi_1 \left((F G + G F) \psi_2 \right)^\conj \\
\end{aligned}
\end{equation}

Let us expand the right hand side and see if we can get back the LHS
Or
\begin{equation}\label{eqn:commutatorHerm:180}
\begin{aligned}
\int d^3 x \psi_1 \left(\symmetric{F}{G} \psi_2 \right)^\conj
&=
\int d^3 x \psi_1 \left((F G + G F) \psi_2 \right)^\conj \\
&=
\int d^3 x \psi_2^\conj \left((G F + F G) \psi_1 \right) \quad\quad \text{(applying \eqnref{eqn:commutator_herm:reverse} twice)} \\
&=
\int d^3 x \psi_2^\conj \left( \symmetric{F}{G} \psi_1 \right) \\
\end{aligned}
\end{equation}

Hmm.  That is the Hermitian identity of \eqnref{eqn:commutator_herm:HermitianTwo}, so we are done.  Not at all complicated after all (albeit less
general than the text where a result for a more general pair of operators was given).

\subsection{Commutator}

Now, how about the (imaginary scaled) commutator case?

\begin{equation}\label{eqn:commutatorHerm:200}
\begin{aligned}
\int d^3 x \psi_2^\conj \left( i \antisymmetric{F}{G} \psi_1 \right) &= \int d^3 x \psi_1 \left( i \antisymmetric{F}{G} \psi_2 \right)^\conj \\
\text{or} \\
\int d^3 x \psi_2^\conj \left( i(F G - G F) \psi_1 \right) &= \int d^3 x \psi_1 \left(i(F G - G F) \psi_2 \right)^\conj \\
\end{aligned}
\end{equation}

Again, let us try just expanding out the RHS

\begin{equation}\label{eqn:commutatorHerm:220}
\begin{aligned}
\int d^3 x \psi_1 \left( i \antisymmetric{F}{G} \psi_2 \right)^\conj
&= \int d^3 x \psi_1 \left(i(F G - G F) \psi_2 \right)^\conj \\
&= -i \int d^3 x \psi_1 \left((F G - G F) \psi_2 \right)^\conj \\
&= -i \int d^3 x \psi_2^\conj \left((G F - F G) \psi_1 \right) \\
&= \int d^3 x \psi_2^\conj \left(i (F G - G F) \psi_1 \right) \\
&= \int d^3 x \psi_2^\conj \left(i \antisymmetric{F}{G}\psi_1 \right) \\
\end{aligned}
\end{equation}

QED.

\section{Future: Relation to Clifford product}

For vector spaces, as noted in \citep{gabookII:PJpauliMatrix},
we can write the Clifford product of two \R{N} vectors in terms of commutators and anti-commutators

\begin{equation}\label{eqn:commutatorHerm:240}
\begin{aligned}
\Bf \Bg = \inv{2} \left( \symmetric{\Bf}{\Bg} + i \antisymmetric{\Bf}{\Bg} \right)
\end{aligned}
\end{equation}

where \(i\) is the pseudoscalar for the space.  So, while \(FG\) is not necessarily Hermitian, it is interesting that the composite operator
\(\symmetric{F}{G} + i \antisymmetric{F}{G}\), which is so close to the product operator of Euclidean vector spaces, is Hermitian.
Explore this geometric analogy later.

%\bibliographystyle{plainnat}
%\bibliography{myrefs}

%\end{document}
