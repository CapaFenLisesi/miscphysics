%
% Copyright � 2012 Peeter Joot.  All Rights Reserved.
% Licenced as described in the file LICENSE under the root directory of this GIT repository.
%

%
%
%\documentclass{article}

%\input{../peeters_macros.tex}
%\input{../peeters_macros2.tex}
%\usepackage[bookmarks=true]{hyperref}

%\usepackage{color,cite,graphicx}
   % use colour in the document, put your citations as [1-4]
   % rather than [1,2,3,4] (it looks nicer, and the extended LaTeX2e
   % graphics package.
%\usepackage{latexsym,amssymb,epsf} % do not remember if these are
   % needed, but their inclusion can not do any damage

\chapter{Ehrenfest's theorem}
\label{chap:ehrenfest}
%\author{Peeter Joot \quad peeterjoot@protonmail.com}
\date{ Jan 22, 2009.  ehrenfest.tex }

%\begin{document}

%\maketitle{}

%\tableofcontents
\section{Motivation}

\citep{mcmahon2005qmd} has a one dimensional treatment of Ehrenfest's theorem,
that the expectation values of the position and momentum operators behave
like Newton's law.

However, he makes use
of commutator and braket notation before either is defined.

That looks like a natural way to do the derivation easily, but let us try
this using instead what is defined up to this point in the text.

\section{Review.  What do we know so far?}

\subsection{Position and momentum operators}

We have been given the definitions of two specific operators, position and momentum,
whos action on a wave function is

\begin{equation}\label{eqn:ehrenfest:21}
\begin{aligned}
\hat{x} \psi &= x \psi \\
\hat{p} \psi &= \left(-i \Hbar \PD{x}{}\right) \psi
\end{aligned}
\end{equation}

In operator form, with the omission of the explicit wave function being operated on this is

\begin{equation}\label{eqn:ehrenfest:41}
\begin{aligned}
\hat{x} &\equiv x  \\
\hat{p} &\equiv -i \Hbar \PD{x}{}
\end{aligned}
\end{equation}

These are perfectly valid operator definitions, but the validity of using the
classical names for these really comes from this upcoming Ehrenfest result where
the average of the action of these operators on a wave function is examined.

\subsection{Expectation (average) value of an operator}

We also have a definition for the expectation value
of an operator \(\hat{A}\), given its specific action \(A\).
This is defined very much like a weighted inner product
and is essentially a field weighted average of the operators action

\begin{equation}\label{eqn:ehrenfest:61}
\begin{aligned}
<\hat{A}> \equiv \int \psi^\conj (A \psi)
\end{aligned}
\end{equation}

The braces show that the operator action \(A\) here applies to the rightmost field variable \(\psi\), and not to its conjugate.

For the position and momentum operators respectively, we have the
expectation values

\begin{equation}\label{eqn:ehrenfest:81}
\begin{aligned}
<\hat{x}> &\equiv \int \psi^\conj (x \psi) \\
<\hat{p}> &\equiv \int \psi^\conj \left(-i \Hbar \PD{x}{}\right) \psi
\end{aligned}
\end{equation}

\subsection{Hermitian operator}

The notation of a Hermitian operator as also been introduced in terms of
left acting operators.  That is, an operator \(\hat{A}\) is Hermitian if

\begin{equation}\label{eqn:ehrenfest:hermitian1}
\begin{aligned}
\int \psi^\conj (A \psi) = \int (\psi A)^\conj \psi
\end{aligned}
\end{equation}

This is a somewhat non-Demystified seeming definition to me since I had seen
Hermitian defined more directly in terms of ``normal'' right acting
expectation integrals.  That is, an operator \(\hat{A}\) is Hermitian if

\begin{equation}\label{eqn:ehrenfest:101}
\begin{aligned}
<\hat{A}>^\conj = <\hat{A}>
\end{aligned}
\end{equation}

The conjugate of an operator's expectation value is
\begin{equation}\label{eqn:ehrenfest:121}
\begin{aligned}
\left(\int \psi^\conj (A \psi)\right)^\conj
&= \int \psi (A^\conj \psi^\conj) \\
&= \int (A^\conj \psi^\conj) \psi \\
\end{aligned}
\end{equation}

So, this second Hermitian definition means that an operator is Hermitian if

\begin{equation}\label{eqn:ehrenfest:141}
\begin{aligned}
\int (A^\conj \psi^\conj) \psi &= \int \psi^\conj (A \psi)
\end{aligned}
\end{equation}

This highlights why the left acting operator notation is pretty reasonable
seeming.  Allowing the conjugation operation to switch an operators action
from right acting to left acting makes the equation prettier, and
recovers \eqnref{eqn:ehrenfest:hermitian1}

\begin{equation}\label{eqn:ehrenfest:161}
\begin{aligned}
(\psi A)^\conj \equiv (A^\conj \psi^\conj)
\end{aligned}
\end{equation}

Here braces have been used to express the limitation of the scope of the action of the operator.

Another way to express this is that one can say that a Hermitian operator when put
in its wave function sandwich has a
conjugate action acting to the left on the conjugate wave function and a non-conjugate
action to the right.  This allows for a final notation nicety, where one can omit the
braces entirely as in

\begin{equation}\label{eqn:ehrenfest:181}
\begin{aligned}
\int \psi^\conj (A \psi) \equiv \int \psi^\conj A \psi \equiv \int (A^\conj \psi^\conj) \psi
\end{aligned}
\end{equation}
or in terms of right and left operator notation the equivalent
\begin{equation}\label{eqn:ehrenfest:201}
\begin{aligned}
\int \psi^\conj (A \psi) \equiv \int \psi^\conj A \psi \equiv \int (\psi A)^\conj \psi
\end{aligned}
\end{equation}

And finally, there is one last way to express this the concept of Hermitian.
We have our definition of a left acting operator

\begin{equation}\label{eqn:ehrenfest:221}
\begin{aligned}
(\psi A)^\conj = A^\conj \psi^\conj
\end{aligned}
\end{equation}

And can make the observation that conjugation of a product is the
product of the conjugates
\begin{equation}\label{eqn:ehrenfest:241}
\begin{aligned}
(\psi A)^\conj = \psi^\conj A^\conj
\end{aligned}
\end{equation}

So we must also have \(A = A^\conj\) for a Hermitian operator.

From this one can observe that the position operator \(\hat{x}\) is Hermitian, but the momentum operator is not (but \(\hat{p}^2\) is ).

\subsection{Variance and Heisenberg principle}

Various calculations have been done to calculate expectation values.

In a few places we have had to show that the product of variances

\begin{equation}\label{eqn:ehrenfest:261}
\begin{aligned}
\Delta A = \sqrt{<A^2> - <A>^2}
\end{aligned}
\end{equation}

for position and momentum all satisfy the famous Heisenberg uncertainty
principle

\begin{equation}\label{eqn:ehrenfest:281}
\begin{aligned}
\Delta x \Delta p \ge \Hbar/2
\end{aligned}
\end{equation}

(in a couple places this formulation is a bit fuzzy since our squared
momentum variance \((\Delta p)^2\) has been negative).

\subsection{The wave equation}

We are also given Schr\"{o}dinger equation in Hamiltonian form

\begin{equation}\label{eqn:ehrenfest:301}
\begin{aligned}
\hat{H} \psi = i \Hbar \PD{t}{\psi}
\end{aligned}
\end{equation}

and have worked with the specific form of the Hamiltonian that applies to
a non-relativistic particle (and not to photons).

\begin{equation}\label{eqn:ehrenfest:321}
\begin{aligned}
\hat{H} = \frac{\hat{p}^2}{2m} + V = -\frac{\Hbar^2}{2m} \grad^2 + V
\end{aligned}
\end{equation}

Most of the text up to this point has been about calculating and interpreting
specific solutions of this equation.

\subsection{Other stuff}

A number of other fundamental topics have been covered, probabilities, normalization, probability current, energy, phase, orthogonality, and so forth.  However, summarizing the rest of these in detail is not required as
background for the Ehrenfest result.

\section{Ehrenfest theorem}

We want to calculate the time derivatives of the expectation values
for position and momentum OPERATORS, and show that these reproduce the
familiar velocity, momentum and force concepts from classical mechanics.

\subsection{Velocity from the derivative of the position operator expectation}

Diving straight in we have

\begin{equation}\label{eqn:ehrenfest:341}
\begin{aligned}
\PD{t}{<\hat{x}>}
&= \PD{t}{} \left( \int \psi^\conj x \psi \right) \\
&= \int \PD{t}{\psi^\conj} x \psi + \int \psi^\conj x \PD{t}{\psi}
\end{aligned}
\end{equation}

Now, here the Hamiltonian can be introduced, replacing the time derivatives.

We have
\begin{equation}\label{eqn:ehrenfest:361}
\begin{aligned}
\PD{t}{\psi} &= -\frac{i}{\Hbar} H \psi \\
\PD{t}{\psi^\conj} &= \frac{i}{\Hbar} H \psi^\conj \\
\end{aligned}
\end{equation}

So we have
\begin{equation}\label{eqn:ehrenfest:381}
\begin{aligned}
\PD{t}{<\hat{x}>}
&= \frac{i}{\Hbar} \int \psi x H {\psi^\conj} - \frac{i}{\Hbar}\int \psi^\conj x H {\psi}
\end{aligned}
\end{equation}

For the Schr\"{o}dinger Hamiltonian we have

\begin{equation}\label{eqn:ehrenfest:401}
\begin{aligned}
H \psi &= - \frac{\Hbar^2}{2m} \PDSq{x}{\psi} + V\psi \\
H \psi^\conj &= - \frac{\Hbar^2}{2m} \PDSq{x}{\psi^\conj} + V\psi^\conj \\
\end{aligned}
\end{equation}

Combining these we have
\begin{equation}\label{eqn:ehrenfest:421}
\begin{aligned}
\PD{t}{<\hat{x}>}
&=
\frac{i}{\Hbar} \int \psi x \left( - \frac{\Hbar^2}{2m} \PDSq{x}{\psi^\conj} + V\psi^\conj \right)
-\frac{i}{\Hbar} \int \psi^\conj x \left(- \frac{\Hbar^2}{2m} \PDSq{x}{\psi} + V\psi \right) \\
&=
\frac{i\Hbar}{2m} \int \left(\psi^\conj x \PDSq{x}{\psi} -\psi x \PDSq{x}{\psi^\conj} \right)
+\frac{i}{\Hbar} \int \left(\psi x V\psi^\conj -\psi^\conj x V\psi \right)
\\
\end{aligned}
\end{equation}

The second term is zero, and by integrating the first term by parts twice we have

\begin{equation}\label{eqn:ehrenfest:441}
\begin{aligned}
\PD{t}{<\hat{x}>}
&= \frac{i\Hbar}{2m} \int \psi^\conj \left(x \PDSq{x}{\psi} - \PDSq{x}{(\psi x)} \right) \\
&= \frac{i\Hbar}{2m} \int \psi^\conj \left(x \PDSq{x}{\psi} - \PD{x}{}\left(x \PD{x}{\psi} + \psi\right) \right) \\
&= \frac{-i\Hbar}{2m} (2) \int \psi^\conj \PD{x}{\psi} \\
&= \frac{1}{m} \int \psi^\conj \left(-i\Hbar \PD{x}{} \right) {\psi} \\
\end{aligned}
\end{equation}

So we now have the QM equivalent of \(p = mv\), directly from the Sch\"{o}dinger equation and the definition of expectation values
of operators.

\begin{equation}\label{eqn:ehrenfest:461}
\begin{aligned}
\PD{t}{<\hat{x}>} &= \frac{<p>}{m}
\end{aligned}
\end{equation}

This is the first inkling that it makes sense to assign the names position and momentum to the corresponding operators
of QM!  Now the QMD derivation is way shorter and tidier, but this needed only integration by parts.  We really do not
need the more advanced operator concepts to get this important result.

\subsection{Force from the derivative of the momentum operator expectation}

Now lets calculate the momentum expectation change with time.

\begin{equation}\label{eqn:ehrenfest:481}
\begin{aligned}
\PD{t}{<p>}
&= \PD{t}{} \int \psi^\conj \left(-i \Hbar \PD{x}{}\right) \psi \\
&= -i \Hbar \int \PD{t}{\psi^\conj} \PD{x}{\psi} +{\psi^\conj} \PD{t}{}\PD{x}{\psi} \\
&= -i \Hbar \int \PD{t}{\psi^\conj} \PD{x}{\psi} +{\psi^\conj} \PD{x}{}\PD{t}{\psi} \\
&= \int \PD{x}{\psi} H \psi^\conj - {\psi^\conj} \PD{x}{} H\psi \\
&= \int \PD{x}{\psi} H \psi^\conj + \PD{x}{\psi^\conj} H\psi \\
&=
\int \PD{x}{\psi} \left(- \frac{\Hbar^2}{2m} \PDSq{x}{\psi^\conj} + V\psi^\conj \right)
+ \PD{x}{\psi^\conj} \left(- \frac{\Hbar^2}{2m} \PDSq{x}{\psi} + V\psi \right)
\\
&=
-
\frac{\Hbar^2}{2m}
\int \PD{x}{\psi} \PDSq{x}{\psi^\conj} + \PD{x}{\psi^\conj} \PDSq{x}{\psi}
+\int \PD{x}{\psi} V\psi^\conj + \PD{x}{\psi^\conj} V\psi
\\
&=
- \frac{\Hbar^2}{2m} \int \PD{x}{}\left(\PD{x}{\psi} \PD{x}{\psi^\conj}\right)
+\int \PD{x}{} \left( \psi V\psi^\conj \right) -\int \psi \PD{x}{V} \psi^\conj
\\
\end{aligned}
\end{equation}

Now, again with the assumption that \(\psi\) and its derivatives are sufficiently small to vanish at the boundaries of the integration (this was also done in the integration by parts above), the first two terms are zero, and the last is an expectation value.  Specifically, we then have

\begin{equation}\label{eqn:ehrenfest:501}
\begin{aligned}
\PD{t}{<p>} &= - \left<\PD{x}{V}\right>
\end{aligned}
\end{equation}

... which appears to be the QM equivalent to the one dimensional version of \(F = -\grad V\), instead all defined in terms of expectation values.

Very cool!  Now, before learning the Lagrangian formalism, I would have been satisfied with this.  We can replace Newton's law with
Schr\"{o}dinger's equation, and logically everything else will follow from that.  Can we apply a procedure like this to
the Lagrangian for the wave equation, and find an expectation equivalent to the classical \(\LL = m\Bv^2/2 - V\)?

An additional obvious question is how to express the expectation value in the three dimensional case instead of the one dimensional case?

%\bibliographystyle{plainnat}
%\bibliography{myrefs}

%\end{document}
