%
% Copyright � 2012 Peeter Joot.  All Rights Reserved.
% Licenced as described in the file LICENSE under the root directory of this GIT repository.
%

%
%
%\documentclass{article}

%\input{../peeters_macros.tex}

%\usepackage[bookmarks=true]{hyperref}

\chapter{Taylor's theorem deviation}
\label{chap:taylors}
%\author{Peeter Joot \quad peeterjoot@protonmail.com}
\date{ Feb. 2, 2008. taylors.tex }

%\begin{document}

%\maketitle{}

\citep{hestenes1999nfc} presents a very simple derivation of Taylor's Theorem,
but I feel a
slightly different (dumber, but longer) presentation would be more effective.

In the same fashion, form the integral

\begin{equation}\label{eqn:taylors:20}
I = \int_{t}^{t+s} F'(u) du
\end{equation}

Now, observe that the this first order derivative can be written in
terms of its second order derivative

\begin{equation}\label{eqn:taylors:40}
(u F'(u))' = F'(u) + u F''(u)
\end{equation}

So we could write

\begin{equation}\label{eqn:taylors:160}
\begin{aligned}
I &= \int_{t}^{t+s} ((u F'(u))' - u F''(u)) du \\
  &= {u F'(u)} \vert_{u=t}^{t+s} - \int_{t}^{t+s} u F''(u)) du \\
  &= (t+s) F'(t+s) - t F'(t) - \int_{t}^{t+s} u F''(u)) du \\
\end{aligned}
\end{equation}

This is true, but not the Taylor expansion we are used to.  Adjusting things slightly leaves a zero term at \(u=t+s\), as follows:

\begin{equation}\label{eqn:taylors:60}
\left((t + s - u) F'(u)\right)' = -F'(u) + (t+s-u) F''(u)
\end{equation}

\begin{equation}\label{eqn:taylors:180}
\begin{aligned}
I &= F(t+s) - F(t) \\
 &= \int_{t}^{t+s} ( - ((t + s - u) F'(u))' + (t+s-u) F''(u) ) du \\
 &= - {(t + s - u) F'(u)} \vert_{u=t}^{t+s} + \int_{t}^{t+s} (t+s-u) F''(u) du \\
 &= s F'(t) + \int_{t}^{t+s} (t+s-u) F''(u) du \\
\end{aligned}
\end{equation}

This results in the first two order terms of the Tailor series, with an explicit remainder term

\begin{equation}\label{eqn:taylors:80}
F(t+s) = F(t) + s F'(t) + \int_{t}^{t+s} (t+s-u) F''(u) du
\end{equation}

This process can be continued for as many terms as desired.  Doing the calculation for the second order term yields

\begin{equation}\label{eqn:taylors:100}
\left(\frac{(t + s - u)^2}{2} F''(u)\right)' = -( t + s - u ) F''(u) + \frac{(t+s-u)}{2} F'''(u)
\end{equation}

And substituting that into the first order expansion above we have

\begin{equation}\label{eqn:taylors:120}
F(t+s) = F(t) + s F'(t) + \frac{s^2}{2} F''(t) + \int_{t}^{t+s} \frac{1}{2}(t+s-u)^2 F'''(u) du
\end{equation}

Induction produces n terms of Taylor's series with an explicit remainder

\begin{equation}\label{eqn:taylors:140}
F(t+s) = \sum_{k=0}^{n} \frac{s^k}{k!} \frac{d^k}{dt^k} F(t) +
                        \int_{t}^{t+s} \frac{1}{n!}(t+s-u)^n \frac{d^{n+1}}{dt^{n+1}} F(u) du
\end{equation}

To truly prove the infinite series result one would have to show that the remainder term tends to zero.

%\bibliographystyle{plainnat} % supposed to allow for \url use.
%\bibliography{myrefs}      % expects file "myrefs.bib"

%\end{document}               % End of document.
