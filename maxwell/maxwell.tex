%
% Copyright � 2012 Peeter Joot.  All Rights Reserved.
% Licenced as described in the file LICENSE under the root directory of this GIT repository.
%

%
%
%\documentclass{article}      % Specifies the document class

%\usepackage{amsmath}
                             % The preamble begins here.
\chapter{Various formulations of Maxwell's equations} % Declares the document's title.
\label{chap:maxwell}
%\author{Peeter Joot \quad peeterjoot@protonmail.com}         % Declares the author's name.
\date{March 25, 2000.  maxwell.tex }

%\begin{document}             % End of preamble and beginning of text.

%\maketitle{Notes from review and re-study of University electrodynamics}

\section{Differential and integral forms of Maxwell's equations}

The standard formulation of Maxwell's equations these days is the differential
form, which is not the most intuitive.  In CGS units, the differential
form is as follows:

\begin{equation}\label{eqn:maxwell:20}
\begin{aligned}
\diverg \BE &= 4\pi\rho \\
\diverg \BB &= 0 \\
\curl \BB &= 4\pi \Bj - \frac{1}{c} \PD{t}{\BE} \\
\curl \BE &= \frac{1}{c} \PD{t}{\BB}
\end{aligned}
\end{equation}

It is interesting to note that these are probably not the form that Maxwell
originally formulated his equations in.  In my 1963 version of Encyclopedia
Britannica, where most of the articles look like they are based more closely on the Author's
original works, Maxwell's equations were given in an integral form where the
integrals were all related the tangential and normal components of the various vectors.  I had
trouble seeing how these two forms were related at first because I hhad not expected to
see the equations in this form.  Closer examination showed that the two forms
were equivalent, and the transformation between the two can be made by applying
applying standard integral transformations.

Specifically this can be done by using Gauss's theorem (where \(V_S\) is the volume enclosed by surface S)

\begin{equation*}
\int_{V_S} \diverg \BF\,dV =
\int_S \mathbf{F} \cdot\, d\BS
\end{equation*}

and Stokes theorem (where \(S_C\) is an arbitrary surface bounding curve C)
\begin{equation*}
\int_{S_C} \curl \BF \cdot d\BS = \oint_C \mathbf{F} \cdot d\mathbf{l}
\end{equation*}

and by integrating Maxwell's equations over either volumes or surfaces.
\begin{equation}\label{eqn:maxwell:40}
\begin{aligned}
\int_{V_S} \diverg \BE \, dV &= 4\pi \int_{V_S} \rho\, dV \\
\int_{V_S} \diverg \BB \, dV &= 0 \\
\int_{S_C} \curl \BB \cdot \, d\BS &= 4\pi \int_{S_C} \Bj \cdot \, d\BS - \frac{1}{c} \int_{S_C} \PD{t}{\BE} \cdot \, d\BS \\
\int_{S_C} \curl \BE \cdot \, d\BS &= \frac{1}{c} \int_{S_C} \PD{t}{\BB} \cdot \, d\BS
\end{aligned}
\end{equation}

The applications of the Gauss and Stokes theorems gives
\begin{equation}\label{eqn:maxwell:60}
\begin{aligned}
\int_{S} \BE \cdot \, d\BS &= 4\pi \int_{V_S} \rho\, dV \\
\int_{S} \BB \cdot \, d\BS &= 0 \\
\oint_{C} \BB \cdot \, d\mathbf{l} &= 4\pi \int_{S_C} \Bj \cdot \, d\BS - \frac{1}{c} \int_{S_C} \PD{t}{\BE} \cdot \, d\BS \\
\oint_{C} \BE \cdot \, d\mathbf{l} &= \frac{1}{c} \int_{S_C} \PD{t}{\BB} \cdot \, d\BS
\end{aligned}
\end{equation}

In terms of normal and tangential components, and after taking the time
derivatives out of the integrals, we have

\begin{equation}\label{eqn:maxwell:80}
\begin{aligned}
\int_{S} E_n \, dS &= 4\pi \int_{V_S} \rho\, dV \\
\int_{S} B_n \, dS &= 0 \\
\oint_{C} B_t \, dl &= 4\pi \int_{S_C} j_n \, dS - \frac{1}{c} \PD{t}{} \int_{S_C} E_n \, dS \\
\oint_{C} E_t \, dl &= \frac{1}{c} \PD{t}{} \int_{S_C} B_n \, dS
\end{aligned}
\end{equation}

This is how
Maxwell's equations are presented in my old
Encyclopedia Britannica
except for the fact that
the above is not in terms of SI units.  I expect that this formulation is closer to how
Maxwell actually presented his equations because they are in terms of more real
seeming quantities.  There are no curl and divergence terms to cloud the mind or
abstract more than required to state the facts.

However, as with the differential form, it is not clear how one would find
\(\BE\) and \(\BB\) given an arbitrary charge and current densities.  More on
this below.

\section{Integral and differential forms of Gauss's law}

Other transformations to and from the differential and integral forms of these
equations are also possible.  For example, the formula \(\diverg \BE = 4\pi\rho\)
can be shown by taking the divergence of the electric field due to a static
charge distribution
\begin{equation*}
\diverg \BE = \diverg \int_V{\frac{\rho (\mathbf{r - r'})\,dV'}{\norm{\mathbf r - \mathbf r'}^3}}
\end{equation*}

This result is derived in my yellow electrodynamics book, but is quite easy to
do.  The divergence can be moved inside the integral where it operates only on
the \(\frac {\mathbf{r - r'}} {\norm{\mathbf r - r'}^3}\) terms, since \(\rho\) is
only a function of \(\mathbf r'\).
Except in a neighborhood
\(\norm{\mathbf{r - r'}} < \epsilon\)
where the derivative cannot be taken,
this divergence works out to be zero which is pretty easy to calculate.
For this region, one can use Gauss's theorem to convert the
integral into a surface integral, which evaluates to \(4\pi\) easily by integrating in spherical polar coordinates
around a small sphere around \(\mathbf r\).

Note that the general expression for the
electric field in time varying conditions is considerably more complicated.
The electrostatics formula
\(\BE = \int_V{\frac{\rho (\mathbf{r - r'})\,dV'}{\norm{\mathbf r - \mathbf r'}^3}}\)
for the electric field is not valid in time
varying fields, but what
about \(\diverg \BE = 4\pi\rho\) (Gauss's law)?  In the statement of Maxwell's equations
there are no qualifications about validity in time varying fields.  This may
make sense since the electrostatics formula for \(\BE\) implies Gauss's law,
but Gauss's law by itself does not imply the former, at least so far as I can
see, leaving potential
degrees of freedom to account for time variance.

\section{Interdependence of the electric and magnetic field equations}

One of the awkward things with the standard formulation of
Maxwell's equations, whether it be the
differential or the integral form, is that there
is an awkward interdependence between the electric and magnetic fields.
The solution of either \(\BE\) or \(\BB\) interdependently seems difficult.  The
fact that there are four equations and two unknowns (assuming that the
change and current distributions are known) is also slightly awkward.
There is also a great deal of asymmetry in this form, which is unpleasant
to look at, and seems to suggest something missing.

In my electrodynamics book, where Maxwell's equations are ``derived''
\footnote
{
Like many derivations in math and physics, knowing the answer before hand
can lead to elegant and sophisticated, but artificial, methods of showing the desired result --
methods that
nobody would naturally attempt to use if actually deriving the result from first principles.
},
these equations can be seen to have a more symmetrical higher level form, where
these equations are written in the form of tensor relations where
the elements of the tensors are the electric and magnetic field components
or scalar multiples of these components.  One of the things that I found
pleasant about this formulation was that the cross product, a rather
arbitrary sort of beast ends up occurring in a natural seeming fashion.  I
think that this confirms the fact that in order to see the higher level
structure of the cross product it must be expressed in a matrix or tensor form.
\footnote{
Another example of this can be found by looking at the three dimensional
derivation of the torque formula, where the
cross product can be seen to be the transformation (per unit angle)
applied to an object moving through incremental three dimensional
rotation.  A half cross product operator can be constructed in a matrix form
that is much less arbitrary seeming than the component form definition.
}

Anyhow,
the interdependence of the \(\BE\) and \(\BB\) field equations can be removed by taking the
curl of the last two of Maxwell's equations above, and by using the following
vector relation

\begin{equation*}
\curl (\curl \BV) = \spacegrad (\diverg \BV) - \delsquared \BV
\end{equation*}

For the magnetic field we have

\begin{equation}\label{eqn:maxwell:100}
\begin{aligned}
\curl (\curl \BB) 			&= \spacegrad (\diverg \BB) - \delsquared \BB \\
\curl \Bigl(4\pi\Bj - \frac{1}{c} \PD{t}{\BE} \Bigr) 	&= \\
4\pi\curl \Bj - \frac{1}{c} \PD{t}{} \curl \BE 	&= \\
4\pi\curl \Bj - \frac{1}{c^2} \PDSq{t}{\BB} &= \\
      	     				&= - \delsquared \BB
\end{aligned}
\end{equation}

We can do the same calculation for the electric field.

\begin{equation}\label{eqn:maxwell:120}
\begin{aligned}
\curl (\curl \BE) &= \spacegrad (\diverg \BE) - \delsquared \BE \\
\curl \Bigl(\frac{1}{c} \PD{t}{\BB}\Bigr) &= \\
\frac{1}{c} \PD{t}{} \curl \BB &= \\
\frac{1}{c} \PD{t}{} \Bigl(4\pi \Bj - \frac{1}{c} \PD{t}{\BE} \Bigr) &= \\
\frac{4\pi}{c} \PD{t}{\Bj} - \frac{1}{c^2} \PDSq{t}{\BE} &= \\
             &= \spacegrad (4\pi\rho) - \delsquared \BE \\
\end{aligned}
\end{equation}

Using these results we can express the
electric field \(\BE\) and the
magnetic field \(\BB\)
as independent differential equations.

\begin{equation}\label{eqn:maxwell:140}
\begin{aligned}
\delsquared \BB - \frac{1}{c^2} \PDSq{t}{\BB} &= - 4\pi \left(\curl \Bj\right) \\
\delsquared \BE - \frac{1}{c^2} \PDSq{t}{\BE} &= - 4\pi \Bigl(\frac{1}{c} \PD{t}{\Bj} + \spacegrad \rho\Bigr)
\end{aligned}
\end{equation}

In the absence of current and charge densities these equations take the simple form of standard wave
equations.
\footnote{
The left hand side
$
\delsquared - \frac{1}{c^2}\PDSq{t}{}
$
is a common operation in physics, and I believe that
it is also referred to as the DeLambertian operator and has a squared box symbol
% how do I write it in tex?
}

\begin{equation}\label{eqn:maxwell:160}
\begin{aligned}
\Bigl(\delsquared - \frac{1}{c^2} \PDSq{t}{} \Bigr) \, \BE &= 0 \\
\Bigl(\delsquared - \frac{1}{c^2} \PDSq{t}{} \Bigr) \, \BB &= 0
\end{aligned}
\end{equation}

In this form, where the charge and current densities are zero, we finally have
a symmetrical description of the electric and the magnetic fields and have equations
for \(\BE\) and \(\BB\) in a mathematically pliable form that has well known
solution techniques.
Note that these are the homogeneous subset of the general equations above.
We can add any combination of scalar multiples to a specific solution
to the general equation to create a solution that meets the boundary
value conditions.

\section{Symmetrical form of time dependent Maxwell's equations}

It can
be noted that the general decoupled differential equations above are more
symmetrical then 4 equation form of Maxwell's equations, although the
right hand sides have a significantly different form at a glance.  There
are similarities.  Both have a \(-4\pi\) term and derivatives of the charge and
current densities.  The electric field equation has space derivatives (the curl term) of
the current density, and the the magnetic field equation has space derivatives of
the charge density (the gradient term) as well as a time derivative of the current
density.

We can write the left hand sides of the equations in a simpler form by using the
space time four vector \((x_\alpha)_\alpha = (x, y, z, ict)\).

\begin{equation}\label{eqn:maxwell:180}
\begin{aligned}
\delambert \,\BE &= - 4\pi \left(\curl \Bj\right) \\
\delambert \,\BB &= - 4\pi \Bigl(\frac{1}{c} \PD{t}{\Bj} + \spacegrad \rho\Bigr)
\end{aligned}
\end{equation}

These equations
can be broken down
into components, yielding six equations

\begin{equation}\label{eqn:maxwell:200}
\begin{aligned}
\delambert \,E_x &= - 4\pi \Bigl(\PD{y}{j_z} - \PD{z}{j_y} \Bigr) \\
\delambert \,E_y &= - 4\pi \Bigl(\PD{z}{j_x} - \PD{x}{j_z} \Bigr) \\
\delambert \,E_z &= - 4\pi \Bigl(\PD{x}{j_y} - \PD{y}{j_x} \Bigr) \\
\delambert \,B_x &= - 4\pi \Bigl(\PD{ct}{j_x} - \PD{x}{\rho} \Bigr)\\
\delambert \,B_y &= - 4\pi \Bigl(\PD{ct}{j_y} - \PD{y}{\rho} \Bigr)\\
\delambert \,B_z &= - 4\pi \Bigl(\PD{ct}{j_z} - \PD{z}{\rho} \Bigr)
\end{aligned}
\end{equation}

Which can be put into the following form by replacing \(x, y, z\) and \(ict\) by their
corresponding space time four vector components \(x_1, x_2, x_3\) and \(x_4\)

\begin{equation}\label{eqn:maxwell:220}
\begin{aligned}
            \delambert \,E_1 &= - 4\pi \Bigl( \PD{x_2}{j_z} -             \PD{x_3}{j_y} \Bigr) \\
            \delambert \,E_2 &= - 4\pi \Bigl( \PD{x_3}{j_x} -             \PD{x_1}{j_z} \Bigr) \\
            \delambert \,E_3 &= - 4\pi \Bigl( \PD{x_1}{j_y} -             \PD{x_2}{j_x} \Bigr) \\
\frac{1}{i} \delambert \,B_1 &= - 4\pi \Bigl( \PD{x_4}{j_x} - \frac{1}{i} \PD{x_1}{\rho} \Bigr)\\
\frac{1}{i} \delambert \,B_2 &= - 4\pi \Bigl( \PD{x_4}{j_y} - \frac{1}{i} \PD{x_2}{\rho} \Bigr)\\
\frac{1}{i} \delambert \,B_3 &= - 4\pi \Bigl( \PD{x_4}{j_z} - \frac{1}{i} \PD{x_3}{\rho} \Bigr)
\end{aligned}
\end{equation}

In component form,
there are precisely two differential terms per equation, with a suggestively
similar form.
Just as the space and time aspects of the left hand side could be consolidated
by introducing a space time four vector, we can similarity introduce a four
vector for the
charge and current densities.
We can define a four vector $(j_\alpha)_\alpha =
(j_x, j_y, j_z, -i \rho)$ and modify the right hand sides of each of these equations
accordingly.

\begin{equation}\label{eqn:maxwell:240}
\begin{aligned}
\delambert E_1 &= -4\pi \Bigl(\PD{x_2}{j_3} - \PD{x_3}{j_2} \Bigr) \\
\delambert E_2 &= -4\pi \Bigl(\PD{x_3}{j_1} - \PD{x_1}{j_3} \Bigr) \\
\delambert E_3 &= -4\pi \Bigl(\PD{x_1}{j_2} - \PD{x_2}{j_1} \Bigr) \\
\delambert B_1 &= -4\pi i \Bigl(\PD{x_4}{j_1} - \PD{x_1}{j_4} \Bigr)\\
\delambert B_2 &= -4\pi i \Bigl(\PD{x_4}{j_2} - \PD{x_2}{j_4} \Bigr)\\
\delambert B_3 &= -4\pi i \Bigl(\PD{x_4}{j_3} - \PD{x_3}{j_4} \Bigr)
\end{aligned}
\end{equation}

This is as symmetrical I can put these equations for now.  There is probably some higher level
view that can be used to see the structure behind the particular set of indices in the right
hand size, but I will leave that for some other time.

%\end{document}               % End of document.
