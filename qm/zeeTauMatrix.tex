%
% Copyright � 2012 Peeter Joot.  All Rights Reserved.
% Licenced as described in the file LICENSE under the root directory of this GIT repository.
%

%
%
%\input{../peeter_prologue_print.tex}
%\input{../peeter_prologue_widescreen.tex}

\chapter{On tensor product generators of the gamma matrices}
\label{chap:zeeTauMatrix}
%\useCCL
\blogpage{http://sites.google.com/site/peeterjoot/math2011/zeeTauMatrix.pdf}
\date{June 20, 2011}
\revisionInfo{zeeTauMatrix.tex}

%\beginArtWithToc
\beginArtNoToc

\section{Motivation}

In \citep{zee2005quantum} he writes

\begin{equation}\label{eqn:zeeTauMatrix:160}
\begin{aligned}
\gamma^0 &=
\begin{bmatrix}
I & 0 \\
0 & -I
\end{bmatrix}
=
I \otimes \tau_3 \\
\gamma^i &=
\begin{bmatrix}
0 & \sigma^i \\
\sigma^i & 0
\end{bmatrix}
=
\sigma^i \otimes i \tau_2 \\
\gamma^5 &=
\begin{bmatrix}
0 & I \\
I & 0
\end{bmatrix}
=
I \otimes \tau_1
\end{aligned}
\end{equation}

The Pauli matrices \(\sigma^i\) I had seen, but not the \(\tau_i\) matrices, nor the \(\otimes\) notation.  Strangerep in \href{http://www.physicsforums.com/showthread.php?p=3365680\#post3365680}{physicsforums} points out that the \(\otimes\) is a Kronecker matrix product, a special kind of tensor product \citep{wiki:tensorProduct}.  Let us do the exercise of reverse engineering the \(\tau\) matrices as suggested.

\section{Guts}

Let us start with \(\gamma^5\).  We want

\begin{equation}\label{eqn:zeeTauMatrix:10}
\gamma^5 = I \otimes \tau_1 =
\begin{bmatrix}
I \tau_{11} & I \tau_{12} \\
I \tau_{21} & I \tau_{22} \\
\end{bmatrix}
=
\begin{bmatrix}
0 & 1 \\
1 & 0
\end{bmatrix}
\end{equation}

By inspection we must have
\begin{equation}\label{eqn:zeeTauMatrix:30}
\tau_1 =
\begin{bmatrix}
0 & 1 \\
1 & 0
\end{bmatrix}
= \sigma^1
\end{equation}

Thus \(\tau_1 = \sigma^1\).  How about \(\tau_2\)?  For that matrix we have

\begin{equation}\label{eqn:zeeTauMatrix:50}
\gamma^i = \sigma^i \otimes i \tau_2 =
\begin{bmatrix}
\sigma^i \tau_{11} & \sigma^i \tau_{12} \\
\sigma^i \tau_{21} & \sigma^i \tau_{22} \\
\end{bmatrix}
=
\begin{bmatrix}
0 & 1 \\
1 & 0
\end{bmatrix}
\end{equation}

Again by inspection we must have
\begin{equation}\label{eqn:zeeTauMatrix:70}
i \tau_2 =
\begin{bmatrix}
0 & 1 \\
-1 & 0
\end{bmatrix},
\end{equation}

so
\begin{equation}\label{eqn:zeeTauMatrix:90}
\tau_2 =
\begin{bmatrix}
0 & -i \\
i & 0
\end{bmatrix}
= \sigma^2.
\end{equation}

This one is also just the Pauli matrix.  For the last we have

\begin{equation}\label{eqn:zeeTauMatrix:110}
\gamma^0 = I \otimes \tau_3 =
\begin{bmatrix}
I \tau_{11} & I \tau_{12} \\
I \tau_{21} & I \tau_{22} \\
\end{bmatrix}
=
\begin{bmatrix}
1 & 0 \\
0 & -1
\end{bmatrix}.
\end{equation}

Our last tau matrix is thus
\begin{equation}\label{eqn:zeeTauMatrix:130}
\tau_3 =
\begin{bmatrix}
1 & 0 \\
0 & -1
\end{bmatrix}
= \sigma^3.
\end{equation}

Curious that there are twill notations used in the same page for exactly the same thing?  It appears \href{http://www.physicsforums.com/showthread.php?p=2844657}{that I was not the only person confused about this}.

\section{The bivector expansion}

Zee writes his wedge products with the commutator, adding a complex factor

\begin{equation}\label{eqn:zeeTauMatrix:140}
\sigma^{\mu\nu} = \frac{i}{2} \antisymmetric{\gamma^\mu}{\gamma^\nu}
\end{equation}

Let us try the direct product notation to expand \(\sigma^{0 i}\) and \(\sigma^{ij}\).  That first is

\begin{equation}\label{eqn:zeeTauMatrix:180}
\begin{aligned}
\sigma^{0 i}
&= \frac{i}{2} \left( \gamma^0 \gamma^i - \gamma^i \gamma^0 \right) \\
&= i \gamma^0 \gamma^i \\
&= i (I \otimes \tau_3)(\sigma^i \otimes i \tau_2) \\
&= i^2 \sigma^i \otimes \tau_3\tau_2 \\
&= - \sigma^i \otimes (-i \tau_1) \\
&= i \sigma^i \otimes \tau_1 \\
&= i
\begin{bmatrix}
0 & \sigma^i \\
\sigma^i & 0
\end{bmatrix},
\end{aligned}
\end{equation}

which is what was expected.  The second bivector, for \(i=j\) is zero, and for \(i\ne j\) is
\begin{equation}\label{eqn:zeeTauMatrix:200}
\begin{aligned}
\sigma^{i j}
&= i \gamma^i \gamma^j \\
&= i (\sigma^i \otimes i \tau_2) (\sigma^j \otimes i \tau_2) \\
&= i^3 (\sigma^i \sigma^j) \otimes I \\
&= i^4 (\epsilon_{ijk} \sigma^k) \otimes I \\
&= \epsilon_{ijk}
\begin{bmatrix}
\sigma^k & 0 \\
0 & \sigma^k
\end{bmatrix}.
\end{aligned}
\end{equation}

\EndArticle
