%
% Copyright � 2012 Peeter Joot.  All Rights Reserved.
% Licenced as described in the file LICENSE under the root directory of this GIT repository.
%

%
%
%\documentclass{article}

%\input{../peeters_macros.tex}
%\input{../peeters_macros2.tex}

%\usepackage[bookmarks=true]{hyperref}

%\usepackage{color,cite,graphicx}
   % use colour in the document, put your citations as [1-4]
   % rather than [1,2,3,4] (it looks nicer, and the extended LaTeX2e
   % graphics package.
%\usepackage{latexsym,amssymb,epsf} % do not remember if these are
   % needed, but their inclusion can not do any damage

\chapter{Chapter 2 Quiz solutions for QM Demystified book}
\label{chap:qmdCh2quiz}
%\author{Peeter Joot \quad peeterjoot@protonmail.com}
\date{ Jan 12, 2009.  qmdCh2quiz.tex }

%\begin{document}

%\maketitle{}
%\tableofcontents
\section{Motivation}

Work through the quiz problems from \citep{mcmahon2005qmd}.  I have treated these problems as only a guideline.  Instead the questions have been used as a
base to develop some comfort in the subject, exploring the math and physics past the scope of the quiz itself.

\section{Problem 1. Separation of variables}

This problem was to show that separation of variables leads to an exponential energy/phase
term.  Let us try this, but do it instead for three dimensions and explore a bit.

We try a test solution of the form

\begin{equation}\label{eqn:qmdCh2quiz:20}
\begin{aligned}
\psi &= X(x) Y(y) Z(z) T(t) \\
\end{aligned}
\end{equation}

and substitute into

\begin{equation}\label{eqn:qmdCh2quiz:40}
\begin{aligned}
\left(-\frac{\Hbar^2}{2m}\grad^2 + V\right) \psi &= i \Hbar \partial_t \psi
\end{aligned}
\end{equation}

differentiating and dividing by \(\psi\) we have
\begin{equation}\label{eqn:qmdCh2quiz:60}
\begin{aligned}
-\frac{\Hbar^2}{2m}
\left(
\frac{X''}{X}
+\frac{Y''}{Y}
+\frac{Z''}{Z}
\right)
 + V &= i \Hbar \frac{T'}{T}
\end{aligned}
\end{equation}

We set the right hand side equal to a constant \(E\) to be determined by boundary value conditions.
According to the dimensions of \(V\) this \(E\) constant can be seen to necessarily be an energy
of some sort.  In terms of this energy, we have for the function \(T\)

\begin{equation}\label{eqn:qmdCh2quiz:80}
\begin{aligned}
i \Hbar \frac{T'}{T} &= E
\end{aligned}
\end{equation}

With a solution of
\begin{equation}\label{eqn:qmdCh2quiz:100}
\begin{aligned}
T(t) &= e^{-i E t/\Hbar}
\end{aligned}
\end{equation}

Now, the left hand side imposes some constraints on \(E\), but these will be potential dependent.
The simplest case, for the wave function of a free particle, is where \(V=0\).

In that case we have
\begin{equation}\label{eqn:qmdCh2quiz:120}
\begin{aligned}
\frac{X''}{X} +\frac{Y''}{Y} +\frac{Z''}{Z} &= - \frac{2 m E}{\Hbar^2}
\end{aligned}
\end{equation}

The sum of each of the terms involved all identically equal a constant, which is perhaps
reasonable to assume to be negative.  If we do so and impose the usual sort of separation of
variables constraint, requiring each of the \(X''/X\), \(Y''/Y\), and \(Z''/Z\) terms to separately
equal some negative constant (to be fixed by boundary conditions), we can write

\begin{equation}\label{eqn:qmdCh2quiz:140}
\begin{aligned}
\frac{X''}{X} &= -{k_1}^2 \\
\frac{Y''}{Y} &= -{k_2}^2 \\
\frac{Z''}{Z} &= -{k_3}^2 \\
\end{aligned}
\end{equation}

So we have for the complete equation a solution proportional to

\begin{equation}\label{eqn:qmdCh2quiz:160}
\begin{aligned}
\psi = X Y Z T &= \exp(i(\Bk \cdot \Bx - E t/\Hbar))
\end{aligned}
\end{equation}

With the additional boundary value constraint of
\begin{equation}\label{eqn:qmdCh2quiz:180}
\begin{aligned}
\Bk^2 &= \frac{2 m E}{\Hbar^2}
\end{aligned}
\end{equation}

\subsection{Time independent equation}

Given that we can express the time variation of the wave function as an exponential, we can use this to calculate the time independent equation.  Let

\begin{equation}\label{eqn:qmdCh2quiz:200}
\begin{aligned}
\psi(\Bx,t) = \psi(\Bx) e^{-iEt/\Hbar}
\end{aligned}
\end{equation}

Substitution back into the equation gives

\begin{equation}\label{eqn:qmdCh2quiz:220}
\begin{aligned}
\left(-\frac{\Hbar^2}{2m}\grad^2 + V\right) \psi(\Bx)e^{-iEt/\Hbar} &= E\psi(\Bx)e^{-iEt/\Hbar}
\end{aligned}
\end{equation}

Or

\begin{equation}\label{eqn:qmdCh2quiz:240}
\begin{aligned}
\left(-\frac{\Hbar^2}{2m}\grad^2 + V\right) \psi(\Bx) &= E\psi(\Bx)
\end{aligned}
\end{equation}

%Or
%\begin{align*}
%\left(-\frac{\Hbar^2}{2m}\grad^2 \right) \psi(\Bx) &= (E - V)\psi(\Bx)
%\end{align*}

\subsection{Constant potential}

Let us consider a slightly more general potential with \(V\) constant in some spatial interval.

By inspection it appears that we should perhaps have

\begin{equation}\label{eqn:qmdCh2quiz:260}
\begin{aligned}
\psi &= \exp(i(\Bk \cdot \Bx - (E - V) t/\Hbar)) \\
\end{aligned}
\end{equation}

Let us check if this is right.  We want
\begin{equation}\label{eqn:qmdCh2quiz:280}
\begin{aligned}
\left(-\frac{\Hbar^2}{2m}\grad^2 + V -i\Hbar \partial_t\right)\psi = 0
\end{aligned}
\end{equation}

which means that we have
\begin{equation}\label{eqn:qmdCh2quiz:300}
\begin{aligned}
0 &=
\left(\frac{\Hbar^2}{2m}\Bk^2 + V -i\Hbar \frac{-i(E - V)}{\Hbar}\right)\psi \\
&= \left(\frac{\Hbar^2}{2m}\Bk^2 + V -(E - V)\right)\psi \\
&= \left(\frac{\Hbar^2}{2m}\Bk^2 - (E - 2V)\right)\psi \\
\end{aligned}
\end{equation}

so our wave number energy constraint is
\begin{equation}\label{eqn:qmdCh2quiz:320}
\begin{aligned}
\Bk^2 = \frac{2 m (E - 2V)}{\Hbar^2}
\end{aligned}
\end{equation}

This looks a bit strange, but can be fixed up writing \(E' = E - 2V\).  Then our test solution takes the form

\begin{equation}\label{eqn:qmdCh2quiz:340}
\begin{aligned}
\psi &= \exp(i(\Bk \cdot \Bx - (E' + V) t/\Hbar)) \\
\Bk^2 &= \frac{2 m (E' + V)}{\Hbar^2}
\end{aligned}
\end{equation}

If you guess wrong, the math forces you to fix the mistake!  A final check to see if this is kosher

\begin{equation}\label{eqn:qmdCh2quiz:360}
\begin{aligned}
0 &=
\left(\frac{\Hbar^2}{2m}\Bk^2 + V -i\Hbar \frac{-i(E + V)}{\Hbar}\right)\psi \\
&= \left(\frac{\Hbar^2}{2m}\Bk^2 + V -(E + V)\right)\psi \\
&= \left(\frac{\Hbar^2}{2m}\Bk^2 - E\right)\psi \\
\end{aligned}
\end{equation}

Okay, we are cool now, and things even make sense.  A reasonable interpretation is probably introduction of a constant potential raises the minimum ground state energy.

Based on just the math, we do not know that \(E\) is necessarily positive, so in addition to the trigonometric solution above is it also reasonable to
allow for possible
hyperbolic solutions?  If so then the following should also be allowed

\begin{equation}\label{eqn:qmdCh2quiz:380}
\begin{aligned}
\psi &= \exp(\Bk \cdot \Bx - i(E + V) t/\Hbar) \\
\Bk^2 &= \frac{2 m E}{\Hbar^2}
\end{aligned}
\end{equation}

We need some physics
to augment the math in order to determine what form of solution is actually valid.  Some of that physics likely comes in the form of the boundary conditions and perhaps other constraints such as normalization.

\subsection{General solution}

Using superposition we should be able to form a wave packet in integral form
by allowing for any set of \(\Bk\) vectors.  Suppose we assemble a test solution by summing over possible wave numbers

\begin{equation}\label{eqn:qmdCh2quiz:400}
\begin{aligned}
\psi = \int A(\Bk) \exp(\Bk \cdot \Bx - i(E + V) t/\Hbar) dk_1 dk_2 dk_3 \\
\end{aligned}
\end{equation}

For generality, allowing both trigonometric and hyperbolic solutions we can
allow the coordinates of \(\Bk\) to be real, imaginary, or zero.

Does this work?  Let us take derivatives and see what constraints we require if it does.

\begin{equation}\label{eqn:qmdCh2quiz:420}
\begin{aligned}
0
&= \left(-\frac{\Hbar^2}{2m}\grad^2 + V -i\Hbar \partial_t\right)\psi \\
&= \int A(\Bk) \left(-\frac{\Hbar^2}{2m}\sum_{j=1}^3 {k_j}^2 - E\right) \exp(\Bk \cdot \Bx - i(E + V) t/\Hbar) dk_1 dk_2 dk_3 \\
\end{aligned}
\end{equation}

Abusing notation somewhat for this complex "vector" \(\Bk\) by writing \(\Bk^2 = \sum {k_j}^2\), we have
as a general solution for this constant potential wave equation

\begin{equation}\label{eqn:qmdCh2quiz:440}
\begin{aligned}
\psi(\Bx,t) = \int A(\Bk) \exp\left(\Bk \cdot \Bx - i\left(-\frac{\Hbar^2 \Bk^2}{2m} + V\right) \frac{t}{\Hbar}\right) dk_1 dk_2 dk_3 \\
\end{aligned}
\end{equation}

So, for trig solutions (plane waves) we have \(\Bk\) purely imaginary, our energy integration constant \(E = -\Hbar^2\Bk^2/2m\) takes a positive value, whereas for hyperbolic solutions, it is negative.

I do not know if it is physically reasonable to allow for hyperbolic solutions.  If not, then we should just
factor out an explicit \(i\), and write
\begin{equation}\label{eqn:qmdCh2quiz:460}
\begin{aligned}
\psi(\Bx,t) = \int A(\Bk) \exp\left(i \left(\Bk \cdot \Bx - \left(\frac{\Hbar^2 \Bk^2}{2m} + V\right) \frac{t}{\Hbar}\right)\right) dk_1 dk_2 dk_3 \\
\end{aligned}
\end{equation}

%\subsection{Particle in a box}
%
%This screams for the boundary values of a 3D particle in a box problem
%(the one dimensional version of this problem is done in the text).
%Here we introduce an infinite potential outside the
%box, and this will require sinusoidal solutions so that they vanish on
%the boundaries since the particle can not get past the potential
%barrier (and must be continuous at the boundary).

\section{Problem 2. Probabilities for a polynomial wavefunction}

The wave function to work with is

\begin{equation}\label{eqn:qmdCh2quiz:480}
\begin{aligned}
\psi = C \frac{1 + ix}{1 + ix^2}
\end{aligned}
\end{equation}

With probability density

\begin{equation}\label{eqn:qmdCh2quiz:500}
\begin{aligned}
\psi\psi^\conj = C^2 \frac{1 + x^2}{1 + x^4}
\end{aligned}
\end{equation}

\subsection{normalize it}

We can do the normalization with a complex integral over an upper half plane
semicircular contour.  On the arc we the integral can be parametrized with

\begin{equation}\label{eqn:qmdCh2quiz:520}
\begin{aligned}
z &= R e^{i\theta} \\
dz &= R i e^{i\theta} d\theta
\end{aligned}
\end{equation}

\begin{equation}\label{eqn:qmdCh2quiz:540}
\begin{aligned}
\int \psi\psi^\conj dz = C^2 \int \frac{1 + R^2 e^{2i\theta}}{1 + R^4 e^{4i\theta}} R i e^{i\theta} d\theta
\end{aligned}
\end{equation}

This is of order \(R^3/R^4\) so will vanish at infinity.

For the remaining part of the integral we integrate on \([-\infty,\infty]\), but duck up and back around the poles
\(\sqrt{i}\), and \(i \sqrt{i}\) in counterclockwise circles.

Now, the only problem is to remember how to do the integral around the poles.  Suppose we have

\begin{equation}\label{eqn:qmdCh2quiz:560}
\begin{aligned}
I = \oint \frac{f(z)}{z -z_0} dz
\end{aligned}
\end{equation}

for a function that is regular at \(z_0\), and integrate in an infinitesimal loop around \(z_0\).  That contour is parametrized by

\begin{equation}\label{eqn:qmdCh2quiz:580}
\begin{aligned}
dz &= r i e^{i\theta} d\theta \\
z - z_0 &= r e^{i\theta}
\end{aligned}
\end{equation}

So this little contour integral has the value
\begin{equation}\label{eqn:qmdCh2quiz:600}
\begin{aligned}
I &= \oint f(z) i d\theta \\
\end{aligned}
\end{equation}

if the contour is made small enough that \(f(z)\) does not vary, then that function takes the value \(f(z_0)\), and we have for a clockwise contour the value

\begin{equation}\label{eqn:qmdCh2quiz:620}
\begin{aligned}
I = \oint \frac{f(z)}{z -z_0} dz = 2 \pi i f(z_0)
\end{aligned}
\end{equation}

Now for this probability density we can do a
partial fractions split around the poles \(\{\pm\sqrt{i}, \pm i \sqrt{i}\}\) of the form

\begin{equation}\label{eqn:qmdCh2quiz:640}
\begin{aligned}
\frac{1 + x^2}{1 + x^4} &=
\frac{A}{x -\sqrt{i}}
+\frac{B}{x +\sqrt{i}}
+\frac{C}{x -i\sqrt{i}}
+\frac{D}{x +i\sqrt{i}}
\end{aligned}
\end{equation}

but we do not really need do to this algebra.  Instead for pole \(p\) since it is first order we can write

\begin{equation}\label{eqn:qmdCh2quiz:660}
\begin{aligned}
\psi(x)\psi^\conj(x) = C^2 \left(\frac{1 + x^2}{1 + x^4} (x - p)\right) \inv{x - p}
\end{aligned}
\end{equation}

The left hand factor here is then regular at the pole, and we can use L'Hopitals rule to evaluate what value this takes at the pole.

For our first quadrant pole we have
\begin{equation}\label{eqn:qmdCh2quiz:680}
\begin{aligned}
{\left. \left(\frac{1 + x^2}{1 + x^4} (x - \sqrt{i})\right) \right\vert}_{x=\sqrt{i}}
&= \frac{ 1 + i }{4 i\sqrt{i}} \\
&= -i\sqrt{2}/4 \\
\end{aligned}
\end{equation}

and for the second quadrant pole we have
\begin{equation}\label{eqn:qmdCh2quiz:700}
\begin{aligned}
{\left. \left(\frac{1 + x^2}{1 + x^4} (x - \sqrt{i})\right) \right\vert}_{x=i\sqrt{i}}
&= \frac{ 1 + (i\sqrt{i})^2 }{4 (i\sqrt{i})^3} \\
&= \frac{ 1 - i }{4 \sqrt{i}} \\
&= \sqrt{2}\frac{ (1 - i)^2 }{8} \\
&= -i\sqrt{2}/4
\end{aligned}
\end{equation}

Combining all the contours we have for the integral now

\begin{equation}\label{eqn:qmdCh2quiz:720}
\begin{aligned}
0
&= 0 + I + 2(-2\pi i)(-i\sqrt{2}/4) C^2 \\
&= 0 + I - \pi \sqrt{2} C^2 \\
\end{aligned}
\end{equation}

Therefore for the unit probability we have as desired

\begin{equation}\label{eqn:qmdCh2quiz:740}
\begin{aligned}
C = \inv{\sqrt{ \pi \sqrt{2}}}
\end{aligned}
\end{equation}

\subsection{definite integral of probability}

Next part of the problem was to evaluate the probability of finding the particle in a specific region (
\([0,1]\) specifically).

Here we need a definite integral, so none of the contour integration tricks will help.  At least I remember how to do that now.

Let us actually do the partial fractions split

\begin{equation}\label{eqn:qmdCh2quiz:760}
\begin{aligned}
\frac{x^2+1}{x^4+1}
&= \frac{x^2+1}{x^4-i^2}  \\
&= \inv{2}\left( \frac{1}{x^2 - i} + \frac{1}{x^2 + i} \right) \\
&=
\inv{4\sqrt{i}}\left(
\frac{1}{x - \sqrt{i}}
-\frac{1}{x + \sqrt{i}} \right)
+\inv{4\sqrt{-i}}\left(
\frac{1}{x - \sqrt{-i}}
-\frac{1}{x + \sqrt{-i}}
\right)
\end{aligned}
\end{equation}

Anti-differentiation gives

\begin{equation}\label{eqn:qmdCh2quiz:780}
\begin{aligned}
%&
\inv{4\sqrt{i}} \ln\left({\frac{x - \sqrt{i}}{x + \sqrt{i}}}\right)
+\inv{4\sqrt{-i}}\ln\left( \frac{x - \sqrt{-i}}{x + \sqrt{-i}} \right) \\
%&=
%-\frac{i\sqrt{i}}{4} \ln\left({\frac{x - \sqrt{i}}{x + \sqrt{i}}}\right)
%-\frac{\sqrt{i}}{4}\ln\left( \frac{x - \sqrt{-i}}{x + \sqrt{-i}} \right) \\
%&=
%\frac{1-i}{4\sqrt{2}} \ln\left({\frac{x - \sqrt{i}}{x + \sqrt{i}}}\right)
%-\frac{1+i}{4\sqrt{2}}\ln\left( \frac{x - \sqrt{-i}}{x + \sqrt{-i}} \right) \\
%&=
%  \frac{1}{4\sqrt{2}} \ln\left({\frac{(x - \sqrt{i})}{(x + \sqrt{i})}}\frac{(x + \sqrt{-i})}{(x - \sqrt{-i})} \right)
%- \frac{i}{4\sqrt{2}} \ln\left( {\frac{(x - \sqrt{i})}{(x + \sqrt{i})}}\frac{(x - \sqrt{-i})}{(x + \sqrt{-i})} \right) \\
%&=
%  \frac{1}{4\sqrt{2}} \ln\left( \frac{ x^2 -1 + \sqrt{i}x (i-1) }{ x^2 -1 + \sqrt{i}x (1-i) } \right)
%- \frac{i}{4\sqrt{2}} \ln\left( \frac{ x^2 +1 + \sqrt{i}x (i-1) }{ x^2 -1 - \sqrt{i}x (1+i) } \right) \\
%&=
%  \frac{1}{4\sqrt{2}} \ln\left( \frac{ \sqrt{2}(x^2 +1) - 2 x }{ \sqrt{2}(x^2 -1) + 2 x }\right)
%- \frac{i}{4\sqrt{2}} \ln\left( \frac{ \sqrt{2}(x^2 +1) - 2 x }{ \sqrt{2}(x^2 -1) - 2 i x }\right) \\
\end{aligned}
\end{equation}
%1/\sqrt{i} = \sqrt{i}/i = -i\sqrt{i}
%1/\sqrt{-i} = \sqrt{-i}/-i = -\sqrt{i}
%\sqrt i = (1+i)/\sqrt{2}
%\sqrt -i = i(1+i)/\sqrt{2} = (i-1)/\sqrt{2}

%i\sqrt i = (i-1)/\sqrt{2}

%For the \([0,1]\) interval we have
%
%\begin{align*}
%\inv{4\sqrt{i}} \ln\left({\frac{\sqrt{2} - 1 -i}{\sqrt{2} + i + 1}}\right)
%+\inv{4i\sqrt{i}}\ln\left( \frac{\sqrt{2} - i +1}{\sqrt{2} + i - 1} \right) \\
%+\sqrt{2}\frac{i-1}{4(i+1)} i\pi
%\end{align*}
%(1/(4 sqrt(i))) ln( (sqrt(2) - 1 -i)/(sqrt(2) + i + 1)) +(1/(4 i sqrt(i))) ln ( (sqrt(2) - i +1)/(sqrt(2) + i - 1) ) +sqrt(2) ((i-1)/(4(i+1))) i ln(-1)

%As a check, we should get the \(\pi \sqrt{2}\) calculated with the contour integral
%
%\begin{align*}
%&\lim_{R \rightarrow \infty}
%\left(
%\inv{4\sqrt{i}} \ln\left({\frac{R - \sqrt{i}}{R + \sqrt{i}}} \frac {R - \sqrt{i}} {R + \sqrt{i}}\right)
%+\inv{4i\sqrt{i}}\ln\left( \frac{R - \sqrt{-i}}{R + \sqrt{-i}} \frac {R - \sqrt{-i}} {R + \sqrt{-i}} \right)
%\right) \\
%&=
%\lim_{R \rightarrow \infty}
%\left(
%\inv{2\sqrt{i}} \ln\left({\frac{R - \sqrt{i}}{R + \sqrt{i}}} \right)
%+\inv{2i\sqrt{i}}\ln\left( \frac{R - \sqrt{-i}}{R + \sqrt{-i}} \right)
%\right)
%\end{align*}

%(i-1)(i+1) = i^2 -1 = -2
%(i+1)(i+1) = i^2 +1 +2i = 2i

It should be possible to simplify this, or use it to verify the contour integral, or directly evaluate the integral for the \([0,1]\) range of the problem,
but this particular form is proving somewhat intractable (or I have made mistakes).  A lazier way is to invoke \href{http://integrals.wolfram.com/index.jsp}{webmathematica}, which gives

\begin{equation}\label{eqn:qmdCh2quiz:800}
\begin{aligned}
\int \frac{1 + x^2}{1 + x^4} dx &=
\frac{-\tan^{-1}(1 - \sqrt{2} x) + \tan^{-1}(1 + \sqrt{2} x)}{ \sqrt{2} }
\end{aligned}
\end{equation}
\begin{equation}\label{eqn:qmdCh2quiz:820}
\begin{aligned}
\int C^2 \frac{1 + x^2}{1 + x^4} dx &=
\inv{2 \pi} (-\tan^{-1}(1 - \sqrt{2} x) + \tan^{-1}(1 + \sqrt{2} x))
\end{aligned}
\end{equation}

However calculating this for \(x=0\) gives zero and for \(x=1\) gives \(0.25\), whereas the text gives \(\approx 0.52\).  The text can not be correct since the density is symmetric, which would imply that the probability to find it in \([-1,1]\) is \(1.04 > 1\).

% google calculator:
%((arctan(1+sqrt(2)) - arctan(1-sqrt(2)))/sqrt(2))/(pi sqrt(2))

\section{Inverse first order wave function}

This problem was to normalize

\begin{equation}\label{eqn:qmdCh2quiz:840}
\begin{aligned}
\psi = \inv{x}e^{i\omega t}\quad x \in [1,2]
\end{aligned}
\end{equation}

and to calculate the probability to find the particle in \([1.5,2]\).

\subsection{normalization}

\begin{equation}\label{eqn:qmdCh2quiz:860}
\begin{aligned}
\int \Abs{\phi}^2 dx
&= \int \inv{x^2} dx \\
&= \inv{1} -\inv{2} \\
&= \inv{2}
\end{aligned}
\end{equation}
%\int phi^2 = 1/2
%\int 2 phi^2 = 1

So our normalized wave function is

\begin{equation}\label{eqn:qmdCh2quiz:880}
\begin{aligned}
\psi = \frac{\sqrt{2}}{x}e^{i\omega t}
\end{aligned}
\end{equation}

\subsection{probability in a range}

By inspection this probability is
\begin{equation}\label{eqn:qmdCh2quiz:900}
\begin{aligned}
P(a,b) = {2}\left(\inv{a} - \inv{b}\right)
\end{aligned}
\end{equation}

So for \([1.5,2]\) we have

\begin{equation}\label{eqn:qmdCh2quiz:920}
\begin{aligned}
P = {2}\left(\frac{2}{3} - \inv{2}\right) = \inv{3} % 2(4/6 - 3/6) = 2/6
\end{aligned}
\end{equation}

\subsection{Probability current}

This wave function provides a super simple example to try a current
calculation with.

\begin{equation}\label{eqn:qmdCh2quiz:940}
\begin{aligned}
J
&= \frac{\Hbar}{2mi}\left(\psi^\conj\psi_x - \psi {\psi^\conj}_x \right) \\
\end{aligned}
\end{equation}

The time factor will cancel out, leaving

\begin{equation}\label{eqn:qmdCh2quiz:960}
\begin{aligned}
J
&= \frac{\Hbar}{2mi}\left(\psi\psi_x - \psi {\psi}_x \right) \\
&= 0
\end{aligned}
\end{equation}

Okay, that is a too simple probability calculation exercise!  Not interesting.

\subsection{expectation values}

\subsubsection{position}
Okay, those were pretty easy integration exercises.  How about using these
to verify the Heisenberg uncertainty principle.  That should be easy
enough with this simple wavefunction.

\begin{equation}\label{eqn:qmdCh2quiz:980}
\begin{aligned}
<x> &= \int_1^2 x \frac{2}{x^2} dx = 2 (\ln(2) - \ln(1)) = 2 \ln(2) \\
<x^2> &= \int_1^2 x^2 \frac{2}{x^2} dx = 2 - 1 = 1 \\
\end{aligned}
\end{equation}

Using the formula for standard deviation on page 50 we have
\begin{equation}\label{eqn:qmdCh2quiz:1000}
\begin{aligned}
\Delta x &= \sqrt{<x^2> - <x>^2} \\
&= \sqrt{1 - (2 \ln(2))^2}
\end{aligned}
\end{equation}

which is a complex number?

Let us go back to the statistical definition of standard deviation from school and see if this makes sense.

\begin{equation}\label{eqn:qmdCh2quiz:1020}
\begin{aligned}
\sigma^2
&= E( x - \overbar{x} )^2 \\
&= \inv{N} \sum (x_i - \overbar{x})^2 \\
&= \inv{N} \sum \left({x_i}^2 - 2 x_i \overbar{x} + {\overbar{x}}^2\right) \\
&= \inv{N} \sum {x_i}^2 - 2 {\overbar{x}}^2 + \frac{N}{N} {\overbar{x}}^2 \\
&= \inv{N} \sum {x_i}^2 - {\overbar{x}}^2 \\
\end{aligned}
\end{equation}

In the QM notation this is

\begin{equation}\label{eqn:qmdCh2quiz:1040}
\begin{aligned}
(\Delta x)^2 &= <x^2> - {<x>}^2
\end{aligned}
\end{equation}

which is strictly positive.  Okay, so I must have a mistake above somewhere.

Let us look at the expectation value of \(x\).  Does \(2 \ln(2) = 1.386\) make sense?  How about a discrete average that approximates it

\begin{equation}\label{eqn:qmdCh2quiz:1060}
\begin{aligned}
\frac{2}{3}\left( 1 \frac{1}{1^2} + 1.5 \inv{1.5^2} + 2 \inv{2^2} \right) = \frac{13}{9} = 1.444
\end{aligned}
\end{equation}

Okay, this makes sense, and it does not make sense in the discrete approximation that \(<x^2>\) could be lower, so that integration
must be wrong.  Take \(II\)

\begin{equation}\label{eqn:qmdCh2quiz:1080}
\begin{aligned}
<x^2> &= \int_1^2 x^2 \frac{2}{x^2} dx = {\left.2x\right\vert}_1^2 = 4 - 2 = 2 \\
\end{aligned}
\end{equation}

Okay, that is better ... dumb mistake. Our std deviation is therefore

\begin{equation}\label{eqn:qmdCh2quiz:1100}
\begin{aligned}
\Delta x = \sqrt{2 - 4 (\ln(2))^2} \approx 2.80
\end{aligned}
\end{equation}

\subsubsection{momentum}

Now, how about the momentum variance?

\begin{equation}\label{eqn:qmdCh2quiz:1120}
\begin{aligned}
<p>
&= -i \Hbar \int_1^2 2 \inv{x} \left(\inv{x}\right)' dx \\
&= 2 i \Hbar \int_1^2 \inv{x^3} dx \\
%&= - i \Hbar \left(\inv{4} - \inv{1}\right) dx \\
&= i \Hbar \left(-\inv{4} + \inv{1}\right) dx \\
&= \frac{3}{4} i \Hbar
\end{aligned}
\end{equation}

\begin{equation}\label{eqn:qmdCh2quiz:1140}
\begin{aligned}
<p^2>
&= -\Hbar^2 \int_1^2 2 \inv{x} \left(\inv{x}\right)'' dx \\
&= -\Hbar^2 \int_1^2 2 \inv{x} \left(-\inv{x^2}\right)' dx \\
&= - 4 \Hbar^2 \int_1^2 \inv{x^4} dx \\
&= - \frac{4}{3} \Hbar^2 \left( -\inv{8} + 1 \right) \\
&= - \frac{7}{6} \Hbar^2
\end{aligned}
\end{equation}

\begin{equation}\label{eqn:qmdCh2quiz:1160}
\begin{aligned}
\frac{<p^2>  - {<p>}^2 }{\Hbar^2}
&= \inv{2}\left(-\frac{7}{3} +\frac{9}{8}\right) \\
&= -\frac{29}{48}
\end{aligned}
\end{equation}

This gives

\begin{equation}\label{eqn:qmdCh2quiz:1180}
\begin{aligned}
\Delta x \Delta p = i \Hbar \sqrt{\frac{29}{24}(1 - 2 (\ln(2))^2) } \approx (0.217) i \Hbar
\end{aligned}
\end{equation}
%sqrt((29/24) (1 - 2 (ln(2))^2))
% 1/2 = 0.5

We expect
\begin{equation}\label{eqn:qmdCh2quiz:1200}
\begin{aligned}
\Delta x \Delta p > \Hbar/2
\end{aligned}
\end{equation}

and ended up with an imaginary value where the \(\Hbar\) factor is less than \(0.5\).  Something is fishy here, and I
do not think it is my algebra this time.

How about
\begin{equation}\label{eqn:qmdCh2quiz:1220}
\begin{aligned}
\frac{\Abs{<p^2>}  - \Abs{<p>}^2 }{\Hbar^2}
&= \inv{2}\left(\frac{7}{3} -\frac{9}{8}\right) \\
&= \frac{29}{48}
\end{aligned}
\end{equation}
%7(8) - 3(9) = 29

Not any different, except that the factor of \(i\) vanishes.  Now the
\href{http://en.wikipedia.org/wiki/Uncertainty_principle#Matrix_mechanics}{wikipedia uncertainly article}
presents this way differently.  How to reconcile the ideas here?

\section{Problem 4}

Unnormalized wavefunction is

\begin{equation}\label{eqn:qmdCh2quiz:1240}
\begin{aligned}
\psi = \inv{x^2 + 9}
\end{aligned}
\end{equation}

\subsection{Second order pole contour integral}

Now, in the problem, the normalization is given and only the position expectation and variance is asked for.  This
normalization factor is interesting to calculate however, since to do the contour integral for this one we have to deal
with a double pole, and I had also forgotten how to do those.

Suppose, again, that we have a regular function \(f(z)\) in the neighborhood
of \(z_0\).  We want to calculate the double pole integral at that point.

\begin{equation}\label{eqn:qmdCh2quiz:1260}
\begin{aligned}
I = \oint \frac{f(z)}{(z-z_0)^2} dz
\end{aligned}
\end{equation}

Integration by parts looks like the way to go.

\begin{equation}\label{eqn:qmdCh2quiz:1280}
\begin{aligned}
I
&= \oint \frac{f(z)}{(z-z_0)^2} dz \\
&= \oint {f(z)}\left(\frac{-1}{z-z_0}\right)' dz \\
&= \oint \left( \left(-\frac{f(z)}{z-z_0}\right)' +\frac{{f(z)}'}{z-z_0} \right) dz \\
\end{aligned}
\end{equation}

Now, for a circular contour around \(z_0\), we have
\(z = z_0 + R e^{i\theta} = z_0 + R e^{i(\theta+2\pi)}\), so

\begin{equation}\label{eqn:qmdCh2quiz:1300}
\begin{aligned}
{\left.-\frac{f(z)}{z-z_0}\right\vert}_{z_0 + R e^{i\phi}}^{z_0 + R e^{i(\phi + 2\pi)}}
&= -\frac{f(z_0 + R e^{i\phi} - f(z_0 + R e^{i(\phi + 2\pi)}} e^{-i\phi} \\
&= 0
\end{aligned}
\end{equation}

So the integral of the first term is zero, and we know how to deal with second provided the derivative is regular at the point of interest.

\begin{equation}\label{eqn:qmdCh2quiz:1320}
\begin{aligned}
\oint \frac{f(z)}{(z-z_0)^2} dz
&= \oint \frac{{f(z)}'}{z-z_0} dz \\
&= 2 \pi i {\left.f(z)'\right\vert}_{z=z_0}
\end{aligned}
\end{equation}

\subsection{normalize}

Now, we are set to normalize the wave function.  We have a pole at \(\pm 3i\)

\begin{equation}\label{eqn:qmdCh2quiz:1340}
\begin{aligned}
I
&= \int \Abs{\psi}^2  \\
&= \int_{z=-\infty}^{\infty} dz \inv{(z - 3i)^2} \inv{(z + 3i)^2} \\
\end{aligned}
\end{equation}

Picking a semicircular arc we have

\begin{equation}\label{eqn:qmdCh2quiz:1360}
\begin{aligned}
I + -2 \pi i {\left.\left(\inv{(z + 3i)^2}\right)'\right\vert}_{z=3i} = 0
\end{aligned}
\end{equation}

The integral is therefore
\begin{equation}\label{eqn:qmdCh2quiz:1380}
\begin{aligned}
I
&= 2 \pi i (-2)\inv{(6i)^3} \\
&= \pi \inv{54} \\
\end{aligned}
\end{equation}
%4/6 1/36
%2/3 1/36
%1/3 1/18
%1/54

and the normalized wave function is

\begin{equation}\label{eqn:qmdCh2quiz:1400}
\begin{aligned}
\psi = \sqrt{\frac{54}{\pi}}\inv{x^2 + 9}
\end{aligned}
\end{equation}

as given in the problem.

\subsection{expectation and variance values}

The expectation value for the position operator is just

\begin{equation}\label{eqn:qmdCh2quiz:1420}
\begin{aligned}
<x> = \frac{54}{\pi} \int \frac{x}{(x^2 + 9)^2} dx = 0
\end{aligned}
\end{equation}

Since it is an odd function.  For the square we have

\begin{equation}\label{eqn:qmdCh2quiz:1440}
\begin{aligned}
<x^2>
&= \frac{54}{\pi} \int \frac{x^2}{(x^2 + 9)^2} dx \\
&= \frac{54}{\pi} 2 \pi i {\left.\left(\frac{x^2}{(x + 3i)^2}\right)'\right\vert}_{3i} \\
&= \cdots \quad \mbox{\text{some algebra}} \\
&= 9
\end{aligned}
\end{equation}

Now, how about the momentum?  This one is odd too, and therefore zero
\begin{equation}\label{eqn:qmdCh2quiz:1460}
\begin{aligned}
<p> = -i \Hbar \frac{54}{\pi} \int \frac{1}{(x^2 + 9)} \frac{-2x}{(x^2 + 9)^2} dx = 0
\end{aligned}
\end{equation}

Last we have the squared momentum operator expectation
\begin{equation}\label{eqn:qmdCh2quiz:1480}
\begin{aligned}
<p^2>
&= - \Hbar^2 \frac{54}{\pi} \int \frac{1}{(x^2 + 9)} \left(\frac{-2x}{(x^2 + 9)^2}\right)' dx \\
&= - \Hbar^2 \frac{54}{\pi} \int \frac{1}{(x^2 + 9)}
\left(\frac{-2}{(x^2 + 9)^2} - {2x}\frac{-2 (2)(x^2+9)(2x)}{(x^2 + 9)^4} \right)
dx \\
&= - \Hbar^2 \frac{54}{\pi} \int \frac{-2}{(x^2 + 9)^3}
\left(1 - {8x^2}\frac{1}{x^2 + 9} \right)
dx \\
&= - \Hbar^2 \frac{54}{\pi} \int \frac{-2}{(x^2 + 9)^4} \left(9 - 7x^2 \right) dx \\
&= 2 \Hbar^2 \frac{54}{\pi} \int \frac{1}{(x - 3i)^4} \frac{1}{(x + 3i)^4} \left(9 - 7x^2 \right) dx \\
\end{aligned}
\end{equation}

Damn.  Now we need a fourth order pole (third derivative) residue.  This is getting messy.  Backing up one step to put
things in a nice form for cheating with Mathematica we have

\begin{equation}\label{eqn:qmdCh2quiz:1500}
\begin{aligned}
<p^2> &= \Hbar^2 \frac{108}{\pi} \int \frac{1}{(x^2 + 9)^4} \left(9 - 7x^2 \right) dx \\
\end{aligned}
\end{equation}
% \frac{1}{(x^2 + 9)^4} \left(9 - 7x^2 \right) dx \\
% ==>
% (9 - 7x^2 )/((x^2 + 9)^4)

and the cheat gives us

\begin{equation}\label{eqn:qmdCh2quiz:1520}
\begin{aligned}
\int \frac{1}{(x^2 + 9)^4} \left(9 - 7x^2 \right) dx
&= \frac{-3 x (-729 + 24 x^2 + x^4) + (9 + x^2)^3 \tan^{-1}(x/3)}{1944 (9 + x^2)^3}
\end{aligned}
\end{equation}

The first term will drop out for the infinite range, leaving

\begin{equation}\label{eqn:qmdCh2quiz:1540}
\begin{aligned}
2\tan^{-1}(\infty/3)\inv{1944} = \pi \inv{1944}
\end{aligned}
\end{equation}

So, if all went well we have
\begin{equation}\label{eqn:qmdCh2quiz:1560}
\begin{aligned}
<p^2>
&= \Hbar^2 \frac{108}{\pi} \pi \inv{1944} \\
&= \Hbar^2 \inv{18}
\end{aligned}
\end{equation}

This provides another numerical verification of the Heisenberg uncertainty relation

\begin{equation}\label{eqn:qmdCh2quiz:1580}
\begin{aligned}
\Delta p \Delta x &= \Hbar \inv{3 \sqrt{2}} 3 \approx 0.7 \Hbar > \Hbar/2
\end{aligned}
\end{equation}

\section{Problem 5}

Find position and momentum expectation values for

\begin{equation}\label{eqn:qmdCh2quiz:1600}
\begin{aligned}
\psi = A(x^5 - a x^3)
\end{aligned}
\end{equation}

... however, the problem does not define the range for the wave function.
For a symmetric finite interval it is simple enough to show that these
are zero, but the solution in the back of the text just wanted "ill posed"
for an infinite range.

\section{Problem 6. Is X operator Hermitian}

\begin{equation}\label{eqn:qmdCh2quiz:1620}
\begin{aligned}
<x>^\conj = \left(\int \psi^\conj x \psi dx\right)^\conj = <x>
\end{aligned}
\end{equation}

answer is therefore yes (with \(<ix>\) being Skew-Hermitian).

Momentum operator follows the same way with integration by parts, but I have
written that up recently in \chapcite{PJQmSusskind} so will not repeat it here.

\section{Problem 7. A current calculation}

\begin{equation}\label{eqn:qmdCh2quiz:1640}
\begin{aligned}
\psi = (A e^{ip\Hbar x} + B e^{-ip\Hbar x})e^{-i p^2 t/2 m\Hbar}
\end{aligned}
\end{equation}

The probability density is just
\begin{equation}\label{eqn:qmdCh2quiz:1660}
\begin{aligned}
\Abs{\psi}^2
&= (A e^{ip\Hbar x} + B e^{-ip\Hbar x}) (A^\conj e^{-ip\Hbar x} + B^\conj e^{ip\Hbar x}) \\
\end{aligned}
\end{equation}

But this does not have to be expanded for the continuity calculation, since we only want the time
derivative which is zero since this is not a function of time

\begin{equation}\label{eqn:qmdCh2quiz:1680}
\begin{aligned}
\PD{t}{\Abs{\psi}^2} &= 0.
\end{aligned}
\end{equation}

Now the current density is

\begin{equation}\label{eqn:qmdCh2quiz:1700}
\begin{aligned}
J = \frac{\Hbar}{2mi}\left( \psi^\conj \partial_x \psi - \psi \partial_x \psi^\conj \right)
\end{aligned}
\end{equation}

the spatial derivatives leave the time phase term untouched so the conjugation takes those out, leaving

\begin{equation}\label{eqn:qmdCh2quiz:1720}
\begin{aligned}
J
&= \frac{\Hbar}{2mi}
i p \Hbar \left((A^\conj e^{-ip\Hbar x} + B^\conj e^{ip\Hbar x}) (A e^{ip\Hbar x} - B e^{-ip\Hbar x})
-(A e^{ip\Hbar x} + B e^{-ip\Hbar x}) (-A^\conj e^{-ip\Hbar x} + B^\conj e^{ip\Hbar x}) \right) \\
&= \frac{\Hbar}{2mi} i p \Hbar \left( 2 \Abs{A}^2 - 2\Abs{B}^2\right) \\
\end{aligned}
\end{equation}

So we have \(\grad \cdot J = 0\) since this is a constant, and therefore

\begin{equation}\label{eqn:qmdCh2quiz:1740}
\begin{aligned}
\PD{t}{\rho} + \grad \cdot J = 0 + 0 = 0
\end{aligned}
\end{equation}

\section{Problem 8}

A square well problem in \([0,a]\) with

\begin{equation}\label{eqn:qmdCh2quiz:1760}
\begin{aligned}
\psi =
i
\frac{\sqrt{3}}{2}
\sqrt{\frac{2}{a}}
\sin\left( \frac{\pi x}{a} \right)
e^{-i E_1 t/\Hbar}
+ \inv{2}
\sqrt{\frac{2}{a}}
\sin\left( \frac{3 \pi x}{a} \right)
e^{-i E_3 t/\Hbar}
\end{aligned}
\end{equation}

Using separation of variables for the wave equation in this case we have

\begin{equation}\label{eqn:qmdCh2quiz:1780}
\begin{aligned}
-\frac{\Hbar^2}{2m} \frac{X''}{X} = i \Hbar \frac{T'}{T} = E
\end{aligned}
\end{equation}

We have

\begin{equation}\label{eqn:qmdCh2quiz:1800}
\begin{aligned}
T = e^{ -i E t/\Hbar }
\end{aligned}
\end{equation}

and

\begin{equation}\label{eqn:qmdCh2quiz:1820}
\begin{aligned}
X = A \sin\left( \frac{\sqrt{2 m E} x}{\Hbar} \right) = A \sin\left( \frac{k \pi x }{a}\right)
\end{aligned}
\end{equation}

To normalizing \(X\) we need squared sine

\begin{equation}\label{eqn:qmdCh2quiz:1840}
\begin{aligned}
\sin^2(u) = \inv{-4}(e^{2iu} + e^{-2iu} - 2) = \inv{-2}\cos(2u) + \inv{2}
\end{aligned}
\end{equation}

So we can integrate to find the normalization factor
\begin{equation}\label{eqn:qmdCh2quiz:1860}
\begin{aligned}
\int_0^a X^2 = A^2 \frac{a}{2} = 1
\end{aligned}
\end{equation}

for
\begin{equation}\label{eqn:qmdCh2quiz:1880}
\begin{aligned}
T_k &= e^{-i E_k t/\Hbar} \\
X_k &= \sqrt{\frac{2}{a}} \sin\left( \frac{k \pi x }{a}\right) \\
E_k &= \inv{2m } \left(\frac{k \pi x}{a} \right)^2 \\
\end{aligned}
\end{equation}

\subsection{Is it normalized?}

Yes, \(3/4 + 1/4 = 1\).

\subsection{What are the values of the energy?}

The work for this is above:

\begin{equation}\label{eqn:qmdCh2quiz:1900}
\begin{aligned}
E_1 &= \inv{2m } \left(\frac{\pi x}{a} \right)^2 \\
E_3 &= \inv{2m } \left(\frac{3 \pi x}{a} \right)^2
\end{aligned}
\end{equation}

\subsection{expectation of position}

We can write the wave function in terms of basis functions for convenience

\begin{equation}\label{eqn:qmdCh2quiz:1920}
\begin{aligned}
\phi_m &= \sqrt{\frac{2}{a}} \sin\left(\frac{k x \pi }{a}\right) \\
\psi &= i \frac{\sqrt{3}}{2} \phi_1 e^{-i E_1 t/\Hbar} + \inv{2} \phi_3 e^{-i E_3 t/\Hbar}
\end{aligned}
\end{equation}

or more generally

\begin{equation}\label{eqn:qmdCh2quiz:1940}
\begin{aligned}
\psi &= \sum_m c_m(t) \phi_m
\end{aligned}
\end{equation}

In terms of this Fourier series our position expectation is

\begin{equation}\label{eqn:qmdCh2quiz:1960}
\begin{aligned}
<x>
&= \sum_{m,n} \int c_m^\conj (\phi_m )^\conj x c_n \phi_n \\
&= \sum_{m,n} c_m^\conj c_n \int x \phi_m^\conj \phi_n \\
&= \sum_{m,n} c_m^\conj c_n \frac{2a }{\pi^2} \int_0^a (a \pi x/a) \sin(m \pi x/a) \sin( n \pi x/a) a (\pi dx/a) \\
% u = \pi x /a
% u(a) = \pi
&= \sum_{m,n} c_m^\conj c_n \frac{2 a}{\pi^2} \int_0^\pi u \sin(m u) \sin( n u) du \\
\end{aligned}
\end{equation}

For the integral for \(m \ne n\) we have

\begin{equation}\label{eqn:qmdCh2quiz:1980}
\begin{aligned}
\int u \sin(m u) \sin( n u) du &=
  \inv{2} \left(\frac{\cos((m - n) x)}{(m - n)^2} - \frac{\cos((m + n) x)}{(m + n)^2} \right)  \\
&+ \inv{2} \left(\frac{x \sin((m - n) x)}{m - n} - \frac{x \sin((m + n) x)}{m + n}\right) \\
\end{aligned}
\end{equation}

The sine terms will drop out at zero and \(\pi\), and the cosine terms will subtract out since they are the same at the boundaries.

\begin{equation}\label{eqn:qmdCh2quiz:2000}
\begin{aligned}
<x>
&= \sum_{n} \Abs{c_n}^2 \frac{2 a }{\pi^2} \int_0^\pi u \sin^2(n u) du \\
\end{aligned}
\end{equation}

Now, for the integral we have
\begin{equation}\label{eqn:qmdCh2quiz:2020}
\begin{aligned}
\int u \sin^2(n u) du &=
\frac{u^2}{4} - \frac{\cos(2 n u)}{8 n^2} - \frac{u \sin(2 n u)}{4 n}
\end{aligned}
\end{equation}

again the sine terms are zero, the cosines subtract away and we have only the
first term making a contribution at the upper bound.  This gives

\begin{equation}\label{eqn:qmdCh2quiz:2040}
\begin{aligned}
<x>
&= \sum_{n} \Abs{c_n}^2 \frac{2 a}{\pi^2} \frac{\pi^2}{4} \\
&= \frac{a}{2} \sum_{n} \Abs{c_n}^2 \\
&= \frac{a}{2}
\end{aligned}
\end{equation}

This is kind of cool.  A likely interpretation is that for any wave function whatsoever for this particle in the box we
have equal probability of finding the particle at any particular point.  Because of this it makes sense
that
the average value for the location of the particle is exactly the average of the positions available in the box.

\subsection{expectation of momentum}

For the momentum expectation we want

\begin{equation}\label{eqn:qmdCh2quiz:2060}
\begin{aligned}
<p>
&= \sum_{m,n} c_n^\conj c_m -i \Hbar \frac{2}{a} \int_0^a \sin( n \pi x /a ) (m \pi /a) \cos ( m \pi x /a ) dx (\pi/a) (a/\pi) \\
&= \sum_{m,n} c_n^\conj c_m -i \Hbar \frac{2 m}{a} \int_0^\pi \sin( n u ) \cos ( m u ) du \\
\end{aligned}
\end{equation}

Since the integral above is zero for all \(m,n\), we have for any wave function for the particle in a box

\begin{equation}\label{eqn:qmdCh2quiz:2080}
\begin{aligned}
<p> = 0
\end{aligned}
\end{equation}

This trivially shows that the statement of the problem that

\begin{equation}\label{eqn:qmdCh2quiz:2100}
\begin{aligned}
\frac{d m <x>}{dt} = <p>
\end{aligned}
\end{equation}

is in fact true, but this does not say much since both sides are zero.

%\bibliographystyle{plainnat}
%\bibliography{myrefs}

%\end{document}
