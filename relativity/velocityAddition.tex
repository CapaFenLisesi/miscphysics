%
% Copyright � 2012 Peeter Joot.  All Rights Reserved.
% Licenced as described in the file LICENSE under the root directory of this GIT repository.
%

%
%
%\documentclass{article}

%\input{../peeters_macros.tex}
%\input{../peeters_macros2.tex}

%\usepackage[bookmarks=true]{hyperref}

%\usepackage{color,cite,graphicx}
   % use colour in the document, put your citations as [1-4]
   % rather than [1,2,3,4] (it looks nicer, and the extended LaTeX2e
   % graphics package.
%\usepackage{latexsym,amssymb,epsf} % do not remember if these are
   % needed, but their inclusion can not do any damage


\chapter{Some notes on Pauli Relativity Velocity addition}
\label{chap:PJpauliVelocityAddition}
%\author{Peeter Joot \quad peeterjoot@protonmail.com}
\date{ Dec 25, 2008.  velocityAddition.tex }

%\begin{document}
%\maketitle{}
%
%\tableofcontents

\section{Motivation}

Fill out some details from part 1.6, velocity addition of \citep{pauli1981tr}.

\section{}

Given a path \(x^i(t')\) i n the moving (primed) frame \(S'\), the aim is to
express the observed velocities for this path from the rest frame \(S\).

From the Lorentz transformation of the coordinates we have (working with \(c=1\)) for a point in the moving frame

\begin{equation}\label{eqn:velocityAddition:20}
\begin{aligned}
x' &= \gamma ( x - vt) \\
t' &= \gamma ( t - vx)
\end{aligned}
\end{equation}

Or reversed (inverting velocities)

\begin{equation}\label{eqn:velocityAddition:40}
\begin{aligned}
x &= \gamma ( x' + vt') \\
t &= \gamma ( t' + vx')
\end{aligned}
\end{equation}

taking differentials we have

\begin{equation}\label{eqn:velocityAddition:60}
\begin{aligned}
dx &= \gamma ( dx' + v dt') \\
dy &= dy' \\
dz &= dz' \\
dt &= \gamma ( dt' + v dx')
\end{aligned}
\end{equation}

Dividing by \(dt'\), this is

\begin{equation}\label{eqn:velocityAddition:80}
\begin{aligned}
\frac{dx}{dt} &= \frac{ dx' + v dt' }{ dt' + v dx'} \\
\frac{dy}{dt} &= \frac{dy'}{\gamma (dt' + v dx')} \\
\frac{dz}{dt} &= \frac{dz'}{\gamma (dt' + v dx')} \\
\end{aligned}
\end{equation}

FIXME: do not like this dividing by differentials.  Try to re-express this
using the chain rule.

Or in terms of velocity coordinates we have Pauli's equation 10.

\begin{equation}\label{eqn:velocityAddition:100}
\begin{aligned}
u_x &= \frac{ {u_x}' + v  }{ 1 + v {u_x}'} \\
u_y &= \frac{{u_y}'}{\gamma (1 + v {u_x}')} \\
u_z &= \frac{{u_z}'}{\gamma (1 + v {u_x}')} \\
\end{aligned}
\end{equation}

Next he writes \(u^2 = \sum_i {u_i}^2\), in terms of the primed velocities

\begin{equation}\label{eqn:velocityAddition:120}
\begin{aligned}
u^2 &= {u_x}^2 +{u_y}^2 +{u_z}^2  \\
&=
\inv{( 1 + v {u_x}' )^2 } \left(
( {u_x}' + v  )^2
+(1-v^2){{u_y}'}^2
+(1-v^2){{u_z}'}^2
\right) \\
&=
\inv{( 1 + v {u_x}' )^2 } \left(
{{u_x}'}^2
+ 2 {u_x}' v
+ v^2
+(1-v^2){{u_y}'}^2
+(1-v^2){{u_z}'}^2
+v^2 {{u_x}'}^2
-v^2 {{u_x}'}^2
\right) \\
&=
\inv{( 1 + v {u_x}' )^2 } \left(
{{u}'}^2
+ 2 {u_x}' v
+ v^2
+v^2 {{u_x}'}^2
-v^2 {{u}'}^2
\right) \\
\end{aligned}
\end{equation}

This leaves the squared velocity of the path as viewed from the rest frame as
\begin{equation}\label{eqn:velocity_addition:Usquare}
\begin{aligned}
u^2 &=
\inv{( 1 + v {u_x}' )^2 } \left(
{{u}'}^2(1-v^2)
+ 2 {u_x}' v
+v^2 (1 + {{u_x}'}^2)
\right)
\end{aligned}
\end{equation}

Now direction cosines for the velocity direction vector are introduced

\begin{equation}\label{eqn:velocityAddition:140}
\begin{aligned}
\ucap' &= (u_x', u_y', u_z')/u = (\cos\alpha', \cos\sigma', \cos\delta') \\
\ucap &= (u_x, u_y, u_z)/u = (\cos\alpha, \cos\sigma, \cos\delta)
\end{aligned}
\end{equation}

and in terms of the direction cosines one has equation 11:

\begin{equation}\label{eqn:velocityAddition:160}
\begin{aligned}
u^2 &=
\inv{( 1 + v {u_x}' )^2 } \left(
{{u}'}^2
+v^2
+ 2 u' \cos\alpha' v
+v^2 {u'}^2 ({\cos\alpha'}^2 - 1 )
\right) \\
&= \inv{( 1 + v u' \cos\alpha' )^2 } \left(
{{u}'}^2
+v^2
+ 2 u' \cos\alpha' v
-v^2 {u'}^2 \sin^2\alpha'
\right) \\
\end{aligned}
\end{equation}

Pauli's equation 11a is \(1-u^2\), which is then factored into a tidy form
\begin{equation}\label{eqn:velocityAddition:180}
\begin{aligned}
1- u^2
&= \inv{( 1 + v u' \cos\alpha' )^2 } \left(
1
+ 2 v u' \cos\alpha'
+v^2 {u'}^2 \cos^2\alpha'
-{{u}'}^2
-v^2
- 2 u' v \cos\alpha'
+v^2 {u'}^2 \sin^2\alpha'
\right) \\
&= \inv{( 1 + v u' \cos\alpha' )^2 } \left(
1
-{{u}'}^2
-v^2
+v^2 {u'}^2
\right) \\
&= \frac{(1 -v^2 ) (1 -{{u}'}^2)}
{( 1 + v u' \cos\alpha' )^2 }
\end{aligned}
\end{equation}

This provides the gamma factor for the effective velocity as observed from the rest frame

\begin{equation}\label{eqn:velocityAddition:200}
\begin{aligned}
\inv{\sqrt{1-u^2}}
&= \frac{ 1 + v u' \cos\alpha' }{\sqrt{1 -v^2 }\sqrt{1 -{u'}^2}}
\end{aligned}
\end{equation}

With the factors of c's retained, and for the special case where the velocity is colinear with the frame motion, this is part of the velocity addition equation found in many intro relativity treatments.  The generalization required is that instead of the second velocity itself we have the projection of that velocity
in the direction of the frame motion.
For the special
case of when the velocity \(u'\) is directed with the path of the moving frame,
the cosine will be unity, and the projection of that velocity in the frame motion direction is exactly the velocity to be compounded:

\begin{equation}\label{eqn:velocityAddition:220}
\begin{aligned}
\inv{\sqrt{1-u^2}}
&= \frac{ 1 + v u' }{\sqrt{1 -v^2 }\sqrt{1 -{u'}^2}}
\end{aligned}
\end{equation}

We have another special case, considering perpendicular motion, for which we have \(\cos\alpha' = 0\), and thus

\begin{equation}\label{eqn:velocityAddition:240}
\begin{aligned}
\inv{\sqrt{1-u^2}}
&= \frac{ 1 }{\sqrt{1 -v^2 }\sqrt{1 -{u'}^2}}
\end{aligned}
\end{equation}

Next he calculates \(tan \alpha\).  That is

\begin{equation}\label{eqn:velocityAddition:260}
\begin{aligned}
\tan\alpha
&= \frac{\sin\alpha}{\cos\alpha} \\
&= \frac{\sqrt{1-\cos^2\alpha}}{\cos\alpha} \\
&= \frac{\sqrt{1-(u_x/u)^2}}{u_x/u} \\
&= \frac{\sqrt{u^2- {u_x}^2}}{u_x} \\
\end{aligned}
\end{equation}

From \eqnref{eqn:velocity_addition:Usquare} we have
\begin{equation}\label{eqn:velocityAddition:280}
\begin{aligned}
u^2 - {u_x}^2
&= \inv{( 1 + v {u_x}' )^2 } \left(
{{u}'}^2(1-v^2)
+ 2 {u_x}' v
+v^2 (1 + {{u_x}'}^2)
-(u_x' + v)^2
\right) \\
&= \inv{( 1 + v {u_x}' )^2 } \left(
{{u}'}^2(1-v^2)
%+ 2 {u_x}' v
+v^2 (1 + {{u_x}'}^2)
- {u_x'}^2
- v^2
%- 2 u_x' v
\right) \\
&= \inv{( 1 + v {u_x}' )^2 } \left(
({{u}'}^2 -{u_x'}^2) (1-v^2)
\right) \\
&= \inv{( 1 + v {u_x}' )^2 } \left(
{{u}'}^2(1 -\cos^2\alpha') (1-v^2)
\right) \\
&= \inv{( 1 + v {u_x}' )^2 } \left(
{{u}'}^2 \sin^2\alpha' (1-v^2)
\right) \\
\end{aligned}
\end{equation}

So, the tangent is
\begin{equation}\label{eqn:velocityAddition:300}
\begin{aligned}
\tan\alpha
&= \pm \frac{ u' \sin\alpha' \sqrt{1-v^2}}{ {u_x}' + v  } \\
&= \pm \frac{ u' \sin\alpha' \sqrt{1-v^2}}{ u'\cos\alpha' + v  } \\
\end{aligned}
\end{equation}

which except for the \(\pm 1\) factor is Pauli's equation twelve.

Note that in this form we see some of the relative vector structure

\begin{equation}\label{eqn:velocityAddition:320}
\begin{aligned}
\frac{v \wedge \gamma_0}{v \cdot \gamma_0}
\end{aligned}
\end{equation}

of the STA four vector formulation (ie: sine and cosine mapping to rejection and projection terms respectively onto the timelike direction).

\subsection{Perpendicular direction cosines}

What are the equivalent relations for the \(y\) and \(z\) direction cosines for the velocity between the two frames?  For the \(y\) direction, our \(\sin^2\sigma\) is

\begin{equation}\label{eqn:velocityAddition:340}
\begin{aligned}
u^2 - u_y^2
&=
\inv{( 1 + v {u_x}' )^2 } \left(
{{u}'}^2(1-v^2)
+ 2 {u_x}' v
+v^2 (1 + {{u_x}'}^2)
-{{u_y}'}^2/\gamma^2
\right) \\
&=
\inv{( 1 + v {u_x}' )^2 } \left(
({{u}'}^2 - {{u_y}'}^2 -{{u_x}'}^2)(1-v^2)
+ 2 {u_x}' v
+v^2
+ {{u_x}'}^2
\right) \\
&=
\inv{( 1 + v {u_x}' )^2 } \left(
({{u}'}^2 - {{u_y}'}^2 -{{u_x}'}^2)(1-v^2)
+ ({u_x}' + v)^2
\right) \\
&=
\inv{( 1 + v {u_x}' )^2 } \left(
{{u_z}'}^2(1-v^2)
+ ({u_x}' + v)^2
\right) \\
\end{aligned}
\end{equation}

This leaves us with a tangent of

\begin{equation}\label{eqn:velocityAddition:360}
\begin{aligned}
\tan^2\sigma
&=
\inv{( 1 + v {u_x}' )^2 } \left(
{{u_z}'}^2(1-v^2)
+ ({u_x}' + v)^2
\right)
/\left(\frac{{u_y}'}{\gamma (1 + v {u_x}')}\right)^2 \\
&=
\frac{ \left(
{{u_z}'}^2(1-v^2)
+ ({u_x}' + v)^2
\right)
}{(1-v^2){{u_y}'}^2}
\end{aligned}
\end{equation}

which leaves
\begin{equation}\label{eqn:velocityAddition:380}
\begin{aligned}
\tan\sigma
&= \pm \frac{ \sqrt{ \cos^2\delta' + \inv{1-v^2}(\cos\alpha' + v)^2 } }{\cos\sigma'}
\end{aligned}
\end{equation}

This is now enough to completely assemble the
velocity vector as observed from the rest frame

\begin{equation}\label{eqn:velocityAddition:400}
\begin{aligned}
\Bu &=
\sqrt{1 -\frac{(1 -v^2 ) (1 -{{u}'}^2)}{( 1 + v u' \cos\alpha' )^2 }}
\begin{bmatrix}
\cos\left(\tan^{-1}\left(\frac{ u' \sin\alpha' \sqrt{1-v^2}}{ u'\cos\alpha' + v  }\right)\right) \\
\cos\left(\tan^{-1}\left(\frac{ \sqrt{ \cos^2\delta' + \inv{1-v^2}(\cos\alpha' + v)^2 } }{\cos\sigma'}\right)\right) \\
\cos\left(\tan^{-1}\left(\frac{ \sqrt{ \cos^2\sigma' + \inv{1-v^2}(\cos\alpha' + v)^2 } }{\cos\delta'}\right)\right)
\end{bmatrix}
\end{aligned}
\end{equation}

Wow.  What a mess (assuming I even got the algebra right)!

%\bibliographystyle{plainnat}
%\bibliography{myrefs}

%\end{document}
