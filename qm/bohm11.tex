%
% Copyright � 2012 Peeter Joot.  All Rights Reserved.
% Licenced as described in the file LICENSE under the root directory of this GIT repository.
%

%
%
%\documentclass{article}

%\input{../peeters_macros.tex}
%\input{../peeters_macros2.tex}

%\usepackage{listings}
%\usepackage{txfonts} % for ointctr... (also appears to make "prettier" \int and \sum's)
% makes \grad look funny though (almost like spacegrad, but narrower)
%\usepackage[bookmarks=true]{hyperref}

%\usepackage{color,cite,graphicx}
   % use colour in the document, put your citations as [1-4]
   % rather than [1,2,3,4] (it looks nicer, and the extended LaTeX2e
   % graphics package.
%\usepackage{latexsym,amssymb,epsf} % do not remember if these are
   % needed, but their inclusion can not do any damage


\chapter{QM notes and problems for Bohm, chapter 11}
\label{chap:bohm11}
%\author{Peeter Joot \quad peeterjoot@protonmail.com }
\date{ May 8, 2009.   bohm11.tex }

%\begin{document}

%\maketitle{}
%\tableofcontents
\section{Bohm Chapter 11 problems}

Problems and additional details from reading of \citep{bohm1989qt}, chapter 11.

\subsection{Problem 1.  Probability currents for step potential}

For this problem I include repetition of material covered in the text.
Aspects of the treatment were not clear, so attempting this myself
should clarify things.

\subsubsection{Setup}

This problem and the associated text has a step potential \(V\) for \(x>0\).  The
idea is that we have a left to right stream of particles with an associated
wave function, with reflected and transmitted coefficients.

Solutions to the stationary equation are sought in each of the intervals

\begin{equation}\label{eqn:bohm11:21}
\begin{aligned}
-\frac{\Hbar^2}{2m}\psi'' + (V-E)\psi = 0
\end{aligned}
\end{equation}

That is
\begin{equation}\label{eqn:bohm11:41}
\begin{aligned}
\psi'' = - \frac{2m}{\Hbar^2} (E-V)\psi
\end{aligned}
\end{equation}

With an assumption of exponential solutions on the left of the barrier
(ie: no decay and associated hyperbolic solutions in the \(V=0\) interval),
we must have \(E>0\).

So, the solution can be written as the sum

\begin{equation}\label{eqn:bohm11:61}
\begin{aligned}
\psi_1 = \sum A_{\pm} \exp\left( \pm i \sqrt{2mE} x / \Hbar \right)
\end{aligned}
\end{equation}

Similarly for \(x>0\), non-hyperbolic solutions are

\begin{equation}\label{eqn:bohm11:81}
\begin{aligned}
\psi_t = \sum B_{\pm} \exp\left( \pm i \sqrt{2m(E-V)} x / \Hbar \right)
\end{aligned}
\end{equation}

Mathematically, there is not anything that prevents picking \(E-V <0\) solutions

\begin{equation}\label{eqn:bohm11:101}
\begin{aligned}
\psi_t = \sum B_{\pm}' \exp\left( \pm \sqrt{-2m(E-V)} x / \Hbar \right)
\end{aligned}
\end{equation}

The book considers this case next, and this is also the subject of
problem 3.
%TODO: Come back to this and think it through.  Where will follow through using
%wave function continuity and derivative continuity with these solutions
%in the \(x>0\) interval?

Back to the \(E>V\) case, continuity at \(x=0\) (\(\psi_1(0) = \psi_t(0)\)) requires

\begin{equation}\label{eqn:bohm11:121}
\begin{aligned}
A_{+} + A_{-} = B_{+} + B_{-}
\end{aligned}
\end{equation}

Whereas derivative continuity, with \(p_1 = \sqrt{2mE}\), and \(p_2 = \sqrt{2m (E-V)}\) requires

\begin{equation}\label{eqn:bohm11:141}
\begin{aligned}
\frac{i p_1}{\Hbar} (A_{+} - A_{-}) = \frac{i p_2}{\Hbar} (B_{+} - B_{-})
\end{aligned}
\end{equation}

This is

\begin{equation}\label{eqn:bohm11:161}
\begin{aligned}
\begin{bmatrix}
1 & 1 \\
p_1 & -p_1
\end{bmatrix}
\begin{bmatrix}
A_{+} \\
A_{-}
\end{bmatrix}
=
\begin{bmatrix}
1 & 1 \\
p_2 & -p_2
\end{bmatrix}
\begin{bmatrix}
B_{+} \\
B_{-}
\end{bmatrix}
\end{aligned}
\end{equation}

Matrix inversion gives us the \(B\) coefficients in terms of \(A\)

\begin{equation}\label{eqn:bohm11:181}
\begin{aligned}
\begin{bmatrix}
B_{+} \\
B_{-}
\end{bmatrix}
&=
\inv{-2p_2}
\begin{bmatrix}
-p_2 & -1 \\
-p_2 & 1
\end{bmatrix}
\begin{bmatrix}
1 & 1 \\
p_1 & -p_1
\end{bmatrix}
\begin{bmatrix}
A_{+} \\
A_{-}
\end{bmatrix} \\
&=
\inv{2p_2}
\begin{bmatrix}
p_2 & 1 \\
p_2 & -1
\end{bmatrix}
\begin{bmatrix}
1 & 1 \\
p_1 & -p_1
\end{bmatrix}
\begin{bmatrix}
A_{+} \\
A_{-}
\end{bmatrix} \\
&=
\inv{2p_2}
\begin{bmatrix}
(p_1 + p_2) & (p_2 - p_1) \\
(p_2 - p_1) & (p_1 + p_2)
\end{bmatrix}
\begin{bmatrix}
A_{+} \\
A_{-}
\end{bmatrix} \\
\end{aligned}
\end{equation}

Or equivalently

\begin{equation}\label{eqn:bohm11:201}
\begin{aligned}
\begin{bmatrix}
A_{+} \\
A_{-}
\end{bmatrix}
&=
\inv{2p_1}
\begin{bmatrix}
(p_1 + p_2) & (p_1 - p_2) \\
(p_1 - p_2) & (p_1 + p_2)
\end{bmatrix}
\begin{bmatrix}
B_{+} \\
B_{-}
\end{bmatrix} \\
\end{aligned}
\end{equation}

Using this last, the
assumption of no further barriers in the \(x>0\) interval (ie: no reflection at
\(x = \infty\)), allows the physical situation to dictate \(B_{-} = 0\).  Then we
have

\begin{equation}\label{eqn:bohm11:221}
\begin{aligned}
\begin{bmatrix}
A_{+} \\
A_{-}
\end{bmatrix}
&=
\frac{B_{+} }{2p_1}
\begin{bmatrix}
p_1 + p_2 \\
p_1 - p_2
\end{bmatrix}
\end{aligned}
\end{equation}

The first is
\begin{equation}\label{eqn:bohm11:241}
\begin{aligned}
A_{+}
&=
\frac{B_{+} }{2p_1} (p_1 + p_2)
\end{aligned}
\end{equation}

Or
\begin{equation}\label{eqn:bohm11:261}
\begin{aligned}
B_{+}
&=
\frac{2 p_1 A_{+} }{p_1 + p_2}
\end{aligned}
\end{equation}

and the second is
\begin{equation}\label{eqn:bohm11:281}
\begin{aligned}
A_{-}
&=
\frac{B_{+} }{2p_1} (p_1 - p_2) \\
&=
\frac{2 p_1 A_{+} }{p_1 + p_2}
\frac{1}{2p_1} (p_1 - p_2) \\
&=
A_{+} \left(
\frac{ p_1 - p_2 }{p_1 + p_2} \right)
\end{aligned}
\end{equation}

This reduces the free parameters in the wave functions to the single
amplitude

\begin{equation}\label{eqn:bohm11:wavefunctions}
\begin{aligned}
\psi_1 &=
A_{+} e^{ i p_1 x/\Hbar }
+A_{+} \left(
\frac{ p_1 - p_2 }{p_1 + p_2} \right)
e^{ -i p_1 x/\Hbar } \\
\psi_t &= A_{+} \frac{2 p_1 }{p_1 + p_2} e^{ i p_2 x/\Hbar }
\end{aligned}
\end{equation}

Inspection provides a check that these do in fact satisfy the desired
continuity requirements at \(x=0\) (both the wave function and its derivative).

Also observe that a sign error typo in the text is implicitly fixed above (
sign factor of \(p_2\) in \(\psi_t\)).  That typo is corrected for right afterward when \(A,B,C\) is calculated, since it would otherwise result in a negative sign in the resulting linear equations there too.

\subsubsection{Probability currents}

From \eqnref{eqn:bohm11:wavefunctions}
the probability currents can be calculated.  The probability
conservation relation follows by taking time derivatives of \(\psi^\conj\psi\), as done in \citep{gabookII:PJprobCurrent}.  We have, for the \(x\) component of the current

\begin{equation}\label{eqn:bohm11:301}
\begin{aligned}
\PD{t}{\psi\psi^\conj} + \grad \cdot \left(
\frac{\Hbar}{2 m i}
\left(
\psi^\conj \grad \psi
-\psi \grad \psi^\conj
\right)
\right) = 0
\end{aligned}
\end{equation}

Or in short

\begin{equation}\label{eqn:bohm11:321}
\begin{aligned}
\PD{t}{\rho} + \grad \cdot \BJ = 0
\end{aligned}
\end{equation}

For \(\psi_t\) we have

\begin{equation}\label{eqn:bohm11:341}
\begin{aligned}
J_t
&=
\frac{\Hbar}{2 m i}
\left(
B_{+}^\conj e^{ -i p_2 x/\Hbar }
B_{+} (i p_2 /\Hbar)e^{ i p_2 x/\Hbar }
-B_{+} e^{ -i p_2 x/\Hbar }
B_{+}^\conj (-i p_2 /\Hbar)e^{ -i p_2 x/\Hbar }
\right) \\
&= \frac{ p_2 \Abs{ B_{+} }^2}{m } \\
\end{aligned}
\end{equation}

In terms of \(A_{+}\), this is

\begin{equation}\label{eqn:bohm11:currentTransmitted}
\begin{aligned}
J_t &= \frac{ 4 p_1^2 p_2 \Abs{ A_{+} }^2}{m (p_1 + p_2)^2}
\end{aligned}
\end{equation}

For \(\psi_1\) the \(x\) component of the current is

\begin{equation}\label{eqn:bohm11:361}
\begin{aligned}
J_1
&=
\frac{\Hbar}{2 m i}
\left(
(A_{+}^\conj e^{ -i p_1 x/\Hbar }
+A_{-}^\conj e^{ i p_1 x/\Hbar })
(A_{+} (i p_1 /\Hbar)e^{ i p_1 x/\Hbar }
+A_{-} (-i p_1 /\Hbar)e^{ -i p_1 x/\Hbar })
\right)
\\
&
-
\frac{\Hbar}{2 m i} \left(
(
A_{+} e^{ i p_1 x/\Hbar } +A_{-} e^{ -i p_1 x/\Hbar })
(A_{+}^\conj (-i p_1 /\Hbar)e^{ -i p_1 x/\Hbar }
+A_{-}^\conj (i p_1 /\Hbar)e^{ i p_1 x/\Hbar })
\right) \\
&=
\frac{p_1}{2m}
\left(
2\Abs{A_{+} }^2
+2\Abs{A_{-} }^2
+
A_{-}^\conj A_{+} e^{ 2 i p_1 x/\Hbar }
-A_{+}^\conj A_{-} e^{ -2 i p_1 x/\Hbar }
-A_{+} A_{-}^\conj e^{ 2 i p_1 x/\Hbar }
+A_{-} A_{+}^\conj e^{ -2 i p_1 x/\Hbar }
\right) \\
&=
\frac{p_1}{m}
\left(
\Abs{A_{+} }^2
-\Abs{A_{-} }^2
\right) \\
\end{aligned}
\end{equation}

Again in terms of \(A_{+}\), we have

\begin{equation}\label{eqn:bohm11:381}
\begin{aligned}
\Abs{A_{+} }^2 -\Abs{A_{-} }^2
&=
\Abs{A_{+}}^2 \left( 1 - \left( \frac{ p_1 - p_2 }{p_1 + p_2} \right)^2 \right) \\
&=
\Abs{A_{+}}^2
\frac{ 4 p_1 p_2 }{(p_1 + p_2)^2} \\
&=
4 \Abs{A_{+}}^2 \frac{ p_1 p_2 }{(p_1 + p_2)^2} \\
\end{aligned}
\end{equation}

Which provides the probability current for the sum of the incident and reflected wave functions

\begin{equation}\label{eqn:bohm11:current1}
\begin{aligned}
J_1
&=
\Abs{A_{+}}^2
\frac{p_1}{m}
\frac{ 4 p_1 p_2 }{(p_1 + p_2)^2}
\end{aligned}
\end{equation}

So we have \(J_1 = J_t\), and this completes the
part of the problem that was to show that the currents in the \(x<0\), and \(x>0\)
intervals are equal.

The individual incident and reflected currents can also be calculated.  These are

\begin{equation}\label{eqn:bohm11:401}
\begin{aligned}
J_i &= \frac{ p_1}{m} \Abs{ A_{+} }^2 \\
J_r &= \frac{ p_1}{m} \Abs{ A_{-} }^2  \\
&=
\Abs{ A_{+} }^2
\frac{ p_1 }{m }
\left( \frac{ p_1 - p_2 }{p_1 + p_2} \right)^2
\end{aligned}
\end{equation}

% WRONG:
%Perhaps unsurprisingly, we see that the total probability current for \(x<0\), is the sum of the incident and reflected
%currents \(J_1 = J_i + J_r\).

\subsubsection{Transmission and reflection coefficients}

As in the text we can calculate the reflection coefficient, the ratio of the magnitudes of the reflected and the incident currents,

\begin{equation}\label{eqn:bohm11:421}
\begin{aligned}
R
&= \frac{J_r}{J_i}  \\
&=
\frac{\Abs{ A_{+} }^2
\frac{ p_1 }{m }
\left( \frac{ p_1 - p_2 }{p_1 + p_2} \right)^2 }{
\frac{ p_1}{m}
\Abs{ A_{+} }^2 } \\
&=
\left( \frac{ p_1 - p_2 }{p_1 + p_2} \right)^2
\\
\end{aligned}
\end{equation}

and can calculate the transmission coefficient, the ratio of the transmitted current \(J_t\), to the incident current \(J_i\).  This is

\begin{equation}\label{eqn:bohm11:441}
\begin{aligned}
T
&= \frac{J_t}{J_i} \\
&=
\frac{
\frac{ 4 p_1^2 p_2 \Abs{ A_{+} }^2}{m (p_1 + p_2)^2}
}
{
\frac{ p_1}{m} \Abs{ A_{+} }^2
} \\
&=
\frac{ 4 p_1 p_2 }{ (p_1 + p_2)^2}
\end{aligned}
\end{equation}

Without first calculating the currents explicitly it was not clear to me how \(T\) was calculated in the text, and doing this first
makes that a bit more sensible.

\subsubsection{Compare the currents to the probability density}

From \eqnref{eqn:bohm11:wavefunctions}, the probability density in the two intervals
can be calculated.  For \(x<0\), we have

\begin{equation}\label{eqn:bohm11:461}
\begin{aligned}
\rho
&= \psi_1^\conj \psi_1 \\
&=
\Abs{ A_{+} }^2
\left(
e^{ -i p_1 x/\Hbar }
+\left( \frac{ p_1 - p_2 }{p_1 + p_2} \right) e^{ i p_1 x/\Hbar }
\right)
\left(
e^{ i p_1 x/\Hbar }
+\left( \frac{ p_1 - p_2 }{p_1 + p_2} \right) e^{ -i p_1 x/\Hbar }
\right)
\\
&=
\Abs{ A_{+} }^2
\left(
1
+
\left( \frac{ p_1 - p_2 }{p_1 + p_2} \right)^2
+
2 \left( \frac{ p_1 - p_2 }{p_1 + p_2} \right) \cos\left( 2 p_1 x/\Hbar \right)
\right) \\
% MISTAKE BELOW.  does not yeild continuity at \(x=0\) whereas this (just above) does.
%&=
%\Abs{ A_{+} }^2
%\left(
%1
%+
%\left( \frac{ p_1 - p_2 }{p_1 + p_2} \right)^2
%+
%2 \left( \frac{ p_1 - p_2 }{p_1 + p_2} \right) \left( 2 \cos^2\left( p_1 x/\Hbar \right) - 1\right)
%\right) \\
%&=
%\Abs{ A_{+} }^2
%\left(
%\left(1 - \left( \frac{ p_1 - p_2 }{p_1 + p_2} \right)\right)^2
%+
%4 \left( \frac{ p_1 - p_2 }{p_1 + p_2} \right) \cos^2\left( p_1 x/\Hbar \right)
%\right) \\
%&=
%4 \Abs{ A_{+} }^2
%\left(
%\left( \frac{ p_1 p_2 }{p_1 + p_2} \right)^2
%+
%\left( \frac{ p_1 - p_2 }{p_1 + p_2} \right) \cos^2\left( p_1 x/\Hbar \right)
%\right)
%\\
\end{aligned}
\end{equation}

And in the \(x>0\) interval we have
\begin{equation}\label{eqn:bohm11:481}
\begin{aligned}
\rho
&= \psi_t^\conj \psi_t \\
&=
\Abs{A_{+}}^2 \frac{4 p_1^2 }{(p_1 + p_2)^2}
\end{aligned}
\end{equation}

Comparing to \eqnref{eqn:bohm11:currentTransmitted}, we see that

\begin{equation}\label{eqn:bohm11:501}
\begin{aligned}
J_t &= \frac{ p_2 }{ m } \rho
\end{aligned}
\end{equation}

Writing \(v_t = p_2/m\) for the velocity associated with the current, we have
the desired relation between the current and probability density in the \(x>0\) interval (the transmitted current and probability density).

\begin{equation}\label{eqn:bohm11:521}
\begin{aligned}
J_t &= v_t \rho
\end{aligned}
\end{equation}

There does not appear to be an such simple relationship between the currents
in the \(x<0\) interval where we have probability interference.

\subsection{Problem 2.  Continuity and probability current conservation}

Show that at \(x=0\) the continuity of these wave functions and their derivatives
imply current conservation at \(x=0\).  This follows from the fact that \(J_1 = J_t\), which is
true not just at \(x=0\) since these are both constant.

\subsection{Problem 3.  Calculate probability current for \texorpdfstring{\(E<V\)}{E less than V}}

For the \(E<V\) case in the step potential problem above, the solution in the
text is found to be for the \(x<0\) interval

\begin{equation}\label{eqn:bohm11:541}
\begin{aligned}
\psi &= I \cos\left( \sqrt{2mE} x/\Hbar + \phi \right) \\
\tan\phi &= \sqrt{(V-E)/E}
\end{aligned}
\end{equation}

and in the \(x>0\) region is

\begin{equation}\label{eqn:bohm11:561}
\begin{aligned}
\psi = I \cos\left( \phi \right) \exp\left( -\sqrt{2m(V-E)} x/\Hbar \right)
\end{aligned}
\end{equation}

A quick check of the derivatives of these shows that we have continuity as desired.

That the probability current for this wave function is zero follows from the real nature of this solution, since for any real
wave function we have the one dimensional probability current as

\begin{equation}\label{eqn:bohm11:581}
\begin{aligned}
J
&= \inv{2mi}\left( \psi^\conj \psi' - \psi (\psi^\conj)' \right) \\
&= \inv{2mi}\left( \psi \psi' - \psi \psi' \right) \\
&= 0
\end{aligned}
\end{equation}

\subsection{Problem 4. Rectangular Barrier Tunneling}

This one is tackled separately in \ref{chap:qmBarrier}.

\subsection{Problem 5. Reflection coefficient for square well}

End result (straightforward calculation) was

\begin{equation}\label{eqn:bohm11:601}
\begin{aligned}
\Abs{\frac{E}{D}}^2 &= \frac{\inv{4}(p_1/p_2 - p_2/p_1)^2 \sin^2( 2 p_2 a /\Hbar)}{\cos^2(2 p_2 a/\Hbar) + \inv{4}(p_1/p_2 + p_2/p_1) \sin^2(2 p_2 a/\Hbar)}
\end{aligned}
\end{equation}

\subsection{Problem 6}

%\bibliographystyle{plainnat}
%\bibliography{myrefs}

%\end{document}
