%
% Copyright � 2012 Peeter Joot.  All Rights Reserved.
% Licenced as described in the file LICENSE under the root directory of this GIT repository.
%

%
%
%\documentclass{article}

%\input{../peeters_macros.tex}
%\input{../peeters_macros2.tex}

%\usepackage[bookmarks=true]{hyperref}

%\usepackage{color,cite,graphicx}
   % use colour in the document, put your citations as [1-4]
   % rather than [1,2,3,4] (it looks nicer, and the extended LaTeX2e
   % graphics package.
%\usepackage{latexsym,amssymb,epsf} % do not remember if these are
   % needed, but their inclusion can not do any damage

\chapter{Evaluating the Gaussian integral}
\label{chap:gaussian}
%\author{Peeter Joot \quad peeterjoot@protonmail.com}
\date{ Jan 05, 2009.  gaussian.tex }

%\begin{document}

%\maketitle{}
%\tableofcontents
\section{Plain old Gaussian}

QM solutions appear to involve a lot of Gaussian integrals.  Looking at one
of the problems in \citep{mcmahon2005qmd} I tried to recall how to evaluate
the simplest of these.  Google says the trick is squaring and polar
substitution.  Let us try this.

Solve
\begin{equation}\label{eqn:gaussian:20}
\begin{aligned}
I = \int_{-\infty}^\infty e^{-\alpha x^2} dx
\end{aligned}
\end{equation}

\begin{equation}\label{eqn:gaussian:40}
\begin{aligned}
I^2
&= \int_{x= -\infty}^\infty e^{-\alpha x^2} dx \int_{y = -\infty}^\infty e^{-\alpha y^2} dy \\
&= \int_{\theta=0}^{2\pi}\int_{r= 0}^\infty e^{-\alpha r^2} r dr d\theta \\
&= 2\pi
{\left.
\frac{e^{-\alpha r^2}}{-2\alpha}
\right\vert}_{r= 0}^\infty  \\
&= \frac{\pi}{\alpha}
\end{aligned}
\end{equation}

So we have

\begin{equation}\label{eqn:gaussian:60}
\begin{aligned}
I = \sqrt{\frac{\pi}{\alpha}}
\end{aligned}
\end{equation}

\section{A couple higher order Gaussian's and normalization exercise}

In order to do the normalization exercise for

\begin{equation}\label{eqn:gaussian:exercise}
\begin{aligned}
\psi = \left(A e^{-\frac{x^2}{a}} +B x e^{-\frac{x^2}{b}}\right) e^{-ict}
\end{aligned}
\end{equation}

We want to calculate
\begin{equation}\label{eqn:gaussian:80}
\begin{aligned}
\int \psi^\conj \psi =
\Abs{A}^2 e^{-2\frac{x^2}{a}} +\Abs{B}^2 x^2 e^{-2\frac{x^2}{b}}
+ (A\overbar{B} + B\overbar{A}) x e^{-x^2\left(\inv{a} + \inv{b}\right)}
\end{aligned}
\end{equation}

so we need the \(n=1,2\) versions of the following Gaussian integrals

\begin{equation}\label{eqn:gaussian:100}
\begin{aligned}
I_n = \int x^n e^{-\alpha x^2} dx
\end{aligned}
\end{equation}

The \(n=1\) case is directly integrable:

\begin{equation}\label{eqn:gaussian:120}
\begin{aligned}
I_1
&= \int x e^{-\alpha x^2} dx \\
&= \int \left(\frac{e^{-\alpha x^2}}{-2\alpha}\right)' dx \\
&= 0
\end{aligned}
\end{equation}

(integration bounds are \(\pm \infty\) so the exponential vanishes).

Next.  \(I_2\) follows with integration by parts

\begin{equation}\label{eqn:gaussian:140}
\begin{aligned}
I_2
&= \int x^2 e^{-\alpha x^2} dx \\
&= \int x \left(x e^{-\alpha x^2}\right) dx \\
&= \int x \left(\frac{e^{-\alpha x^2}}{-2\alpha}\right)' dx \\
&= -\int \frac{e^{-\alpha x^2}}{-2\alpha} dx \\
&= \inv{2\alpha}\int e^{-\alpha x^2} dx \\
&= \inv{2\alpha} \sqrt{\frac{\pi}{\alpha}}
\end{aligned}
\end{equation}

So the normalization required for \eqnref{eqn:gaussian:exercise} is
\begin{equation}\label{eqn:gaussian:160}
\begin{aligned}
1 = \Abs{A}^2 \sqrt{\frac{\pi a}{2}} + \frac{\Abs{B}^2}{2}
\sqrt{\frac{\pi b}{2}} \frac{b}{2}
\end{aligned}
\end{equation}

The values \(a\), and \(b\) are presumably due to boundary conditions, and this then fixes \(\Abs{A}\) in terms of \(\Abs{B}\) or the other way around.

\section{Generalized}

Let

\begin{equation}\label{eqn:gaussian:180}
\begin{aligned}
I_n = \int_{-\infty}^\infty x^n e^{-\alpha x^2} dx
\end{aligned}
\end{equation}

We have solved this for \(I_0 = \sqrt{\pi/\alpha}\), \(I_1 = 0\), and \(I_2\).  A quick calculation shows that \(I_{2k+1} = 0\) too:

\begin{equation}\label{eqn:gaussian:200}
\begin{aligned}
I_n
&= \int_{-\infty}^\infty x^n e^{-\alpha x^2} dx \\
&= \int_{0}^\infty x^n e^{-\alpha x^2} dx +\int_{-\infty}^0 x^n e^{-\alpha x^2} dx \\
&= \int_{0}^\infty x^n e^{-\alpha x^2} dx +\int_{\infty}^0 (-x)^n e^{-\alpha x^2} (-dx) \\
&= \int_{0}^\infty x^n e^{-\alpha x^2} dx +\int_0^{\infty} (-x)^n e^{-\alpha x^2} dx \\
&= \int_{0}^\infty (x^n + (-x)^n)e^{-\alpha x^2} dx \\
\end{aligned}
\end{equation}

But if \(n\) is odd \((-x)^n = -x^n\), so this is zero.

Now, for \(n\) even, we can integrate by parts, as done for \(I_2\).

\begin{equation}\label{eqn:gaussian:220}
\begin{aligned}
I_{2m}
&= \int x^{2m} e^{-\alpha x^2} dx \\
&= \int x^{2m-1} \left(x e^{-\alpha x^2}\right) dx \\
&= \int x^{2m-1} \left(\frac{e^{-\alpha x^2}}{-2\alpha}\right)' dx \\
&= -\int (2m-1) x^{2m-2} \frac{e^{-\alpha x^2}}{-2\alpha} dx \\
\end{aligned}
\end{equation}

This gives us a recurrence relationship for the even order terms
\begin{equation}\label{eqn:gaussian:240}
\begin{aligned}
I_{2m} &= \frac{2m-1}{2\alpha} I_{2m-2}.
%&= \inv{2\alpha} \sqrt{\frac{\pi}{\alpha}}
\end{aligned}
\end{equation}

Expanding this explicitly for the first few \(m\) shows the pattern

\begin{equation}\label{eqn:gaussian:260}
\begin{aligned}
\begin{array}{l l l}
I_2 &= \frac{2-1}{2\alpha} I_0 &= \frac{1}{2\alpha} \sqrt{\frac{\pi}{\alpha}} \\
I_4 &= \frac{4-1}{2\alpha}\frac{2-1}{2\alpha} I_0 &= \frac{3.1}{(2\alpha)^2} \sqrt{\frac{\pi}{\alpha}} \\
I_6 &= \frac{6-1}{2\alpha}\frac{4-1}{2\alpha}\frac{2-1}{2\alpha} I_0 &= \frac{5.3.1}{(2\alpha)^3} \sqrt{\frac{\pi}{\alpha}} \\
\end{array}
\end{aligned}
\end{equation}

Or
\begin{equation}\label{eqn:gaussian:280}
\begin{aligned}
I_{0} &= \sqrt{\frac{\pi}{\alpha}} \\
I_{2m-1} &= 0 \\
I_{2m} &= \frac{(2m-1)(2m-3)\cdots(3)(1)}{(2\alpha)^m} \sqrt{\frac{\pi}{\alpha}}
\end{aligned}
\end{equation}

%\bibliographystyle{plainnat}
%\bibliography{myrefs}

%\end{document}
