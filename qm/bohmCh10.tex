%
% Copyright � 2012 Peeter Joot.  All Rights Reserved.
% Licenced as described in the file LICENSE under the root directory of this GIT repository.
%

%
%
%\documentclass{article}

%\input{../peeters_macros.tex}
%\input{../peeters_macros2.tex}

%\usepackage{listings}
%\usepackage{txfonts} % for ointctr... (also appears to make "prettier" \int and \sum's)
%\usepackage[bookmarks=true]{hyperref}

%\usepackage{color,cite,graphicx}
   % use colour in the document, put your citations as [1-4]
   % rather than [1,2,3,4] (it looks nicer, and the extended LaTeX2e
   % graphics package.
%\usepackage{latexsym,amssymb,epsf} % do not remember if these are
   % needed, but their inclusion can not do any damage


\chapter{Bohm Chapter 10 problems}
\label{chap:bohmCh10}
%\author{Peeter Joot \quad peeterjoot@protonmail.com }
\date{ April 23, 2009.  bohmCh10.tex }

%\begin{document}

%\maketitle{}
%\tableofcontents
\section{Bohm Chapter 10 problems}

Problems and additional details from reading of \citep{bohm1989qt}, chapter 10.

Differing from the text, the notation \(\expectation{O}\) has been used instead \(\overbar{O}\) mostly due to not knowing how to format the a wide overbar, and getting peculiar looking results.

\subsection{P1. Uncertainty calculations}

Calculate \(\Delta x \Delta p\) for a few wave functions

\subsubsection{Gaussian wave function}

\begin{equation}\label{eqn:bohmCh10:20}
\begin{aligned}
\psi = \alpha_1 e^{-\alpha x^2/2}
\end{aligned}
\end{equation}

Normalization

\begin{equation}\label{eqn:bohmCh10:40}
\begin{aligned}
1
&= \Abs{\alpha_1}^2 \int e^{-\alpha x^2} dx \\
%&= \Abs{\alpha_1}^2 \sqrt{\pi/\alpha}
\end{aligned}
\end{equation}

Position expectation is zero, since it is odd:

\begin{equation}\label{eqn:bohmCh10:60}
\begin{aligned}
\expectation{x} \propto \int x e^{-\alpha x^2} dx = 0
\end{aligned}
\end{equation}

And the second power

\begin{equation}\label{eqn:bohmCh10:80}
\begin{aligned}
\expectation{x^2}
&= \Abs{\alpha_1}^2 \int x^2 e^{-\alpha x^2} dx \\
&= \Abs{\alpha_1}^2 \int x (e^{-\alpha x^2}/-2\alpha)' dx \\
&= \Abs{\alpha_1}^2 \int (e^{-\alpha x^2}/2\alpha) dx \\
&= \inv{2\alpha} \mathLabelBox{\Abs{\alpha_1}^2 \int e^{-\alpha x^2} dx}{\(=1\)} \\
&= \inv{2\alpha}
\end{aligned}
\end{equation}

For the first momentum expectation, we have zero again since we end up with an odd integral:

\begin{equation}\label{eqn:bohmCh10:100}
\begin{aligned}
\expectation{p}
&= -i \Hbar \Abs{\alpha_1}^2 \int e^{-\alpha x^2 /2} \frac{d}{dx} e^{-\alpha x^2 /2} dx \\
&= -i \Hbar \Abs{\alpha_1}^2 \int e^{-\alpha x^2 /2} (-\alpha x) e^{-\alpha x^2 /2} dx \\
&= 0
\end{aligned}
\end{equation}

And for the second power

\begin{equation}\label{eqn:bohmCh10:120}
\begin{aligned}
\expectation{p^2}
&= - \Hbar^2 \Abs{\alpha_1}^2 \int e^{-\alpha x^2 /2} \frac{d}{dx} ((-\alpha x) e^{-\alpha x^2 /2}) dx \\
&= \Hbar^2 \Abs{\alpha_1}^2 \alpha \int e^{-\alpha x^2 /2} \frac{d}{dx} (x e^{-\alpha x^2 /2}) dx \\
&= -\Hbar^2 \Abs{\alpha_1}^2 \alpha \int \left(\frac{d}{dx}e^{-\alpha x^2 /2}\right) (x e^{-\alpha x^2 /2}) dx \\
&= \Hbar^2 \alpha^2 \mathLabelBox{\Abs{\alpha_1}^2 \int x^2 e^{-\alpha x^2} dx}{\(\expectation{x^2} = 1/2\alpha\)} \\
&= \Hbar^2 \alpha \inv{2}
\end{aligned}
\end{equation}

Assembling results

\begin{equation}\label{eqn:bohmCh10:140}
\begin{aligned}
\Delta x \Delta p
&= \sqrt{\Hbar^2 \alpha \inv{2} \inv{2\alpha}} \\
&= \sqrt{\Hbar^2 \inv{4}} \\
&= \frac{\Hbar}{2}
\end{aligned}
\end{equation}

This is the expected result since equality with \(\Hbar/2\) occurs only with the Gaussian.

\subsubsection{Absolute valued exponential wave function}

\begin{equation}\label{eqn:bohmCh10:160}
\begin{aligned}
\psi &= \alpha_2 e^{-\alpha\Abs{x}}
\end{aligned}
\end{equation}

Normalization

\begin{equation}\label{eqn:bohmCh10:180}
\begin{aligned}
1
&= 2 \Abs{\alpha_2}^2 \int_0^\infty e^{-2 \alpha{x}} dx \\
&= 2 \Abs{\alpha_2}^2 \left. \frac{e^{-2 \alpha{x}}}{-2\alpha} \right\vert_0^\infty \\
&= 2 \Abs{\alpha_2}^2 \inv{2\alpha}
\end{aligned}
\end{equation}

\begin{equation}\label{eqn:bohmCh10:200}
\begin{aligned}
\expectation{x} = 0
\end{aligned}
\end{equation}

(odd).

\begin{equation}\label{eqn:bohmCh10:220}
\begin{aligned}
\expectation{x^2}
&= 2 \Abs{\alpha_2}^2 \int_0^\infty x^2 e^{-2\alpha x} dx \\
&= \Abs{\alpha_2}^2 \inv{2 \alpha^3}
\end{aligned}
\end{equation}

For the momentum we need derivatives

\begin{equation}\label{eqn:bohmCh10:240}
\begin{aligned}
\frac{d}{dx} e^{-\alpha \Abs{x}}
&=
\left\{
\begin{array}{l l}
\frac{d}{dx}(e^{-\alpha x} & \quad \mbox{\(x>0\)} \\
\frac{d}{dx}(e^{\alpha x} & \quad \mbox{\(x<0\)} \\
\end{array}
\right. \\
&=
\left\{
\begin{array}{l l}
-\alpha (e^{-\alpha x} & \quad \mbox{\(x>0\)} \\
\alpha (e^{\alpha x} & \quad \mbox{\(x<0\)} \\
\end{array}
\right. \\
&=
-\alpha \sgn(x) e^{-\alpha \Abs{x}}
\end{aligned}
\end{equation}

\begin{equation}\label{eqn:bohmCh10:260}
\begin{aligned}
\expectation{p}
&= i \Hbar \Abs{\alpha_2}^2 \alpha \int \sgn(x) e^{-2\alpha\Abs{x}} dx  \\
&= 0
\end{aligned}
\end{equation}

(odd)

\begin{equation}\label{eqn:bohmCh10:280}
\begin{aligned}
\expectation{p^2}
&= (-i \Hbar)^2 \Abs{\alpha_2}^2 \alpha^2 \int (\sgn(x))^2 e^{-2\alpha\Abs{x}} dx  \\
&= - 2(\Hbar)^2 \Abs{\alpha_2}^2 \alpha^2 \int_0^\infty e^{-2\alpha\Abs{x}} dx  \\
&= 2(\Hbar)^2 \Abs{\alpha_2}^2 \alpha^2 \inv{2\alpha} \\
&= (\Hbar)^2 \Abs{\alpha_2}^2 \alpha \\
\end{aligned}
\end{equation}

Putting results together we have

\begin{equation}\label{eqn:bohmCh10:300}
\begin{aligned}
\expectation{p^2} \expectation{x^2}
&= (\Hbar)^2 \alpha_2^4 \inv{2 \alpha^2} \\
&= (\Hbar)^2/2
\end{aligned}
\end{equation}

Same as in the Gaussian.

\subsubsection{Squared polynomial}

\begin{equation}\label{eqn:bohmCh10:320}
\begin{aligned}
\psi = \frac{\alpha_3}{(\alpha^2 + x^2)^2}
\end{aligned}
\end{equation}

For this one, the integrals were evaluated with Mathematica online integrator, where the contributions at \(\infty\) were scaled \(\arctan(x/\alpha)\) values.

\begin{equation}\label{eqn:bohmCh10:340}
\begin{aligned}
1
&= \Abs{\alpha_3}^2 \int \frac{dx}{(\alpha^2 + x^2)^4} \\
&= \Abs{\alpha_3}^2 \frac{15 \pi}{48 \alpha^7}
\end{aligned}
\end{equation}

\begin{equation}\label{eqn:bohmCh10:360}
\begin{aligned}
\expectation{x} &= 0
\end{aligned}
\end{equation}

\begin{equation}\label{eqn:bohmCh10:380}
\begin{aligned}
\expectation{x^2}
&= \Abs{\alpha_3}^2 \int \frac{x^2 dx}{(\alpha^2 + x^2)^4} \\
&= \Abs{\alpha_3}^2 \frac{3 \pi}{48 \alpha^5}
\end{aligned}
\end{equation}

\begin{equation}\label{eqn:bohmCh10:400}
\begin{aligned}
\expectation{p}
&= -i \Hbar \Abs{\alpha_3}^2 \int \frac{dx}{(\alpha^2 + x^2)^2} \frac{d}{dx} \frac{1}{(\alpha^2 + x^2)^2}  \\
&= -i \Hbar \Abs{\alpha_3}^2 \int \frac{dx}{(\alpha^2 + x^2)^2} \frac{-4x}{(\alpha^2 + x^2)^2} \\
&= 0
\end{aligned}
\end{equation}

\begin{equation}\label{eqn:bohmCh10:420}
\begin{aligned}
\expectation{p^2}
&= (-i \Hbar)^2 \Abs{\alpha_3}^2 \int \frac{dx}{(\alpha^2 + x^2)^2} \frac{d^2}{dx^2} \frac{1}{(\alpha^2 + x^2)^2}  \\
&= 4 \Hbar^2 \Abs{\alpha_3}^2 \int \frac{dx}{(\alpha^2 + x^2)^2} \frac{d}{dx} \frac{x}{(\alpha^2 + x^2)^3} \\
&= 4 \Hbar^2 \Abs{\alpha_3}^2 \int dx
\left( \frac{1}{(\alpha^2 + x^2)^5} \frac{x(-3)(2x)}{(\alpha^2 + x^2)^6} \right)
\\
&= 4 \Hbar^2 \Abs{\alpha_3}^2 \int dx \frac{\alpha^2 - 5x^2}{(\alpha^2 + x^2)^6}
\\
&= 4 \Hbar^2 \Abs{\alpha_3}^2 \frac{105 \pi}{960 \alpha^9}
\\
\end{aligned}
\end{equation}

Assembling

\begin{equation}\label{eqn:bohmCh10:440}
\begin{aligned}
\expectation{p^2} \expectation{x^2}
&= 4 \Hbar^2 \alpha_3^4 \frac{105 \pi}{960 \alpha^9} \frac{3 \pi}{48 \alpha^5} \\
&= \Hbar^2 \frac{7}{25}
\end{aligned}
\end{equation}

For
\begin{equation}\label{eqn:bohmCh10:460}
\begin{aligned}
\Delta{p} \Delta{x}
&= \Hbar \frac{\sqrt{7}}{5} \\
&\approx 0.52 \Hbar  \\
&> \Hbar/2
\end{aligned}
\end{equation}

\subsection{P2. Correlation coefficients}

It is noted that a classical correlation coefficient for random variables \(x\), and \(p\) has the form

\begin{equation}\label{eqn:bohmCh10:480}
\begin{aligned}
C_{n,m}
&= \expectation{x^n p^m} - \expectation{x^n}\expectation{p^m}
\end{aligned}
\end{equation}

However, for operator expectation values and average of both orderings is more reasonable

\begin{equation}\label{eqn:bohmCh10:500}
\begin{aligned}
C_{n,m}
&= \inv{2} \left(
\expectation{x^n p^m} - \expectation{x^n}\expectation{p^m}
+ \expectation{p^m x^n} - \expectation{p^m}\expectation{x^n}
\right) \\
&= \inv{2} \left( \expectation{x^n p^m} + \expectation{p^m x^n} \right) - \expectation{x^n}\expectation{p^m}
\end{aligned}
\end{equation}

With the operator substitution \(p \rightarrow -i \Hbar d/dx\) this provides equation (7) in the text.

\subsubsection{first correlations}

Calculate \(C_{1,1}\), and \(C_{2,2}\) for

\begin{equation}\label{eqn:bohmCh10:520}
\begin{aligned}
\psi = \alpha e^{-\alpha x^2/2}
\end{aligned}
\end{equation}

First the normalization and first and second order expectations.

\begin{equation}\label{eqn:bohmCh10:540}
\begin{aligned}
1
&= \alpha^2 \int e^{-\alpha x^2} dx \\
&= \alpha^2 \sqrt{\pi/\alpha} \\
\implies \\
\alpha &= \pi^{1/3}
\end{aligned}
\end{equation}

\begin{equation}\label{eqn:bohmCh10:560}
\begin{aligned}
\expectation{x} &= 0
\end{aligned}
\end{equation}

\begin{equation}\label{eqn:bohmCh10:580}
\begin{aligned}
\expectation{p} &= 0
\end{aligned}
\end{equation}

And in particular

\begin{equation}\label{eqn:bohmCh10:600}
\begin{aligned}
\expectation{p} \expectation{x} &= 0 \\
\end{aligned}
\end{equation}

\subsubsection{second correlations}

\begin{equation}\label{eqn:bohmCh10:620}
\begin{aligned}
\expectation{x^2}
&= \alpha^2 \int x^2 e^{-\alpha x^2} dx \\
&= \alpha^2 \int x x e^{-\alpha x^2} dx \\
&= \alpha^2 \int x (e^{-\alpha x^2}/-\alpha)' dx \\
&= \alpha^2 \int e^{-\alpha x^2}/\alpha dx \\
&= \alpha \int e^{-\alpha x^2} dx \\
&= \alpha \sqrt{\pi/\alpha} \\
&= \sqrt{\alpha \pi} \\
&= \pi^{2/3} \\
\end{aligned}
\end{equation}

\begin{equation}\label{eqn:bohmCh10:640}
\begin{aligned}
\expectation{p^2}
&= -\alpha^2 \Hbar^2 \int e^{-\alpha x^2/2} \frac{d^2}{dx^2} e^{-\alpha x^2/2} dx \\
&= \alpha^2 \Hbar^2 \int \frac{d}{dx} e^{-\alpha x^2/2} \frac{d}{dx} e^{-\alpha x^2/2} dx \\
&= \alpha^2 \Hbar^2 \int (-\alpha x)^2 e^{-\alpha x^2} dx \\
&= \alpha^4 \Hbar^2 \int x^2 e^{-\alpha x^2} dx \\
&= \alpha^4 \Hbar^2 \int x x e^{-\alpha x^2} dx \\
&= \alpha^4 \Hbar^2 \int x (e^{-\alpha x^2}/-\alpha)' dx \\
&= \alpha^3 \Hbar^2 \int e^{-\alpha x^2} dx \\
&= \alpha^3 \Hbar^2 \inv{2\alpha} \\
&= \alpha^2 \Hbar^2  \\
&= \pi^{2/3} \Hbar^2
\end{aligned}
\end{equation}

\begin{equation}\label{eqn:bohmCh10:660}
\begin{aligned}
\expectation{p^2} \expectation{x^2} &= \Hbar^2 \pi^{4/3}
\end{aligned}
\end{equation}

For the first terms we want

\begin{equation}\label{eqn:bohmCh10:680}
\begin{aligned}
\inv{2} \left( \expectation{x^2 p^2} + \expectation{p^2 x^2} \right)
&=
\frac{-\Hbar^2 \alpha^2}{2} \int \left(
e^{-\alpha x^2/2} x^2 \frac{d^2}{dx^2} (e^{-\alpha x^2/2} )
+ e^{-\alpha x^2/2} \frac{d^2}{dx^2} ( x^2 e^{-\alpha x^2/2} ) \right) dx \\
&=
\frac{-\Hbar^2 \alpha^2}{2} \int \left(
 \frac{d^2}{dx^2} ( e^{-\alpha x^2/2} x^2 ) e^{-\alpha x^2/2}
+ e^{-\alpha x^2/2} \frac{d^2}{dx^2} ( x^2 e^{-\alpha x^2/2} ) \right) dx \\
&=
{-\Hbar^2 \alpha^2} \int e^{-\alpha x^2/2} \frac{d^2}{dx^2} ( e^{-\alpha x^2/2} x^2 ) dx \\
&=
{\Hbar^2 \alpha^2} \int \left( \frac{d}{dx} e^{-\alpha x^2/2} \right) \left( \frac{d}{dx} ( e^{-\alpha x^2/2} x^2 ) \right) dx \\
&=
{\Hbar^2 \alpha^2} \int (-\alpha x) e^{-\alpha x^2} (2x + x^2(-\alpha x)) dx \\
&=
{\Hbar^2 \alpha^3} \int x^2 e^{-\alpha x^2} (-2 + x^2 \alpha ) dx \\
&=
{-\Hbar^2 \alpha^2} \sqrt{\pi/\alpha}  \\
&=
-\Hbar^2 \pi
\end{aligned}
\end{equation}

This leaves

\begin{equation}\label{eqn:bohmCh10:700}
\begin{aligned}
C_{2,2} &=
-\Hbar^2 \left( \pi + \pi^{2/3} \right)
\end{aligned}
\end{equation}

\subsection{P3. First correlations zero for real wave function}

Show that the first correlation coefficient is zero for any real wave function.

\begin{equation}\label{eqn:bohmCh10:720}
\begin{aligned}
C_{1,1} &= \inv{2}\left( \expectation{x p} + \expectation{p x} \right) - \expectation{x}\expectation{p}
\end{aligned}
\end{equation}

Calculate instead the equivalent problem

\begin{equation}\label{eqn:bohmCh10:740}
\begin{aligned}
2 C_{1,1}/(-i\Hbar) &= \left( \expectation{x \frac{d}{dx}} + \expectation{\frac{d}{dx} x} \right) - 2 \expectation{x}\expectation{\frac{d}{dx}}
\end{aligned}
\end{equation}

For the anti-commutator part we have

\begin{equation}\label{eqn:bohmCh10:760}
\begin{aligned}
\expectation{x \frac{d}{dx}} + \expectation{\frac{d}{dx} x}
&=
\int \psi x \psi' + \psi (x \psi)' \\
&=
\int \psi x \psi' - \psi' x \psi \\
&= 0
\end{aligned}
\end{equation}

and for the remainder if one is zero then the sum is.  In particular

\begin{equation}\label{eqn:bohmCh10:780}
\begin{aligned}
\expectation{\frac{d}{dx}}
&= \int_{-\infty}^\infty dx \psi \psi' \\
&= \inv{2} \int_{-\infty}^\infty dx (\psi^2)' \\
&= \inv{2} \left. \psi^2 \right\vert_{-\infty}^\infty
\end{aligned}
\end{equation}

Provided the wave function vanishes in the square at \(\pm \infty\), then we are done.

\subsection{P4}
\subsection{P5}
\subsection{P6}

The phase space text on this page is not clear to me.  Revisit after study
of phase space, Poisson brackets, and Liouville's theorem in a classical
context.

\subsection{P7. wave function for the position and momentum operators for the equality uncertainty case}

Note that in the definitions of \(\alpha\) and \(\beta\) right before equation (25) in the text, the symbols are reversed.  For consistency with condition (1)
this should be

\begin{equation}\label{eqn:bohmCh10:800}
\begin{aligned}
\beta &= (x - \overbar{x}) \\
\alpha &= (p - \overbar{p})
\end{aligned}
\end{equation}

Where condition (1) for equality in the Schwartz inequality for this
generalized uncertainty principle is

\begin{equation}\label{eqn:bohmCh10:820}
\begin{aligned}
\alpha \psi &= C \beta \psi
\end{aligned}
\end{equation}

Putting the two together for this problem one has

\begin{equation}\label{eqn:bohmCh10:840}
\begin{aligned}
(p - \overbar{p}) \psi &= C (x - \overbar{x}) \psi \\
\implies \\
p \psi &= \overbar{p} \psi + C (x - \overbar{x}) \psi \\
\end{aligned}
\end{equation}

or
\begin{equation}\label{eqn:bohmCh10:860}
\begin{aligned}
\frac{\Hbar}{i} \frac{d\psi}{dx} &= \overbar{p} \psi + C (x - \overbar{x}) \psi \\
\end{aligned}
\end{equation}

Integrating, as was done in the \(\overbar{x} = \overbar{p} = 0\) case in the text,
one has

\begin{equation}\label{eqn:bohmCh10:880}
\begin{aligned}
\ln \psi &= \frac{i}{\Hbar}( \overbar{p} x + C (x - \overbar{x})^2/2 ) + \ln D \\
\end{aligned}
\end{equation}

or
\begin{equation}\label{eqn:bohmCh10:900}
\begin{aligned}
\psi &= D e^{i\overbar{p} x/\Hbar} e^{ i C (x - \overbar{x})^2/2\Hbar } \\
\end{aligned}
\end{equation}

Note that this shows there is a typo in equation (26) in the text too (\(i\overbar{p} x\) needs the \(1/\Hbar\) factor).  References to \(C\) in the text preceding this should be \(C/\Hbar\) in a few cases too.

It was shown above that real wave functions have the vanishing \(C_{1,1}\) coefficient required for uncorrelated operators, and if that is the case \(i C/\Hbar\), must be a real negative constant.  Denoting that as \(-a\) as in the text one has

\begin{equation}\label{eqn:bohmCh10:920}
\begin{aligned}
\psi \propto e^{i\overbar{p} x/\Hbar} e^{ -a(x - \overbar{x})^2/2} \\
\end{aligned}
\end{equation}

\subsection{P8. Calculate uncertainty for the initially Gaussian wave function}

Wave function for the problem is

\begin{equation}\label{eqn:bohmCh10:940}
\begin{aligned}
\psi &= \alpha \exp\left( -(A - iB) \frac{x^2}{2} \right) \\
A &= \frac{(\Delta k)^2}{ 1 + \frac{\Hbar^2 t^2}{m^2}(\Delta k)^4 } \\
B &= (\Delta k)^4 \frac{\Hbar t}{m} \frac{1}{ 1 + \frac{\Hbar^2 t^2}{m^2}(\Delta k)^4 } \\
\end{aligned}
\end{equation}

The normalization is

\begin{equation}\label{eqn:bohmCh10:960}
\begin{aligned}
1
&= \Abs{\alpha}^2 \int \exp\left( -A {x^2} \right) \\
&= \Abs{\alpha}^2 \sqrt{\frac{\pi}{A}}
\end{aligned}
\end{equation}

First moment
\begin{equation}\label{eqn:bohmCh10:980}
\begin{aligned}
\expectation{x} = 0
\end{aligned}
\end{equation}

Second moment
\begin{equation}\label{eqn:bohmCh10:1000}
\begin{aligned}
\expectation{x^2}
&=
\Abs{\alpha}^2 \int x^2 \exp\left( -A x^2 \right) \\
&=
\Abs{\alpha}^2 \int x (\exp\left( -A x^2 \right)/(-2A))' \\
&=
\Abs{\alpha}^2 \int \exp\left( -A x^2 \right)/2A \\
&=
\Abs{\alpha}^2 \inv{2A} \sqrt{ \frac{\pi}{A}} \\
&=
\inv{2A} \\
&=
\inv{2} \frac{ 1 + \frac{\Hbar^2 t^2}{m^2}(\Delta k)^4 }{(\Delta k)^2}
 \\
&=
\inv{2}\left(\inv{(\Delta k)^2} + \frac{\Hbar^2 t^2}{m^2}(\Delta k)^2 \right) \\
\end{aligned}
\end{equation}

For the momentum expectation

\begin{equation}\label{eqn:bohmCh10:1020}
\begin{aligned}
\expectation{p}/(-i\Hbar\Abs{\alpha}^2)
&= \int dx
\exp\left( -(A + iB) \frac{x^2}{2} \right)
\frac{d}{dx}
\exp\left( -(A - iB) \frac{x^2}{2} \right)  \\
&= \int dx
\exp\left( -(A + iB) \frac{x^2}{2} \right)
(-(A -iB)x)
\exp\left( -(A - iB) \frac{x^2}{2} \right)  \\
&=
-(A -iB)
\int x \exp\left( -A x^2 \right) dx
\\
&= 0
\end{aligned}
\end{equation}

Second moment
\begin{equation}\label{eqn:bohmCh10:1040}
\begin{aligned}
\expectation{p^2}/((-i\Hbar)^2\Abs{\alpha}^2)
&= \int dx
\exp\left( -(A + iB) \frac{x^2}{2} \right)
\frac{d}{dx}
(-(A -iB)x) \exp\left( -(A - iB) \frac{x^2}{2} \right)  \\
&= -\int dx
(-(A -iB)x) \exp\left( -(A - iB) \frac{x^2}{2} \right)
\frac{d}{dx}
\exp\left( -(A + iB) \frac{x^2}{2} \right)
\\
&= -\int dx
(-(A -iB)x) \exp\left( -(A - iB) \frac{x^2}{2} \right)
(-(A +iB)x) \exp\left( -(A - iB) \frac{x^2}{2} \right)  \\
&= -\int x^2 dx
(A^2 + B^2) \exp\left( -(A - iB) \frac{x^2}{2} \right)
\exp\left( -(A - iB) \frac{x^2}{2} \right)  \\
&= -\int x^2 dx (A^2 + B^2) \exp\left( -A x^2 \right) \\
&= - (A^2 + B^2) \int x (\exp\left( -A x^2 \right)/(-2A))' \\
&= -\frac{A^2 + B^2}{2A} \int \exp\left( -A x^2 \right) \\
\end{aligned}
\end{equation}

So we have
\begin{equation}\label{eqn:bohmCh10:1060}
\begin{aligned}
\expectation{p^2} &= \Hbar^2 \frac{A^2 + B^2}{2A}
\end{aligned}
\end{equation}

But
\begin{equation}\label{eqn:bohmCh10:1080}
\begin{aligned}
A^2 + B^2
&=
\inv{(1 + \frac{\Hbar^2 t^2}{m^2}(\Delta k)^4)^2} ((\Delta k)^4 + (\Delta k)^8 \left(\frac{\Hbar t}{m}\right)^2 ) \\
&=
\inv{(1 + \frac{\Hbar^2 t^2}{m^2}(\Delta k)^4)^2} (\Delta k)^4(1 + (\Delta k)^4 \left(\frac{\Hbar t}{m}\right)^2 ) \\
&=
\frac{(\Delta k)^4}{1 + \frac{\Hbar^2 t^2}{m^2}(\Delta k)^4} \\
&= (\Delta k)^2 A
%A &= \frac{(\Delta k)^2}{ 1 + \frac{\Hbar^2 t^2}{m^2}(\Delta k)^4 } \\
\end{aligned}
\end{equation}

So we have the constant second moment as desired

\begin{equation}\label{eqn:bohmCh10:1100}
\begin{aligned}
\expectation{p^2} &= \Hbar^2 (\Delta k)^2/2
\end{aligned}
\end{equation}

\begin{equation}\label{eqn:bohmCh10:1120}
\begin{aligned}
(\Delta x \Delta p)^2
&= \Hbar^2 \frac{(\Delta k)^2}{4} \left( \inv{(\Delta k)^2} + \frac{\Hbar^2 t^2}{m^2}(\Delta k)^2 \right) \\
&= \left(\frac{\Hbar}{2}\right)^2 \left( 1 + \frac{\Hbar^2 t^2}{m^2}(\Delta k)^4 \right) \\
\end{aligned}
\end{equation}

This matches all the expectations.  At \(t=0\) we have equality for the minimum uncertainty, and it grows as
time increases.

\subsection{P9. (first P9 of the chapter.)}

if \(\psi_A(x_1)\) and \(\psi_B(x_2)\) are independent, then the probability integral over all space is

\begin{equation}\label{eqn:bohmCh10:1140}
\begin{aligned}
\int P(x_1, x_2) dx_1 dx_2
&=
\iint \psi_A(x_1) \psi_B(x_2) dx_1 dx_2 \\
&=
\int \psi_A(x_1) dx_1 \int \psi_B(x_2) dx_2 \\
\end{aligned}
\end{equation}

So, if they are also separately normalized then so is this one.

\subsection{P9. (second P9 of the chapter.)}

Prove, ``It may be shown that if \(\psi\) is not an eigenfunction of the operator \(O\) must show some fluctuation.''

By fluctuation, I assume he means deviation of the moments,
as in the variational differences

\begin{equation}\label{eqn:bohmCh10:1160}
\begin{aligned}
\expectation{O^n} - (\expectation{O})^n
\end{aligned}
\end{equation}

How to prove this?  Suppose that the wave function is almost an eigenfunction
differing by a bit

\begin{equation}\label{eqn:bohmCh10:1180}
\begin{aligned}
O \psi = \lambda \psi + \epsilon
\end{aligned}
\end{equation}

Then we have for the moments

\begin{equation}\label{eqn:bohmCh10:1200}
\begin{aligned}
\expectation{O}
&= \int \psi^\conj O \psi \\
&= \int \psi^\conj ( \lambda \psi + \epsilon ) \\
&= \lambda + \int \psi^\conj \epsilon \\
\end{aligned}
\end{equation}

\begin{equation}\label{eqn:bohmCh10:1220}
\begin{aligned}
\expectation{O^2}
&= \int \psi^\conj O^2 \psi \\
&= \int \psi^\conj O \left( \lambda \psi + \epsilon \right) \\
&= \lambda \expectation{O} + \int \psi^\conj O \epsilon \\
&= \lambda \expectation{O} + \int \epsilon^\conj O \psi \\
&= \lambda \expectation{O} + \int \epsilon^\conj \left(\lambda \psi + \epsilon\right) \\
&= \left(\expectation{O} - \int \psi^\conj \epsilon \right)\expectation{O} +
\lambda \int \epsilon^\conj \psi
+ \int \epsilon^\conj \epsilon
\\
&= \expectation{O}^2 - \int \psi^\conj \epsilon \expectation{O} + \left( \int \psi^\conj \epsilon - \expectation{O}\right) \int \epsilon^\conj \psi + \int \epsilon^\conj \epsilon
\\
&= \expectation{O}^2
- \expectation{O} \left( \int \epsilon^\conj \psi + \int \psi^\conj \epsilon \right)
+ \Abs{\int \psi^\conj \epsilon}^2
+ \int \epsilon^\conj \epsilon
\\
\end{aligned}
\end{equation}

So we have

\begin{equation}\label{eqn:bohmCh10:1240}
\begin{aligned}
\expectation{O^2} - \expectation{O}^2
&=
\Abs{\int \psi^\conj \epsilon}^2 + \int \epsilon^\conj \epsilon - 2 \expectation{O} \Re \int \epsilon^\conj \psi \\
\end{aligned}
\end{equation}

There is no good reason to assume that this RHS should be zero in general, so at least for the second order moment this shows that we have the fluctuation when the wave
function is not an eigenfunction.

\subsection{P10}

\begin{equation}\label{eqn:bohmCh10:1260}
\begin{aligned}
\psi = \frac{A}{(x-x_0)^2 + (\Delta x)^2}
\end{aligned}
\end{equation}

Normalizing, picking the upper plane contour around \(i \Delta x\), we have

\begin{equation}\label{eqn:bohmCh10:1280}
\begin{aligned}
1 &= \int \psi^\conj \psi \\
&= A^2 \int_{-\infty}^\infty \frac{dx}{(x-x_0)^2 + (\Delta x)^2} \\
&= A^2 \int_{-\infty}^\infty \frac{dx}{x^2 + (\Delta x)^2} \\
&= A^2 \int_{-\infty}^\infty \frac{dx}{(x + i\Delta x)(x - i \Delta x)} \\
&= A^2 \frac{2 \pi i }{2 i\Delta x} \\
&= A^2 \frac{\pi}{\Delta x} \\
\end{aligned}
\end{equation}

So we have the desired normalization

\begin{equation}\label{eqn:bohmCh10:1300}
\begin{aligned}
A &= \sqrt{\frac{\Delta x}{\pi}} \\
\end{aligned}
\end{equation}

So how do you show that the now normalized wave function is an eigenfunction of \(x\)

\begin{equation}\label{eqn:bohmCh10:1320}
\begin{aligned}
\psi &= \sqrt{\frac{\Delta x}{\pi}} \frac{1}{(x-x_0)^2 + (\Delta x)^2} \\
\end{aligned}
\end{equation}

Would a calculation of the expectation value for the position operator be sufficient?  That is

\begin{equation}\label{eqn:bohmCh10:1340}
\begin{aligned}
\expectation{x}
&= A^2 \IIinf \frac{x dx }{(x-x_0)^2 + (\Delta x)^2} \\
&= A^2 \IIinf \frac{(u + x_0) du }{u^2 + (\Delta x)^2} \\
&= x_0 + A^2 \IIinf \frac{u du }{u^2 + (\Delta x)^2} \\
\end{aligned}
\end{equation}

Now \(u/(u^2 + \alpha^2)\) has antiderivative \(\ln(x^2 + \alpha^2)/2\), so the PV value of this integral for \(\Delta x \ne 0\)
is

\begin{equation}\label{eqn:bohmCh10:1360}
\begin{aligned}
PV A^2 \IIinf \frac{u du }{u^2 + (\Delta x)^2}
PV \frac{\Delta x}{\pi} \IIinf \frac{u du }{u^2 + (\Delta x)^2}
&= \inv{2} \lim_{R\rightarrow \infty} \frac{\Delta x}{\pi} \ln\left( \frac{R^2 + (\Delta x)^2}{(-R)^2 + (\Delta x)^2} \right) \\
&= 0
\end{aligned}
\end{equation}

So, for all \(\Delta x \ne 0\) (where that \(\PV\) integral goes messy), we have

\begin{equation}\label{eqn:bohmCh10:1380}
\begin{aligned}
\expectation{x} &= x_0
\end{aligned}
\end{equation}

Is this sufficient to show that this wave function approaches an eigenfunction as \(\Delta x \rightarrow 0\).
I suppose that one could loosely argue that the \(A^2\) term kills off the log term at the limit of \(\Delta x = 0\) (if you are careful in the argument about how exactly \(\Delta x \rightarrow 0\) with \(R \rightarrow \infty\)).

\subsection{P11. Delta function example as limit}

\subsubsection{The problem}

Show that for

\begin{equation}\label{eqn:bohmCh10:1400}
\begin{aligned}
\delta_\epsilon(x - x_0) &= \frac{A}{(x-x_0)^2 + \epsilon^2}
\end{aligned}
\end{equation}

The limit has a delta function action

\begin{equation}\label{eqn:bohmCh10:1420}
\begin{aligned}
\delta(x - x_0) &= \lim_{\epsilon \rightarrow 0} \delta_\epsilon(x - x_0)
\end{aligned}
\end{equation}

Calculate \(A\), and explain why it is different than \(A\) in problem 10.

\subsubsection{Normalization}

First for the constant

\begin{equation}\label{eqn:bohmCh10:1440}
\begin{aligned}
1 &= \IIinf dx \delta_\epsilon(x - x_0) \\
&= A \IIinf \frac{dx}{x^2 + \epsilon^2} \\
&= A \IIinf \frac{dx}{(x + i\epsilon)(x - i \epsilon)} \\
&= A \frac{2 \pi i}{ 2 i \epsilon } \\
&= A \frac{\pi}{\epsilon } \\
\end{aligned}
\end{equation}

So we have

\begin{equation}\label{eqn:bohmCh10:1460}
\begin{aligned}
A = \frac{\epsilon }{\pi} \\
\end{aligned}
\end{equation}

This constant is necessarily different from the eigenfunction normalization, since that involved normalization in the square.  The resulting
nascent delta function is

\begin{equation}\label{eqn:bohmCh10:1480}
\begin{aligned}
\delta_\epsilon(x - x_0) &= \frac{\epsilon }{\pi} \frac{1}{(x-x_0)^2 + \epsilon^2}
\end{aligned}
\end{equation}

\subsubsection{Delta function action}

How do we show that \(\delta_\epsilon\) behaves as a delta function in the limit?  The delta function is really defined by how it
acts on a test function in an integration operation, so let us calculate

\begin{equation}\label{eqn:bohmCh10:1500}
\begin{aligned}
\IIinf \delta_\epsilon(x - x_0) f(x) dx
&=
\frac{\epsilon }{\pi} \IIinf \frac{f(x) dx}{(x-x_0)^2 + \epsilon^2} \\
&=
\frac{\epsilon }{\pi} \IIinf \frac{f(u + x_0) du}{u^2 + \epsilon^2} \\
&=
\frac{\epsilon }{\pi} \IIinf \frac{f(u + x_0) du}{(u + i \epsilon)(u - i \epsilon)} \\
\end{aligned}
\end{equation}

An upper half plane contour, assuming that \(f(x_0 + i\epsilon)\) is a regular point (and that f(z) has no poles in the upper half plane) gives us

\begin{equation}\label{eqn:bohmCh10:1520}
\begin{aligned}
\IIinf \delta_\epsilon(x - x_0) f(x) dx
&=
\frac{\epsilon }{\pi} 2 \pi i \frac{f(i \epsilon + x_0) }{2 i \epsilon} \\
&=
f(i \epsilon + x_0)
\end{aligned}
\end{equation}

So in the limit if \(f(x)\) is regular all the way down the \(x_0 + i\epsilon\) trajectory, we have

\begin{equation}\label{eqn:bohmCh10:1540}
\begin{aligned}
\lim_{\epsilon \rightarrow 0} \IIinf \delta_\epsilon(x - x_0) f(x) dx &= f(x_0)
\end{aligned}
\end{equation}

which is precisely the operational definition of the delta function.

\subsection{P12. Delta function differentiation}

Prove by successive differentiation that

\begin{equation}\label{eqn:bohmCh10:1560}
\begin{aligned}
\frac{d^n f(x)}{dx^n}
&= \int_{-\infty}^{\infty} \delta(x - x_0) \frac{d^n f(x_0)}{dx_0} dx_0 \\
\end{aligned}
\end{equation}

Doing the integration by parts, and change of variables for the delta function derivatives:

\begin{equation}\label{eqn:bohmCh10:1580}
\begin{aligned}
\frac{d^n f(x)}{dx^n}
&= \IIinf \frac{d^n \delta(x - x_0)}{dx^n} f(x_0) dx_0 \\
&= \IIinf \frac{d^n \delta(x-x_0)}{{dx_0}^n} (-1)^n f(x_0) dx_0 \\
%&= -(-1)^n \int_\infty^{-\infty} \frac{d^n \delta(x - x_0)}{dx_0^n} f(x_0) dx_0 \\
%&= (-1)^n \int_{-\infty}^{\infty} \frac{d^n \delta(x - x_0)}{dx_0^n} f(x_0) dx_0 \\
&= -(-1)^n \int_{-\infty}^{\infty} \frac{d^{n-1} \delta(x - x_0)}{dx_0^{n-1}} \frac{df(x_0)}{dx_0} dx_0 \\
&= (-1)^2 (-1)^n \int_{-\infty}^{\infty} \frac{d^{n-2} \delta(x - x_0)}{dx_0^{n-2}} \frac{d^2 f(x_0)}{dx_0^2} dx_0 \\
&= \cdots \\
&= (-1)^{n-1} (-1)^n \int_{-\infty}^{\infty} \frac{d^{n-(n-1)} \delta(x - x_0)}{dx_0^{n-(n-1)}} \frac{d^{n-1} f(x_0)}{dx_0^{n-1}} dx_0 \\
&= (-1)^{n-1} (-1)^n \int_{-\infty}^{\infty} \frac{d \delta(x - x_0)}{dx_0} \frac{d^{n-1} f(x_0)}{dx_0^{n-1}} dx_0 \\
&= (-1)^n (-1)^n \int_{-\infty}^{\infty} \delta(x - x_0) \frac{d^n f(x_0)}{dx_0} dx_0 \\
&= \int_{-\infty}^{\infty} \delta(x - x_0) \frac{d^n f(x_0)}{dx_0} dx_0 \\
\end{aligned}
\end{equation}

\subsection{P13. Eigenfunctions for particle in one dimensional box}

Box of side \(L\).  Eigenfunction of \(p\) are given as exponentials in the problem.  Stepping back slightly to see where these come from
consider the operator eigenvalue statement itself.  This will in fact indirectly solve the problem

\begin{equation}\label{eqn:bohmCh10:1600}
\begin{aligned}
\lambda \psi
&= p \psi_\lambda \\
&= \left(-i \Hbar \PD{x}{} \right) \psi_\lambda \\
\end{aligned}
\end{equation}

This can be integrated

\begin{equation}\label{eqn:bohmCh10:1620}
\begin{aligned}
(\ln \psi_\lambda)' &= i \lambda/\Hbar \\
\implies \\
\ln \psi_\lambda &= \frac{i \lambda x }{\Hbar} + \ln A \\
\implies \\
\psi_\lambda &= A \exp\left( \frac{i \lambda x }{\Hbar} \right) \\
\end{aligned}
\end{equation}

Considering two such eigenfunctions (normalization omitted) with an orthogonality requirement in the \([0,L]\) interval we have for \(\lambda \ne \mu\)

\begin{equation}\label{eqn:bohmCh10:1640}
\begin{aligned}
\int_0^L (\psi_\mu)^\conj \psi_\lambda dx
&=
\int_0^L
\exp\left( \frac{-i \mu x }{\Hbar} \right) \exp\left( \frac{i \lambda x }{\Hbar} \right) \\
&=
\int_0^L \exp\left( \frac{i (\lambda -\mu) x }{\Hbar} \right) \\
&=
\left. \inv{ i (\lambda -\mu)/\Hbar} \exp\left( \frac{i (\lambda -\mu) x }{\Hbar} \right) \right\vert_{0}^L
 \\
&=
\inv{ i (\lambda -\mu)/\Hbar} \left( \exp\left( \frac{i (\lambda -\mu) L }{\Hbar} \right) - 1 \right)
 \\
\end{aligned}
\end{equation}

So, for orthogonality we need \(\lambda L/\Hbar = 2\pi n_\lambda\), or more simply

\begin{equation}\label{eqn:bohmCh10:1660}
\begin{aligned}
\psi_n = \inv{\sqrt{L}} \exp\left( \frac{ 2 \pi i n x }{L } \right) \\
\end{aligned}
\end{equation}

It is interesting to see that the one dimensional particle in a box can be reduced to a first order differential equation or
eigenvalue problem, instead of looking for solutions to the energy operator equation \((p^2/2m) \psi = E \psi\).

\subsection{P14. Fourier series representation of delta function}

TODO.

\subsection{P15}

This one has a prereq on the ch3 problems, which I did not do.  Revisit.

%\bibliographystyle{plainnat}
%\bibliography{myrefs}

%\end{document}
