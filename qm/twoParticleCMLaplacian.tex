%
% Copyright � 2012 Peeter Joot.  All Rights Reserved.
% Licenced as described in the file LICENSE under the root directory of this GIT repository.
%

%
%
%\input{../peeter_prologue.tex}

\chapter{Two particle center of mass Laplacian change of variables}
\label{chap:twoParticleCMLaplacian}
%\useCCL
\blogpage{http://sites.google.com/site/peeterjoot/math2009/twoParticleCMLaplacian.pdf}
\date{Nov 30, 2009}
\revisionInfo{twoParticleCMLaplacian.tex }

\beginArtWithToc
%\beginArtNoToc

Exercise 15.2 in \citep{bohm1989qt} is to do a center of mass change of variables for the two particle Hamiltonian

\begin{equation}\label{eqn:bohm15:pr2:1}
\begin{aligned}
H =
- \frac{\Hbar^2}{2 m_1} {\spacegrad_1}^2
- \frac{\Hbar^2}{2 m_2} {\spacegrad_2}^2
+ V(\Br_1 -\Br_2).
\end{aligned}
\end{equation}

Before trying this, I was surprised that this would result in a diagonal form for the transformed Hamiltonian, so it is well worth doing the problem to see why this is the case.  He uses

\begin{equation}\label{eqn:bohm15:pr2:2}
\begin{aligned}
\BXI &= \Br_1 - \Br_2 \\
\BEta &= \inv{M}( m_1 \Br_1 + m_2 \Br_2 ).
\end{aligned}
\end{equation}

Lets use coordinates \({x_k}^{(1)}\) for \(\Br_1\), and \({x_k}^{(2)}\) for \(\Br_2\).  Expanding the first order partial operator for \(\PDi{{x_1}^{(1)}}{}\) by chain rule in terms of \(\BEta\), and \(\BXI\) coordinates we have

\begin{equation}\label{eqn:twoParticleCMLaplacian:25}
\begin{aligned}
\PD{x_1^{(1)}}{}
&=
\PD{x_1^{(1)}}{\eta_k} \PD{\eta_k}{}
+\PD{x_1^{(1)}}{\xi_k} \PD{\xi_k}{} \\
&=
\frac{m_1}{M} \PD{\eta_1}{}
+ \PD{\xi_1}{}.
\end{aligned}
\end{equation}

We also have

\begin{equation}\label{eqn:twoParticleCMLaplacian:45}
\begin{aligned}
\PD{x_1^{(2)}}{}
&=
\PD{x_1^{(2)}}{\eta_k} \PD{\eta_k}{}
+\PD{x_1^{(2)}}{\xi_k} \PD{\xi_k}{} \\
&=
\frac{m_2}{M} \PD{\eta_1}{}
- \PD{\xi_1}{}.
\end{aligned}
\end{equation}

The second partials for these \(x\) coordinates are not a diagonal quadratic second partial operator, but are instead

\begin{equation}\label{eqn:bohm15:pr2:3}
\begin{aligned}
\PD{x_1^{(1)}}{} \PD{x_1^{(1)}}{}
&=
\frac{(m_1)^2}{M^2} \frac{\partial^2}{\partial \eta_1 \partial \eta_1}{}
+\frac{\partial^2}{\partial \xi_1 \partial \xi_1}{}
+2 \frac{m_1}{M} \frac{\partial^2}{\partial \xi_1 \partial \eta_1}{} \\
\PD{x_1^{(2)}}{} \PD{x_1^{(2)}}{}
&=
\frac{(m_2)^2}{M^2} \frac{\partial^2}{\partial \eta_1 \partial \eta_1}{}
+\frac{\partial^2}{\partial \xi_1 \partial \xi_1}{}
-2 \frac{m_2}{M} \frac{\partial^2}{\partial \xi_1 \partial \eta_1}{}.
\end{aligned}
\end{equation}

The desired result follows directly, since the mixed partial terms conveniently cancel when we sum \((1/m_1) \PDi{x_1^{(1)}}{} \PDi{x_1^{(1)}}{} +(1/m_2) \PDi{x_1^{(2)}}{} \PDi{x_1^{(2)}}{}\).  This leaves us with

\begin{equation}\label{eqn:bohm15:pr2:4}
\begin{aligned}
H =
\frac{-\Hbar^2}{2} \sum_{k=1}^3 \left(
\inv{M} \frac{\partial^2}{\partial \eta_k \partial \eta_k}{}
+ \left( \inv{m_1} + \inv{m_2} \right) \frac{\partial^2}{\partial \xi_k \partial \xi_k}{}
\right)
+ V(\BXI),
\end{aligned}
\end{equation}

With the shorthand of the text

\begin{equation}\label{eqn:bohm15:pr2:5}
\begin{aligned}
\spacegrad_{\BEta} &= \sum_k \frac{\partial^2}{\partial \eta_k \partial \eta_k}{} \\
\spacegrad_{\BXI} &= \sum_k \frac{\partial^2}{\partial \xi_k \partial \xi_k}{},
\end{aligned}
\end{equation}

this is the result to be proven.

\EndArticle
%\EndNoBibArticle
