%
% Copyright � 2012 Peeter Joot.  All Rights Reserved.
% Licenced as described in the file LICENSE under the root directory of this GIT repository.
%

%
%
%\documentclass{article}

%\input{../peeters_macros.tex}
%\input{../peeters_macros2.tex}

%\usepackage[bookmarks=true]{hyperref}

%\usepackage{color,cite,graphicx}
   % use colour in the document, put your citations as [1-4]
   % rather than [1,2,3,4] (it looks nicer, and the extended LaTeX2e
   % graphics package.
%\usepackage{latexsym,amssymb,epsf} % do not remember if these are
   % needed, but their inclusion can not do any damage


\chapter{fletcher64}
\label{chap:fletcher}
%\author{Peeter Joot \quad peeterjoot@protonmail.com }
\date{ March dd, 2009.  fletcher.tex }

%\begin{document}

%\maketitle{}
%\tableofcontents
\section{Fast modulus by one less than power of 2?}

Statement in some code is

``\((x \& 0xFFFFFFFF) + (x \rightshift 32)\) is a faster way to do \(x \Mod 0xFFFFFFFF\), except the max result may be
\(0x1FFFFFFFE\), so need another round of this.''

How can this be justified?

\subsection{Translating the bit ops to math ops}

Suppose we have

\begin{equation}\label{eqn:fletcher:modPower2}
\begin{aligned}
x = y \times 2^B + r \quad \mbox{where \(r \le 2^{B-1}\)}
\end{aligned}
\end{equation}

then we have

\begin{equation}\label{eqn:fletcher:22}
\begin{aligned}
y &= x \Div 2^{B} = x \rightshift B \\
r &= x \Mod 2^B = x \& 2^{B-1}
\end{aligned}
\end{equation}

So we have
\begin{equation}\label{eqn:fletcher:42}
\begin{aligned}
x
&= (x \& 2^{B-1}) \times 2^B + x \rightshift B \\
&= (x \Mod 2^B) \times 2^B + x \Div 2^B \\
\end{aligned}
\end{equation}

\subsection{The shift expression}

From \eqnref{eqn:fletcher:modPower2} we have

\begin{equation}\label{eqn:fletcher:62}
\begin{aligned}
x
&= y \times 2^B + r \\
&= y \times (2^B -1) + y + r \\
\end{aligned}
\end{equation}

Therefore we have

\begin{equation}\label{eqn:fletcher:82}
\begin{aligned}
x \Mod (2^B -1) &= (y + r) \Mod (2^B -1) \\
\end{aligned}
\end{equation}

The net effect is that the original desired modulus calculation has been reduced to one of a lesser numerical order.  In particular, for the
code in question we have \(B=32\), and an unsigned 64-bit value \(x\).  This means the value \(y \le 0xFFFFFFFF\), whereas \(r \le 0xFFFFFFFF\), so the new value
\(y + r \le 0x1FFFFFFFE\) with up to 31 bits knocked off.  A second round of reduction can give a value no more than \(0xFFFFFFFE + 1 = 0xFFFFFFFF\).

The conclusion is that this ``fast modulus'' is not exactly a modulus replacement.  Instead it is either the true modulus (ie: \(x \Mod 0xFFFFFFFF < 0xFFFFFFFF\)), but may for some \(x\) actually be equal to \(0xFFFFFFFF\).  This almost modulus operation was however sufficient in the code in question, since the idea was only to reduce the magnitude to a 32-bit quantity.

\section{Fletcher64}

A 32-bit word extension of the \href{http://en.wikipedia.org/wiki/Fletcher%27s_checksum}{wikipedia fletcher algorithm} is

\begin{equation}\label{eqn:fletcher:102}
\begin{aligned}
&\text{while} ( i < n )  \\
&\quad   S_1 += A_i \\
&\quad   S_2 += S_1 \\
\\
&C = \Mod_{32}(S_2) | \Mod_{32}(S_1)
\end{aligned}
\end{equation}

Where \(\Mod_{32}(x)\) is the fast modulus operation described above.

The summation equivalent to the loop above is
\begin{equation}\label{eqn:fletcher:122}
\begin{aligned}
S_1 &= \sum_{i=0}^{n-1} A_i \\
S_2 &= \sum_{i=0}^{n-1} (n-i)A_i \\
\end{aligned}
\end{equation}

Note that the above loop assumes that the data is constrained enough that no arithmetic overflows occur within the loop iteration.
Strictly speaking a more accurate transcription of the wiki fletcher32 algorithm using 64-bit accumulators is

\begin{equation}\label{eqn:fletcher:142}
\begin{aligned}
&\text{while} ( i < n )  \\
&\quad   S_1 += A_i \\
&\quad   S_2 += S_1 \\
&\quad   S_1 = \Mod_{32}(S_1) \\
&\quad   S_2 = \Mod_{32}(S_2) \\
\\
&C = S_2 | S_1
\end{aligned}
\end{equation}

Optimizations (like the use of the magic number 360 in the wikipedia article) are possible to avoid the modulus reduction in each iteration of the loop.

In particular, the modulus accumulation of residues is equivalent to the modulus of the sums.  Say

\begin{equation}\label{eqn:fletcher:162}
\begin{aligned}
a_1 &= (a_1 \Div m ) m + r_1 \\
a_2 &= (a_2 \Div m ) m + r_2
\end{aligned}
\end{equation}

So we have
\begin{equation}\label{eqn:fletcher:182}
\begin{aligned}
a_1 + a_2 &= (a_1 \Div m + a_2 \Div m ) m + r_1 + r_2 \\
\implies \\
(a_1 + a_2) \Mod m &= (r_1 + r_2) \Mod m \\
\end{aligned}
\end{equation}

Or more generally

\begin{equation}\label{eqn:fletcher:202}
\begin{aligned}
\left(\sum_i a_i \right) \Mod m &= \sum_i (a_i \Mod m) \Mod m \\
\end{aligned}
\end{equation}

\subsection{Max loops before truncation}

With 32K page sizes and 4 byte words we have a max number of iterations for the loop of

\begin{equation}\label{eqn:fletcher:222}
\begin{aligned}
\frac{32 \times 1024 }{4} = 8192
\end{aligned}
\end{equation}

Assuming the biggest size value of \(M = A_i = 0xFFFFFFFF\) when do our summation variables wrap?

\begin{equation}\label{eqn:fletcher:242}
\begin{aligned}
S_1 &= \sum_{i=0}^{n-1} A_i \le n M \\
S_2 &= \sum_{i=0}^{n-1} (n-i)A_i <= M \sum_{i=0}^{n-1} (n-i) = M n(n-1)/2 < M n^2/2
\end{aligned}
\end{equation}

The overflow point is dominated by the \(S_2\) sum, so if our max accumulator value is \(M_a\) we want

\begin{equation}\label{eqn:fletcher:262}
\begin{aligned}
M_a \ge M n^2/2 \\
\implies \\
\sqrt{\frac{ 2 M_a }{ M}} \ge n \\
\end{aligned}
\end{equation}

For 64-bit max accumulator and 32-bit words we have
\begin{equation}\label{eqn:fletcher:282}
\begin{aligned}
\sqrt{2 \frac{2^{64} - 1}{2^{32} - 1}} = 92 681.9
\end{aligned}
\end{equation}

which is far bigger than \(8192\).

% google calc:
%log( (2^64-1)/8192 + 1)/log( 2 )

The biggest number of bits in the word that we can use without having to worry about carries are as follows

\begin{tabular}{|l|l|l|}
\hline
Page Size & N & bits \\
\hline
4K & 712 & 46 \\
8K & 1524 & 43 \\
16K & 3196 & 41 \\
32K & 6721 & 39 \\
\hline
\end{tabular}


%\section{Scratch notes}
%
%\begin{align*}
%f(x)
%&= (x \& 2^{B-1}) + x \rightshift B \\
%&= (x \Mod 2^B) + x \Div 2^B \\
%&\questionEquals x \Mod (2^B-1)
%\end{align*}

%\bibliographystyle{plainnat}
%\bibliography{myrefs}

%\end{document}
