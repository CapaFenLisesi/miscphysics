%
% Copyright � 2012 Peeter Joot.  All Rights Reserved.
% Licenced as described in the file LICENSE under the root directory of this GIT repository.
%

%
%
%\documentclass{article}

%\input{../peeters_macros.tex}
%\input{../peeters_macros2.tex}

%\usepackage[bookmarks=true]{hyperref}

%\usepackage{color,cite,graphicx}
   % use colour in the document, put your citations as [1-4]
   % rather than [1,2,3,4] (it looks nicer, and the extended LaTeX2e
   % graphics package.
%\usepackage{latexsym,amssymb,epsf} % do not remember if these are
   % needed, but their inclusion can not do any damage


\chapter{Dot product linearity by construction}
\label{chap:dotLinearity}
%\author{Peeter Joot \quad peeterjoot@protonmail.com }
\date{ March 13, 2009.  dotLinearity.tex }

%\begin{document}

%\maketitle{}
%\tableofcontents
\section{Motivation}

Reading of \citep{byron1992mca} it is observed that the dot product when defined via geometrical constructs such as

\begin{equation}\label{eqn:dotLinearity:20}
\begin{aligned}
\Bx \cdot \By = \Abs{\Bx} \Abs{\By} \cos\theta
\end{aligned}
\end{equation}

is linear

\begin{equation}\label{eqn:dotLinearity:40}
\begin{aligned}
\Bx \cdot (\By + \Bz) = \Bx \cdot \By + \Bx \cdot \Bz
\end{aligned}
\end{equation}

Despite the fact that this is obvious when the dot product is given in algebraic form, this does not look obvious geometrically, so my
immediate thought was ``how would you show this geometrically''.  Sure enough, in the next paragraph is the statement that the reader will
want to show this by construction.  Here is such a demonstration and construction.

\imageFigure{../figures/miscphysics/dot_lin}{Sum of two vectors and their angles with another}{fig:dot_linearity}{0.4}

\section{Info from the figure}
\subsection{Law of cosines}

From \cref{fig:dot_linearity}, with \(\Be_1 = \xcap\) we have

\begin{equation}\label{eqn:dotLinearity:60}
\begin{aligned}
\By &= \Abs{\By} \cos\theta \Be_1 + \Abs{\By} \sin\theta \Be_2 \\
\Bz &= \Abs{\Bz} \cos\alpha \Be_1 + \Abs{\Bz} \sin\alpha \Be_2
\end{aligned}
\end{equation}

The vector sum \(\By + \Bz\) is therefore

\begin{equation}\label{eqn:dotLinearity:80}
\begin{aligned}
\By + \Bz &= (\Abs{\By} \cos\theta + \Abs{\Bz} \cos\alpha )\Be_1 + (\Abs{\By} \sin\theta + \Abs{\Bz} \sin\alpha ) \Be_2
\end{aligned}
\end{equation}

By using Pythagoras's law, a calculation of the squared length, produces the law of cosines

\begin{equation}\label{eqn:dotLinearity:100}
\begin{aligned}
\Abs{\By + \Bz}^2
&= (\Abs{\By} \cos\theta + \Abs{\Bz} \cos\alpha )^2 + (\Abs{\By} \sin\theta + \Abs{\Bz} \sin\alpha )^2 \\
&= \Abs{\By}^2 + \Abs{\Bz}^2 + 2 \Abs{\By} \Abs{\Bz} (\cos\theta \cos\alpha + \sin\theta \sin\alpha) \\
&= \Abs{\By}^2 + \Abs{\Bz}^2 + 2 \Abs{\By} \Abs{\Bz} \cos(\theta -\alpha) \\
\end{aligned}
\end{equation}

\subsection{Linearity by construction}

Okay, that is a digression, ... back to the original problem.  We get the dot product linearity follows from direct calculation
of the cosine of the angle between \(\xcap\) and \(\By + \Bz\).  Again from the figure we have

\begin{equation}\label{eqn:dotLinearity:120}
\begin{aligned}
\cos\beta &= \frac{\Abs{\By}\cos\theta + \Abs{\By}\cos\alpha}{\Abs{\By + \Bz}}
\end{aligned}
\end{equation}

or
\begin{equation}\label{eqn:dotLinearity:140}
\begin{aligned}
{\Abs{\By + \Bz}} \cos\beta &= {\Abs{\By}\cos\theta + \Abs{\By}\cos\alpha} \\
\end{aligned}
\end{equation}

But \(\xcap \cdot \By = \Abs{\By}\cos\theta \), and \(\xcap \cdot \Bz = \Abs{\By}\cos\alpha \), so we have

\begin{equation}\label{eqn:dotLinearity:160}
\begin{aligned}
{\Abs{\By + \Bz}} \cos\beta &= \xcap \cdot \By + \xcap \cdot \Bz
\end{aligned}
\end{equation}

Multiplying by \(\Abs{\Bx}\) we have

\begin{equation}\label{eqn:dotLinearity:180}
\begin{aligned}
\Abs{\Bx} {\Abs{\By + \Bz}} \cos\beta &= \Bx \cdot \By + \Bx \cdot \Bz
\end{aligned}
\end{equation}

The left hand side is \(\Bx \cdot (\By + \Bz)\), which completes the desired demonstration.

%\bibliographystyle{plainnat}
%\bibliography{myrefs}

%\end{document}
