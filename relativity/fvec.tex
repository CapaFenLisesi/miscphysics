%
% Copyright � 2012 Peeter Joot.  All Rights Reserved.
% Licenced as described in the file LICENSE under the root directory of this GIT repository.
%

%
%
%\documentclass{article}      % Specifies the document class

%\input{../peeters_macros.tex}
%
% The real thing:
%

                             % The preamble begins here.
\chapter{Relativistic dynamics from Lagrangian}
\label{chap:fvec}
%\author{Peeter Joot \quad peeterjoot@protonmail.com}         % Declares the author's name.
\date{ June 10, 2008.  fvec.tex }

%\begin{document}             % End of preamble and beginning of text.

%\maketitle{}

\section{}

David Tong's problem 3 on Lagrangian's is to show that the following

\begin{equation}
S = \kappa \sqrt{1 - \Bv^2/c^2} - \varphi
\end{equation}

can be used to find the equation of motion.  The constant \(\kappa\) is determined
by requiring correspondence with the classical limit for small velocities:

\begin{equation*}
\kappa \sqrt{1 - \Bv^2/c^2} \simeq \kappa \left( 1 - \inv{2} \Bv^2/c^2 \right) = \text{constant} + \inv{2} m \Bv^2 + \cdots
\end{equation*}

Thus by neglecting the higher order terms, we have the Newtonian kinetic energy for \(\kappa = -mc^2\), and the action to minimize is:

\begin{equation}
S = -m c^2\sqrt{1 - \Bv^2/c^2} - \varphi \simeq \inv{2}m \Bv^2 - \left(\varphi + m c^2\right)
\end{equation}

\subsection{Relativistic three vector equation of motion}

Performing the calculations

\begin{equation}\label{eqn:fvec:40}
\begin{aligned}
\PDb{S}{x^i} &= \frac{d}{dt} \PDb{S}{\dot{x}^i} \\
-\PDb{\varphi}{x^i} &= \frac{d}{dt} \left(-mc^2 \mathLabelBox{\inv{\sqrt{1-\Bv^2/c^2}}}{\(=\gamma\)} (\inv{2})(-2\dot{x}^i/c^2)\right) \\
-\PDb{\varphi}{x^i} &= \frac{d}{dt} \left( m \gamma \dot{x}^i\right) \\
\sum \Be_i (-\PDb{\varphi}{x^i}) &= \sum \Be_i \frac{d}{dt} \left( m \gamma \dot{x}^i\right) \\
\end{aligned}
\end{equation}

This provides the expected relativistically corrected equation of Newton's law:

\begin{equation}
\frac{d (\gamma \Bp)}{dt} = -\grad \varphi
\end{equation}

\subsection{Relativistic four vector equation of motion}

Logically the above is not satisfactory.  We want a four vector version of it.  Can the same Lagrangian
be used to achieve this.  I thought I had made such a calculation with what I guessed was the appropriate
modification of the Lagrangian equations to treat space and time symmetrically:

\begin{equation}
\PDb{S}{x^{\mu}} = \frac{d}{ds} \left( \PDb{S}{\frac{d x^{\mu}}{ds}} \right)
\end{equation}

I have no proof that this is valid, and really need to study some variational calculus before I can make any
such claim.  I have also been unable to satisfactorily reproduce my original derivation of:

\begin{equation}
F = \frac{d}{d\tau}\left( m\frac{dX}{d\tau} \right)
\end{equation}

I may have made compensatory errors to arrive at the answer I desired.  Have to go dig up my original notes
where I thought I had done this.

\section{OLDER FOUR VECTOR NOTES}

I was summarizing for myself the various four-vectors of mechanics:

\begin{equation}\label{eqn:fvec:60}
\begin{aligned}
X &= ct + \mathbf{X} \\
V &= \frac{d X}{d\tau} = \gamma(c + \mathbf{v}) \\
P &= m V = E/c + \gamma\mathbf{p} \\
f &= m\frac{d^2 X}{d\tau^2} = m\frac{d V}{d\tau} \\
\end{aligned}
\end{equation}

where:

\begin{equation}\label{eqn:fvec:80}
\begin{aligned}
\gamma^{-2} &= 1 - {\lvert \mathbf{v}/c \rvert}^2 \\
ds = c d\tau &= {\left(\frac{dx}{d\lambda} \cdot \frac{dx}{d\lambda}\right)}^{1/2} d\lambda \\
X \cdot X = {\lvert X \rvert}^2 &= c^2t^2 - {\lvert \mathbf{X} \rvert}^2 \\
E &= \int f \cdot (c d\tau) \\
\mathbf{v} &= \frac{d\mathbf{X}}{dt} \\
\mathbf{p} &= m\mathbf{v} \\
\end{aligned}
\end{equation}

Invariants for the first three four vectors are:

\begin{equation}\label{eqn:fvec:100}
\begin{aligned}
{\lvert X \rvert}^2 &= c^2 t^2 - {\lvert \mathbf{X} \rvert}^2 = c^2 \tau^2 \\
{\lvert V \rvert}^2 &= \gamma^2 (c^2 - {\lvert \mathbf{v} \rvert}^2) = c^2 \\
{\lvert P \rvert}^2 &= m^2 {\lvert V \rvert}^2 = m^2 c^2 \\
\end{aligned}
\end{equation}

Is the Minkowski norm of the four vector force:

\begin{equation}\label{eqn:fvec:20}
f = m\frac{d^2 X}{d\tau^2}
\end{equation}

also an invariant?  I think it has to be.  Assuming that is the case, what would the value (and significance if any) of this be?

%\end{document}               % End of document.
