%
% Copyright � 2012 Peeter Joot.  All Rights Reserved.
% Licenced as described in the file LICENSE under the root directory of this GIT repository.
%

%
%
%\input{../peeter_prologue_print.tex}
%\input{../peeter_prologue_widescreen.tex}

\chapter{Classical Electrodynamic gauge interaction}
\label{chap:gaugeInteractionHamiltonian}
%\useCCL
\blogpage{http://sites.google.com/site/peeterjoot/math2010/gaugeInteractionHamiltonian.pdf}
\date{Oct 22, 2010}
\revisionInfo{gaugeInteractionHamiltonian.tex}

\beginArtWithToc
%\beginArtNoToc

\section{Motivation}

In \citep{desai2009quantum} chapter 6, we have a statement that in classical mechanics the electromagnetic interaction is due to a transformation of the following form

\begin{equation}\label{eqn:gaugeInteractionHamiltonian:1}
\begin{aligned}
\Bp &\rightarrow \Bp - \frac{e}{c} \BA \\
E &\rightarrow E - e \phi
\end{aligned}
\end{equation}

Let us verify that this does produce the classical interaction law.  Putting a more familiar label on this, we should see that we obtain the Lorentz force law from a transformation of the Hamiltonian.

\section{Hamiltonian equations}

Recall that the Hamiltonian was defined in terms of conjugate momentum components \(p_k\) as
\begin{equation}\label{eqn:gaugeInteractionHamiltonian:4}
H(x_k, p_k) = \xdot_k p_k - \LL(x_k, \xdot_k),
\end{equation}

we can take \(x_k\) partials to obtain the first of the Hamiltonian system of equations for the motion
\begin{equation}\label{eqn:gaugeInteractionHamiltonian:50}
\begin{aligned}
\PD{x_k}{H}
&= - \PD{x_k}{\LL}  \\
&= - \frac{d}{dt} \PD{\xdot_k}{\LL}
\end{aligned}
\end{equation}

With \(p_k \equiv \PDi{\xdot_k}{\LL}\), and taking \(p_k\) partials too, we have the system of equations

\begin{subequations}
\begin{equation}\label{eqn:gaugeInteractionHamiltonian:5}
\PD{x_k}{H} = - \frac{d p_k}{dt}
\end{equation}
\begin{equation}\label{eqn:gaugeInteractionHamiltonian:6}
\PD{p_k}{H} = \xdot_k
\end{equation}
\end{subequations}

\section{Classical interaction}

Starting with the free particle Hamiltonian

\begin{equation}\label{eqn:gaugeInteractionHamiltonian:20}
H = \frac{\Bp}{2m},
\end{equation}

we make the transformation required to both the energy and momentum terms

\begin{equation}\label{eqn:gaugeInteractionHamiltonian:21}
H - e\phi = \frac{\left(\Bp - \frac{e}{c} \BA\right)^2 }{2m} = \inv{2m} \Bp^2 - \frac{e}{m c} \Bp \cdot \BA + \inv{2m} \left(\frac{e}{c}\right)^2 \BA^2
\end{equation}

From \eqnref{eqn:gaugeInteractionHamiltonian:6} we find

\begin{equation}\label{eqn:gaugeInteractionHamiltonian:22}
\frac{d x_k}{dt} = \PD{p_k}{H} = \inv{m} \left( p_k - \frac{e}{c} A_k \right),
\end{equation}

or
\begin{equation}\label{eqn:gaugeInteractionHamiltonian:23}
p_k = m \frac{d x_k}{dt} + \frac{e}{c} A_k.
\end{equation}

Taking derivatives and employing \eqnref{eqn:gaugeInteractionHamiltonian:5} we have
\begin{equation}\label{eqn:gaugeInteractionHamiltonian:70}
\begin{aligned}
\frac{d p_k}{dt}
&= m \frac{d^2 x_k}{dt^2} + \frac{e}{c} \frac{d A_k}{dt}  \\
&= -\PD{x_k}{H} \\
&=
\inv{m} \frac{e}{c} p_n \PD{x_k}{A_n} - e \PD{x_k}{\phi}
- \inv{m} \left(\frac{e}{c}\right)^2 A_k \PD{x_k}{A_k} \\
&=
\inv{m} \frac{e}{c} \left(
m \frac{d x_n}{dt} + \frac{e}{c} A_n
\right)
\PD{x_k}{A_n}
 - e \PD{x_k}{\phi}
- \inv{m} \left(\frac{e}{c}\right)^2 A_k \PD{x_k}{A_k} \\
&=
\frac{e}{c} \frac{d x_n}{dt}
\PD{x_k}{A_n}
 - e \PD{x_k}{\phi}
\end{aligned}
\end{equation}

Rearranging and utilizing the convective derivative expansion \(d/dt = (d x_a/dt) \PDi{x_a}{}\) (ie: chain rule), we have

\begin{equation}\label{eqn:gaugeInteractionHamiltonian:25}
\begin{aligned}
m \frac{d^2 x_k}{dt^2}
&=
\frac{e}{c}
\frac{d x_n}{dt}
\left(
\PD{x_k}{A_n}
-
\PD{x_n}{A_k}
\right)
 - e \PD{x_k}{\phi}
\end{aligned}
\end{equation}

We guess and expect that the first term of \eqnref{eqn:gaugeInteractionHamiltonian:25} is \(e (\Bv/c \cross \BB)_k\).  Let us verify this

\begin{equation}\label{eqn:gaugeInteractionHamiltonian:90}
\begin{aligned}
(\Bv \cross \BB)_k
&= \xdot_m B_d \epsilon_{k m d} \\
&= \xdot_m ( \epsilon_{d a b} \partial_a A_b ) \epsilon_{k m d} \\
&= \xdot_m \partial_a A_b \epsilon_{d a b} \epsilon_{d k m}
\end{aligned}
\end{equation}

Since \(\epsilon_{d a b} \epsilon_{d k m} = \delta_{a k} \delta_{b m} - \delta_{a m} \delta_{b k}\) we have

\begin{equation}\label{eqn:gaugeInteractionHamiltonian:110}
\begin{aligned}
(\Bv \cross \BB)_k
&= \xdot_m \partial_a A_b \epsilon_{d a b} \epsilon_{d k m} \\
&=
\xdot_m \partial_a A_b \delta_{a k} \delta_{b m}
-\xdot_m \partial_a A_b \delta_{a m} \delta_{b k} \\
&=
\xdot_m ( \partial_k A_m - \partial_m A_k )
\end{aligned}
\end{equation}

Except for a difference in dummy summation variables, this matches what we had in \eqnref{eqn:gaugeInteractionHamiltonian:25}.  Thus we are able to put that into the traditional Lorentz force vector form

\begin{equation}\label{eqn:gaugeInteractionHamiltonian:30}
m \frac{d^2 \Bx}{dt^2} = e \frac{\Bv}{c} \cross \BB + e \BE.
\end{equation}

It is good to see that we get the classical interaction from this transformation before moving on to the trickier seeming QM interaction.

\EndArticle
