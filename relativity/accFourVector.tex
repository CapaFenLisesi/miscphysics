%
% Copyright � 2012 Peeter Joot.  All Rights Reserved.
% Licenced as described in the file LICENSE under the root directory of this GIT repository.
%

%
%
%\documentclass{article}

%\input{../peeters_macros.tex}
%\input{../peeters_macros2.tex}

%%\usepackage{listings}
%%\usepackage{txfonts} % for ointctr... (also appears to make "prettier" \int and \sum's)
%\usepackage[bookmarks=true]{hyperref}

%\usepackage{color,cite,graphicx}
   % use colour in the document, put your citations as [1-4]
   % rather than [1,2,3,4] (it looks nicer, and the extended LaTeX2e
   % graphics package.
%\usepackage{latexsym,amssymb,epsf} % do not remember if these are
   % needed, but their inclusion can not do any damage


\chapter{Relativistic acceleration}
\label{chap:accFourVector}
%\author{Peeter Joot \quad peeterjoot@protonmail.com }
\date{ April 10, 2009.   accFourVector.tex }

%\begin{document}

%\maketitle{}
%\tableofcontents
\section{Motivation}

Continuing on with reading of \citep{pauli1981tr}, having
clarified aspects of the four vector velocity in \chapcite{PJrelativityFourVectorVelocity}, it is now
time to move on to acceleration.

Do the chain rule calculations for the acceleration four vector equation given in equation (193).

\section{Compute it}

Compute the spatial and timelike components of the acceleration

\begin{equation}\label{eqn:accFourVector:20}
\begin{aligned}
B^\mu
&= \frac{d^2 x^\mu}{d\tau^2} \\
&= \frac{d }{d\tau} \left( \frac{d x^\mu }{d\tau} \right) \\
&= \frac{d }{d\tau} \left( \frac{d x^\mu }{dt} \frac{dt}{d\tau} \right) \\
&= \left( \frac{d }{d\tau} \frac{d x^\mu }{dt} \right) \frac{dt}{d\tau} + \frac{d x^\mu }{dt} \frac{d^2t}{d\tau^2} \\
&= \frac{d^2 x^\mu }{dt^2} \left( \frac{dt}{d\tau} \right)^2 + \frac{d x^\mu }{dt} \frac{d^2t}{d\tau^2} \\
\end{aligned}
\end{equation}

For \(\mu \in \{1,2,3\}\), the \({d^2 x^\mu }/{dt^2}\) terms are the regular old spatial acceleration components.
, and \(dx^4/dt = c\).  Writing \(\Bu^2 = \sum_{k=1}^3 (dx^k/dt)^2\), and \(\beta^2 = \Bu^2/c^2\), we have

\begin{equation}\label{eqn:accFourVector:40}
\begin{aligned}
B^k &= \frac{d^2 x^k }{dt^2} \inv{1-\beta^2} + \frac{d x^k }{dt} \frac{d^2t}{d\tau^2} \\
B^4 &= 0 + c \frac{d^2t}{d\tau^2} \\
\end{aligned}
\end{equation}

In both of these is the \(d^2t/d\tau^2\) term.  Let us expand that.

\begin{equation}\label{eqn:accFourVector:60}
\begin{aligned}
\frac{d^2t}{d\tau^2}
&= \frac{d}{d\tau} \left( \inv{\sqrt{ 1 - \Bu^2/c^2 }} \right) \\
&= \frac{-1}{c^2} \frac{(-1/2) }{({ 1 - \Bu^2/c^2 })^{3/2}} \frac{d\Bu^2}{d\tau} \\
&= \frac{1}{c^2} \frac{(1/2) }{({ 1 - \Bu^2/c^2 })^{3/2}} 2 \Bu \cdot \frac{d\Bu}{d\tau} \\
&= \frac{1}{c^2} \frac{1}{({ 1 - \Bu^2/c^2 })^{3/2}} \Bu \cdot \frac{d\Bu}{dt} \frac{dt}{d\tau} \\
&= \frac{1}{c^2} \frac{1}{({ 1 - \Bu^2/c^2 })^{2}} \Bu \cdot \frac{d\Bu}{dt} \\
\end{aligned}
\end{equation}

In vector form, with \(\Ba = d\Bu/dt\), we now have the following

\begin{equation}\label{eqn:acc_four_vector:accSpaceTimeSplit}
\begin{aligned}
\BB &= \Ba \inv{1-\beta^2} + \Bu (\Bu \cdot \Ba) \inv{c^2} \inv{ (1-\beta^2)^2} \\
B^4 &= \inv{c} (\Bu \cdot \Ba) \inv{ (1-\beta^2)^2}
\end{aligned}
\end{equation}

This reproduces the equation from the Pauli text (except for the imaginary factor \(i\) due to the Minkowski notation).
%  Except for \(\gamma\) factors this calculation has a similar final form to that of the decomposition of acceleration in terms of radial components.  This is kind of

\section{Approximate expansion}

This relativistic acceleration should match the Newtonian acceleration for small velocities.  Lets expand it to verify and inspect the form.  Taylor expansions of the \(\gamma\) factors is required.

\begin{equation}\label{eqn:accFourVector:80}
\begin{aligned}
\inv{1 - \beta^2}
&=
1
+ \frac{(-1)}{1!}(-\beta^2)
+ \frac{(-1)(-2)}{2!}(-\beta^2)^2
+ \frac{(-1)(-2)(-3)}{3!}(-\beta^2)^3
+ \cdots \\
&=
1
+ \beta^2
+ \beta^4
+ \beta^6
+ \cdots \\
\end{aligned}
\end{equation}

This is convergent since \(\beta < 1\), and for non-relativistic rates the higher order terms die off very quickly.

For the \(\gamma^4\) term we want

\begin{equation}\label{eqn:accFourVector:100}
\begin{aligned}
\inv{(1 - \beta^2)^2}
&=
1
+ \frac{(-2)}{1!}(-\beta^2)
+ \frac{(-2)(-3)}{2!}(-\beta^2)^2
+ \frac{(-2)(-3)(-4)}{3!}(-\beta^2)^3
+ \cdots \\
&=
1
+ 2 \beta^2
+ 3 \beta^4
+ 4 \beta^6
+ \cdots \\
\end{aligned}
\end{equation}

Again, this is convergent.  Substitution back into \eqnref{eqn:acc_four_vector:accSpaceTimeSplit} we have for the spatial part

\begin{equation}\label{eqn:accFourVector:120}
\begin{aligned}
\BB &= \Ba
\left(
1
+ \beta^2
+ \beta^4
+ \beta^6
+ \cdots
\right)
+ \Bu (\Bu \cdot \Ba) \inv{c^2}
\left(
1
+ 2 \beta^2
+ 3 \beta^4
+ 4 \beta^6
+ \cdots
\right)
\\
\end{aligned}
\end{equation}

Writing \(\Bbeta = \Bu/c\), this is

\begin{equation}\label{eqn:accFourVector:140}
\begin{aligned}
\BB &=
\Ba + \Bbeta (\Bbeta \cdot \Ba)
+ \Bbeta^2 \left( \Ba + 2 \Bbeta (\Bbeta \cdot \Ba) \right)
+ \Bbeta^4 \left( \Ba + 3 \Bbeta (\Bbeta \cdot \Ba) \right)
+ \cdots
\\
\end{aligned}
\end{equation}

for small \(\Abs{\Bbeta}\) we have the Newtonian acceleration.  Another case that kills off terms is the circular motion condition \(\Bbeta \cdot \Ba = 0\), for which we have just

\begin{equation}\label{eqn:accFourVector:160}
\begin{aligned}
\BB &= \Ba
\left(
1
+ \beta^2
+ \beta^4
+ \beta^6
+ \cdots
\right)
\end{aligned}
\end{equation}

So for circular motion the first order of magnitude correction to the acceleration is
\begin{equation}\label{eqn:accFourVector:180}
\begin{aligned}
\BB &= \Ba ( 1 + \Bu^4/c^4 )
\end{aligned}
\end{equation}

On the other hand for non-circular motion the more general first adjustment to the Newtonian acceleration is

\begin{equation}\label{eqn:accFourVector:200}
\begin{aligned}
\BB
&= \Ba + \frac{\Bu}{c} \Abs{\frac{\Bu}{c}}\Abs{\Ba} \cos(\Bu,\Ba) \\
&= \Ba + \ucap \left(\frac{\Bu}{c}\right)^2 \Abs{\Ba} \cos(\Bu,\Ba) \\
&= \ucap \Abs{\Ba} \cos(\Bu,\Ba) \left(1 + \left(\frac{\Bu}{c}\right)^2 \right)
+ \ucap \left(\ucap \wedge \Ba\right)
\\
\end{aligned}
\end{equation}

Or, putting back the explicit dot products

\begin{equation}\label{eqn:accFourVector:220}
\begin{aligned}
\BB
&= \ucap (\ucap \cdot \Ba) \left(1 + \left(\frac{\Bu}{c}\right)^2 \right) + \ucap \left(\ucap \wedge \Ba\right)
\\
\end{aligned}
\end{equation}

We see here that we have a scale correction only in the direction of the projection of the acceleration onto the direction of the velocity, and
in the perpendicular direction to the acceleration the components go untouched.

%and the second correction is
%
%\begin{align*}
%\BB
%&= \Ba \left( 1 + \frac{\Bu^4}{c^4} \right) +
%\frac{\Bu}{c}
%\Abs{\frac{\Bu}{c} }
%\Abs{\Ba} \cos(\Bu,\Ba) \left( 1 + 2 \frac{\Bu^4}{c^4} \right) \\
%&= \Ba \left( 1 + \frac{\Bu^4}{c^4} \right)
%+
%\frac{\Bu}{c}
%\Abs{\frac{\Bu}{c} }
%\Abs{\Ba} \cos(\Bu,\Ba)
%+ 2 \ucap \Abs{\Ba} \cos(\Bu,\Ba) \frac{\Bu^6}{c^6} \\
%&=
%\ucap \Abs{\Ba} \cos(\Bu,\Ba) \left( 1 + \frac{\Bu^2}{c^2} + \frac{\Bu^4}{c^4} + 2 \frac{\Bu^6}{c^6} \right) +
%\ucap (\ucap \wedge {\Ba}) \left( 1 + \frac{\Bu^4}{c^4} \right)
%\end{align*}
%% a = a ucap ucap = a . ucap ucap + a ^ ucap ucap
%% a = ucap ucap a = a . ucap ucap + ucap ucap ^ a

%\bibliographystyle{plainnat}
%\bibliography{myrefs}

%\end{document}
