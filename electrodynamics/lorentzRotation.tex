%
% Copyright � 2012 Peeter Joot.  All Rights Reserved.
% Licenced as described in the file LICENSE under the root directory of this GIT repository.
%

%
%
%\input{../peeter_prologue_print.tex}
%\input{../peeter_prologue_widescreen.tex}

\chapter{Lorentz Force Trajectory}
\label{chap:lorentzRotation}
%\blogpage{http://sites.google.com/site/peeterjoot/maxwell/lorentz_rotation.pdf}
\date{May 7, 2008.}
%\useCCL
\revisionInfo{lorentzRotation.tex}

\beginArtWithToc
%\beginArtNoToc

\section{Solving the Lorentz force equation in the non-relativistic limit}

\subsection{The problem}
%% http://www.physicsforums.com/showthread.php?t=231213

\citep{doran2003gap} treats the solution of the Lorentz force equation in covariant form.  Let us try this for non-relativistic motion for constant fields, but without making the usual assumptions about perpendicular electric and magnetic fields, or alignment of the axis.  Our equation to solve is

%While some simplified variants of the equation can be solved by picking an appropriate axis of symmetry, is a solution for the trajectory for a more general field pair possible without the softistication of the GA methods?
%
%Here is a bash at starting it.  My approach essentially replays a rigid body rotation derivation backwards \href{http://www.damtp.cam.ac.uk/user/tong/dynamics/three.pdf}{such as the one found in Tong}.
%
%In the Lorentz Force Law, we essentially have an equation like the rigid body rotation equation \(y' = \omega \times y + x_0'\), that resulted from \(y = R x + x_0\).  That is a good hint about the anti-derivative required for this Lorentz problem, so we have to go backwards from the cross product, and solve for the rotation:

\begin{equation}\label{eqn:lorentzRotation:10}
\frac{d}{dt} \left( \gamma m \Bv \right) = q \left( \BE + \frac{\Bv}{c} \times \BB \right),
\end{equation}

so in the non-relativistic limit we want to solve the matrix equation

\begin{equation}\label{eqn:lorentzRotation:30}
\begin{aligned}
\Bv' &= \frac{q}{m} \BE + \frac{q}{ m c } \Omega \Bv \\
\Omega &=
\begin{bmatrix}
0 & B_3 & -B_2 \\
-B_3 & 0 & B_1 \\
B_2 & -B_1 & 0 \\
\end{bmatrix}.
\end{aligned}
\end{equation}
%Following the rigid body treatment (in Tong above) this antisymmetric matrix can be expressed in terms of a rotation matrix (ie: essentially it is a rotation matrix derivative with a rotation factored out of it).  So, let
%
%\begin{equation}\label{eqn:lorentzRotation:70}
%\Omega = R' R^\T,
%\end{equation}
%
%and use one more trick from the rigid body analysis:
%
%\begin{equation}\label{eqn:lorentzRotation:90}
%(RR^\T)' = R' R^\T + R {R'}^\T = I' = 0
%\end{equation}
%
%\begin{align*}
%\Bv' - R' R^\T \Bv
%&= \Bv' + R {R'}^\T \Bv  \\
%&= R (R^\T \Bv' + {R'}^\T \Bv)  \\
%&\implies \\
%R (R^\T \Bv)' &= \frac{q}{m} \BE
%\end{align*}
%
%Thus the solution can be written as two equations, one explicit for \(\Bv\), and one matrix differential equation to solve for R:
%
%\begin{equation}\label{eqn:lorentzRotation:110}
%\Bv = \frac{q}{m} \BE t + R^\T \Bc
%\end{equation}
%
%\begin{equation}\label{eqn:lorentzRotation:130}
%R' = \frac{q}{m}
%\begin{bmatrix}
%0 & B_3 & -B_2 \\
%-B_3 & 0 & B_1 \\
%B_2 & -B_1 & 0 \\
%\end{bmatrix}
%R
%\end{equation}
%
%\section{Solving for the rotation}
%
%I have not tried solving that last equation numerically (or analytically), but intuition says that diagonalization would do the trick.  ie: with a solution having a term of the form:
%
%\begin{equation}\label{eqn:lorentzRotation:150}
%e^{Dt}
%\end{equation}
%
%I should try this with an example to see if it all holds together (at the bare minimum, if I did things right, it should work for \(\BB\) along the \(\zcap\) axis;)
%
%Now as mentioned in \citep{doran2003gap} there is a treatment of this problem in chapter 5 on spacetime algebra.  So far reading that book, I had temporarily skipped that chapter for some easier stuff in chapter 6 (vector calculus chapter).  Going back and reading just this fragment, I can not say I fully understand their treatment.  It is interesting looking though;)  They end up reformulating the equation as:
%
%\begin{equation}\label{eqn:lorentzRotation:170}
%m v' = q F \cdot v,
%\end{equation}
%
%where \(F\) is a combined electrodynamic field:
%
%\begin{equation}\label{eqn:lorentzRotation:190}
%F = \BE + I \BB,
%\end{equation}
%
%and \(v = dx/d\tau\) is the four vector proper velocity.
%
%Then they introduce a rotor parametrization of the velocity, and an equation to solve for the rotor that is similar to the rotation matrix equation I had (factor of two difference because the rotor is a double sided half angle operator unlike the single sided rotation matrix) :
%
%\begin{equation}\label{eqn:lorentzRotation:210}
%R' = \frac{q}{2m} F R
%\end{equation}
%
%There is a lot of similarities to what I hacked up, but it will probably take me a while before I can get to the point to digest and compare the two.   Their rotor equation ends up with terms for both electric field and magnetic field whereas mine is magnetic only.  Something that involves both makes more sense.

\subsection{First attempt}

This is very much like the plain old LDE

\begin{equation}\label{eqn:lorentzRotation:300}
x' = a + b x,
\end{equation}

which we can solve using integrating factors
\begin{equation}\label{eqn:lorentzRotation:700}
\begin{aligned}
x' - b x &= a \\
e^{b t} \left( x e^{-b t} \right)' &=
\end{aligned}
\end{equation}

This we can rearrange and integrate to find a solution to the non-homogeneous problem

\begin{equation}\label{eqn:lorentzRotation:320}
x e^{-b t} = \int a e^{-b t}.
\end{equation}

This solution to the non-homogeneous equation is thus
\begin{equation}\label{eqn:lorentzRotation:340}
x - x_0 = e^{b t} \int_{\tau = 0}^t a e^{-b \tau} = \frac{a}{b} \left(e^{bt} - 1 \right).
\end{equation}

Because this already incorporates the homogeneous solution \(x = C e^{b t}\), this is also the general form of the solution.

Can we do something similar for the matrix equation of \eqnref{eqn:lorentzRotation:30}?  It is tempting to try, rearranging in the same way like so

\begin{equation}\label{eqn:lorentzRotation:360}
e^{ \frac{q}{m c}\Omega t} \left( e^{-\frac{q}{m c} \Omega t} \Bv \right)' = \frac{q}{m} \BE.
\end{equation}

Our matrix exponentials are perfectly well formed, but we will run into trouble attempting this.  We can get as far as the integral above before running into trouble

\begin{equation}\label{eqn:lorentzRotation:380}
\Bv - \Bv_0 = \frac{q}{m}
e^{ \frac{q }{m c} \Omega t}
\left( \int_{\tau = 0}^t
e^{ -\frac{q }{m c} \Omega \tau}
\right) \BE.
\end{equation}

Only when \(\Det{\Omega} \ne 0\) do we have

\begin{equation}\label{eqn:lorentzRotation:400}
\int e^{ -\frac{q }{m c} \Omega \tau} =
- \frac{m c}{q} \Omega^{-1} e^{ -\frac{q }{m c} \Omega \tau},
\end{equation}

but in our case, this determinant is zero, due to the antisymmetry that is built into our magnetic field tensor.  It appears that we need a different strategy.

\subsection{Second attempt}

It is natural to attempt to pull out our spectral theorem toolbox.  We find three independent eigenvalues for our matrix \(\Omega\) (one of which is naturally zero due to the singular nature of the matrix).

These eigenvalues are

\begin{equation}\label{eqn:lorentzRotation:420}
\begin{aligned}
\lambda_1 &= 0 \\
\lambda_2 &= i\Abs{\BB} \\
\lambda_3 &= -i\Abs{\BB}.
\end{aligned}
\end{equation}

The corresponding orthonormal eigenvectors are found to be

\begin{equation}\label{eqn:lorentzRotation:440}
\begin{aligned}
\Bu_1 &= \frac{\BB}{\Abs{\BB}} \\
\Bu_2 &=
\inv{\Abs{\BB} \sqrt{ 2(B_1^2 + B_3^2) } }
\begin{bmatrix}
i B_1 B_2 - B_3 \Abs{\BB} \\
-i(B_1^2 + B_3^2) \\
i B_2 B_3 + B_1 \Abs{\BB} \\
\end{bmatrix} \\
\Bu_3 &=
\inv{\Abs{\BB} \sqrt{ 2(B_1^2 + B_3^2) } }
\begin{bmatrix}
-i B_1 B_2 - B_3 \Abs{\BB} \\
i(B_1^2 + B_3^2) \\
-i B_2 B_3 + B_1 \Abs{\BB} \\
\end{bmatrix}
\end{aligned}
\end{equation}

The last pair of eigenvectors are computed with the assumption that not both of \(B_1\) and \(B_3\) are zero.  This allows for the spectral decomposition

\begin{equation}\label{eqn:lorentzRotation:460}
\begin{aligned}
U &=
\begin{bmatrix}
\Bu_1 & \Bu_2 & \Bu_3
\end{bmatrix} \\
D &= \begin{bmatrix}
0 & 0 & 0 \\
0 & i \Abs{\BB} & 0 \\
0 & 0 & -i \Abs{\BB}
\end{bmatrix} \\
\Omega &= U D U^\conj
\end{aligned}
\end{equation}

We can use this to decouple our equation

\begin{equation}\label{eqn:lorentzRotation:480}
\Bv' = \frac{q}{m} \BE + \frac{q}{ m c } U D U^\conj \Bv
\end{equation}

forming instead
\begin{equation}\label{eqn:lorentzRotation:500}
\begin{aligned}
\Bw &= U^\conj \Bv \\
\BF &= U^\conj \BE \\
\Bw' &= \frac{q}{m} \BF + \frac{q}{ m c } D \Bw.
\end{aligned}
\end{equation}

Written out explicitly, this is a set of three independent equations

\begin{equation}\label{eqn:lorentzRotation:520}
\begin{aligned}
w_1' &= \frac{q}{m} F_1 \\
w_2' &= \frac{q}{m} F_2 + \frac{q i \Abs{\BB}}{ m c } w_2 \\
w_3' &= \frac{q}{m} F_3 - \frac{q i \Abs{\BB}}{ m c } w_3
\end{aligned}
\end{equation}

Utilizing \eqnref{eqn:lorentzRotation:340} our solution is

\begin{equation}\label{eqn:lorentzRotation:530}
\begin{aligned}
w_1 - w_1(0) &= \frac{q}{m} F_1 t \\
w_2 - w_2(0) &= - \frac{i c F_2 }{\Abs{\BB}} \left( e^{ \frac{q i \Abs{\BB} t}{ m c } } - 1 \right) \\
w_3 - w_3(0) &= \frac{i c F_3 }{\Abs{\BB}} \left( e^{ -\frac{q i \Abs{\BB} t}{ m c } } - 1 \right)
\end{aligned}
\end{equation}

Reinserting matrix form we have

\begin{equation}\label{eqn:lorentzRotation:540}
\Bw - \Bw(0) =
\begin{bmatrix}
\frac{q}{m} \Be_1^\T U^\conj \BE t \\
\frac{2 c \Be_2^\T U^\conj \BE }{\Abs{\BB}}
e^{ \frac{q i \Abs{\BB} t}{ 2 m c } } \sin \left( \frac{q \Abs{\BB} t}{ 2 m c } \right) \\
\frac{2 c \Be_3^\T U^\conj \BE }{\Abs{\BB}}
e^{ -\frac{q i \Abs{\BB} t}{ 2 m c } } \sin \left( \frac{q \Abs{\BB} t}{ 2 m c } \right) \\
\end{bmatrix}
\end{equation}

with

\begin{equation}\label{eqn:lorentzRotation:560}
\begin{aligned}
f_1 &= \frac{q}{m} t \\
f_2 &= \frac{2 c }{\Abs{\BB}} e^{ \frac{q i \Abs{\BB} t}{ 2 m c } } \sin \left( \frac{q \Abs{\BB} t}{ 2 m c } \right) \\
f_3 &= \frac{2 c }{\Abs{\BB}} e^{ -\frac{q i \Abs{\BB} t}{ 2 m c } } \sin \left( \frac{q \Abs{\BB} t}{ 2 m c } \right)
\end{aligned}
\end{equation}

We have

\begin{equation}\label{eqn:lorentzRotation:580}
\Bv - \Bv(0) = U
\begin{bmatrix}
f_1 \Be_1^\T U^\conj \BE \\
f_2 \Be_2^\T U^\conj \BE \\
f_3 \Be_3^\T U^\conj \BE \\
\end{bmatrix}.
\end{equation}

Observe that the dot products embedded here can be nicely expressed in terms of the eigenvectors since

\begin{equation}\label{eqn:lorentzRotation:600}
U^\conj \BE
=
\begin{bmatrix}
\Bu_1^\conj \cdot \BE \\
\Bu_2^\conj \cdot \BE \\
\Bu_3^\conj \cdot \BE
\end{bmatrix}.
\end{equation}

Our solution is thus a weighted sum of projections of the electric field vector \(\BE\) onto the eigenvectors formed strictly from the magnetic field tensor

\begin{equation}\label{eqn:lorentzRotation:620}
\Bv - \Bv(0) = \sum f_i \Bu_i ( \Bu_i^\conj \cdot \BE ).
\end{equation}

Recalling that \(\Bu_i = \BB/\Abs{\BB}\), the unit vector that lies in the direction of the magnetic field, we have

\begin{equation}\label{eqn:lorentzRotation:640}
\Bv - \Bv(0) = \frac{q t}{m} \Bcap (\Bcap \cdot \BE)
+ \sum_{i=2}^3 f_i \Bu_i ( \Bu_i^\conj \cdot \BE ).
\end{equation}

Also observe that this is a manifestly real valued solution since remaining eigenvectors are conjugate pairs \(\Bu_2 = \Bu_3^\conj\) as are the differential solutions \(f_2 = f_3^\conj\).  This leaves us with

\begin{equation}\label{eqn:lorentzRotation:660}
\Bv - \Bv(0) = \frac{q t}{m} \Bcap (\Bcap \cdot \BE)
+
\frac{4 c }{\Abs{\BB}} \sin \left( \frac{q \Abs{\BB} t}{ 2 m c } \right)
\Real \left(
e^{ \frac{q i \Abs{\BB} t}{ 2 m c } }
\Bu_2 ( \Bu_2^\conj \cdot \BE )
\right).
\end{equation}

It is natural to express \(\Bu_2\) in terms of the direction cosines \(b_i\) of the magnetic field vector \(\BB = \Abs{\BB}(b_1, b_2, b_3)\)

\begin{equation}\label{eqn:lorentzRotation:680}
\Bu_2 =
\inv{\sqrt{ 2(b_1^2 + b_3^2) }}
\begin{bmatrix}
i b_1 b_2 - b_3 \\
-i(b_1^2 + b_3^2) \\
i b_2 b_3 + b_1 \\
\end{bmatrix}.
\end{equation}

Is there a way to express this beastie in a coordinate free fashion?  I experimented with this a bit looking for something that would provide some additional geometrical meaning but did not find it.  Further play with that is probably something for another day.
%This seems slightly reminisent of the first order Legrengre projections.  Do we have relations here to components projected onto a reference plane?

What we do see is that our velocity has two main components, one of which is linearly increasing in proportion to the colinearity of the magnetic and electric fields.  The other component is oscillatory.  With a better geometrical description of that eigenvector we could perhaps understand the mechanics a bit better.
%  That second component of the solution does not appear to be precisely related to the perpendicular to the fields, so what exactly

\EndArticle
