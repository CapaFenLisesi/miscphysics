%
% Copyright � 2012 Peeter Joot.  All Rights Reserved.
% Licenced as described in the file LICENSE under the root directory of this GIT repository.
%

%
%
%\input{../peeter_prologue_print.tex}
%\input{../peeter_prologue_widescreen.tex}

\chapter{Infinite square well wavefunction}
\label{chap:liboff41}
%\useCCL
\blogpage{http://sites.google.com/site/peeterjoot/math2010/liboff41.pdf}
\date{May 31, 2010}
\revisionInfo{liboff41.tex}

%\beginArtWithToc
\beginArtNoToc

\section{Motivation}

Work problem 4.1 from \citep{liboff2003iqm}, calculation of the eigensolution for an infinite square well, with boundaries \([-a/2, a/2]\).  It is actually a bit tidier seeming to generalize this slightly to boundaries \([a,b]\), which also implicitly solves the problem.  This is surely a problem that is done in 700 other QM texts, but I liked the way I did it this time so am writing it down.

\section{Guts}

Our equation to solve is \(i \Hbar \Psi_t = -(\Hbar^2/2m) \Psi_{xx}\).  Separation of variables \(\Psi = T \phi\) gives us

\begin{equation}\label{eqn:liboff41:1}
\begin{aligned}
T &\propto e^{-i E t/\Hbar } \\
\phi'' &= -\frac{2 m E }{\Hbar^2} \phi
\end{aligned}
\end{equation}

With \(k^2 = 2 m E/\Hbar^2\), we have

\begin{equation}\label{eqn:liboff41:2}
\begin{aligned}
\phi = A e^{i k x } + B e^{-i k x},
\end{aligned}
\end{equation}

and the usual \(\phi(a) = \phi(b) = 0\) boundary conditions give us

\begin{equation}\label{eqn:liboff41:3}
\begin{aligned}
0 =
\begin{bmatrix}
e^{i k a } & e^{-i k a} \\
e^{i k b } & e^{-i k b}
\end{bmatrix}
\begin{bmatrix}
A \\
B
\end{bmatrix}.
\end{aligned}
\end{equation}

We must have a zero determinant, which gives us the constraints on \(k\) immediately
\begin{equation}\label{eqn:liboff41:26}
\begin{aligned}
0 &= e^{i k (a - b)} - e^{i k (b-a)} \\
&= 2 i \sin( k (a - b) ).
\end{aligned}
\end{equation}

So our constraint on \(k\) in terms of integers \(n\), and the corresponding integration constant \(E\)

\begin{equation}\label{eqn:liboff41:4}
\begin{aligned}
k &= \frac{n \pi}{b - a} \\
E &= \frac{\Hbar^2 n^2 \pi^2 }{2 m (b-a)^2}.
\end{aligned}
\end{equation}

One of the constants \(A,B\) can be eliminated directly by picking any one of the two zeros from \eqnref{eqn:liboff41:3}

\begin{equation}\label{eqn:liboff41:46}
\begin{aligned}
&A e ^{i k a } + B e^{-i k a} = 0 \\
&\implies \\
&B = -A e ^{2 i k a }
\end{aligned}
\end{equation}

So we have

\begin{equation}\label{eqn:liboff41:5}
\begin{aligned}
\phi = A \left( e^{i k x } - e^{ ik (2a - x) } \right).
\end{aligned}
\end{equation}

Or,
\begin{equation}\label{eqn:liboff41:5b}
\begin{aligned}
\phi = 2 A i e^{i k a} \sin( k (x-a ))
\end{aligned}
\end{equation}

Because probability densities, currents and the expectations of any operators will always have paired \(\phi\) and \(\phi^\conj\) factors, any constant phase factors like \(i e^{i k a}\) above can be dropped, or absorbed into the constant \(A\), and we can write

\begin{equation}\label{eqn:liboff41:5d}
\begin{aligned}
\phi = 2 A \sin( k (x-a ))
\end{aligned}
\end{equation}

The only thing left is to fix \(A\) by integrating \(\Abs{\phi}^2\), for which we have

\begin{equation}\label{eqn:liboff41:66}
\begin{aligned}
1 &= \int_a^b \phi \phi^\conj dx \\
&= A^2 \int_a^b dx \left( e^{i k x } - e^{ ik (2a - x) } \right) \left( e^{-i k x } - e^{ -ik (2a - x) } \right) \\
&= A^2 \int_a^b dx \left( 2 - e^{ik(2a - 2x)} - e^{ik(-2a + 2x)} \right) \\
&= 2 A^2 \int_a^b dx \left( 1 - \cos (2 k (a - x)) \right)
\end{aligned}
\end{equation}

This last trig term vanishes over the integration region and we are left with

\begin{equation}\label{eqn:liboff41:6}
\begin{aligned}
A = \inv{ \sqrt{2 (b-a)}},
\end{aligned}
\end{equation}

which essentially completes the problem.  A final substitution back into \eqnref{eqn:liboff41:5b} allows for a final tidy up

\begin{equation}\label{eqn:liboff41:5c}
\begin{aligned}
\phi = \sqrt{\frac{2}{b-a}} \sin( k (x-a )).
\end{aligned}
\end{equation}

\EndArticle
