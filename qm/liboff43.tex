%
% Copyright � 2012 Peeter Joot.  All Rights Reserved.
% Licenced as described in the file LICENSE under the root directory of this GIT repository.
%

%
%
%\input{../peeter_prologue_print.tex}
%\input{../peeter_prologue_widescreen.tex}

\chapter{More problems from Liboff chapter 4}
\label{chap:liboff43}
%\useCCL
\blogpage{http://sites.google.com/site/peeterjoot/math2010/liboff43.pdf}
\date{June 25, 2010}
\revisionInfo{liboff43.tex}

%\beginArtWithToc
\beginArtNoToc

\section{Motivation}

Some more problems from \citep{liboff2003iqm}.

\section{Problem 4.11}

Some problems on Hermitian adjoints.  The starting point is the definition of the adjoint \(A^\dagger\) of \(A\) in terms of the inner product

\begin{equation}\label{eqn:liboff43:20}
\begin{aligned}
\braket{\hatA^\dagger \phi}{\psi} = \braket{\phi}{\hatA \psi}
\end{aligned}
\end{equation}

\subsection{4.11 a}

\begin{equation}\label{eqn:liboff43:40}
\begin{aligned}
\braket{ \phi }{ (a \hatA + b \hatB) \psi }
&=
a \braket{ \phi }{ \hatA \psi } + b \braket{ \phi }{ \hatB \psi }  \\
&=
a \braket{ \hatA^\dagger \phi }{ \psi } + b \braket{ \hatB^\dagger \phi }{ \psi }  \\
&=
\braket{ a^\conj \hatA^\dagger \phi }{ \psi } + \braket{ b^\conj \hatB^\dagger \phi }{ \psi }  \\
&=
\braket{ (a^\conj \hatA^\dagger + b^\conj \hatB^\dagger ) \phi }{ \psi }  \\
&\implies \\
(a \hatA + b \hatB)^\dagger = (a^\conj \hatA^\dagger + b^\conj \hatB^\dagger)
\end{aligned}
\end{equation}

\subsection{4.11 b}
\begin{equation}\label{eqn:liboff43:60}
\begin{aligned}
\braket{ \phi }{ \hatA \hatB \psi }
&=
\braket{ \hatA^\dagger \phi }{ \hatB \psi }  \\
&=
\braket{ \hatB^\dagger \hatA^\dagger \phi }{ \psi }  \\
&\implies \\
(\hatA \hatB )^\dagger &=
\hatB^\dagger \hatA^\dagger
\end{aligned}
\end{equation}

%\subsection{4.11 c}
\subsection{4.11 d}

Hermitian adjoint of \(D^2\), where \(D = \PDi{x}{}\).  Here we need the integral form of the inner product

\begin{equation}\label{eqn:liboff43:80}
\begin{aligned}
\braket{\phi}{D^2 \psi}
&=
\int \phi^\conj \PD{x}{}\PD{x}{\psi} \\
&=
-\int \PD{x}{\phi^\conj} \PD{x}{\psi} \\
&=
\int \psi \PD{x}{}\PD{x}{\phi^\conj} \\
&\implies \\
(D^2)^\dagger &= D^2
\end{aligned}
\end{equation}

Since the text shows that the square of a Hermitian operator is Hermitian, one perhaps wonders if \(D\) is (but we expect not since \(\hatp = -i \Hbar D\) is Hermitian).

Suppose \(\hatA = aD\), we have

\begin{equation}\label{eqn:liboff43:100}
\begin{aligned}
\hatA^\dagger = -a^\conj D,
\end{aligned}
\end{equation}

so for this to be Hermitian (\(\hatA = \hatA^\dagger\)) we must have \(- a^\conj = a\).  If \(a = r e^{i\theta}\), we have

\begin{equation}\label{eqn:liboff43:120}
\begin{aligned}
-1 = e^{2 i\theta}
\end{aligned}
\end{equation}

So \(\theta = \pi (1/2 + n)\), and \(a = \pm i r\).  This fixes the scalar multiples of \(D\) that are required to form a Hermitian operator

\begin{equation}\label{eqn:liboff43:140}
\begin{aligned}
\hatA &= \pm i r D
\end{aligned}
\end{equation}

where \(r\) is any real positive constant.

\subsection{4.11 e}

\begin{equation}\label{eqn:liboff43:160}
\begin{aligned}
(\hatA \hatB - \hatB \hatA)^\dagger &= - (\hatA^\dagger \hatB^\dagger - \hatB^\dagger \hatA^\dagger)
\end{aligned}
\end{equation}

\subsection{4.11 f}

\begin{equation}\label{eqn:liboff43:180}
\begin{aligned}
(\hatA \hatB + \hatB \hatA)^\dagger &= \hatA^\dagger \hatB^\dagger + \hatB^\dagger \hatA^\dagger
\end{aligned}
\end{equation}

\subsection{4.11 g}

\begin{equation}\label{eqn:liboff43:200}
\begin{aligned}
i (\hatA \hatB - \hatB \hatA)^\dagger &= i ( \hatA^\dagger \hatB^\dagger - \hatB^\dagger \hatA^\dagger)
\end{aligned}
\end{equation}

\subsection{4.11 h}

This one was to calculate \((\hatA^\dagger)^\dagger\).  Intuitively I had expect that \((\hatA^\dagger)^\dagger = \hatA\).  How could one show this?

Trying to show this with Dirac notation, I got all mixed up initially.

Using the more straightforward and old fashioned integral notation (as in \citep{bohm1989qt}), this is more straightforward.  We have the Hermitian conjugate defined by

\begin{equation}\label{eqn:liboff43:220}
\begin{aligned}
\int \psi_2^\conj (\hatA \psi_1) = \int (\hatA^\dagger \psi_2^\conj) \psi_1,
\end{aligned}
\end{equation}

Or, more symmetrically, using braces to indicate operator direction

\begin{equation}\label{eqn:liboff43:240}
\begin{aligned}
\int \psi_2^\conj (\hatA \psi_1) = \int (\psi_2^\conj \hatA^\dagger) \psi_1.
\end{aligned}
\end{equation}

Introduce a couple of variable substitutions for clarity

\begin{equation}\label{eqn:liboff43:260}
\begin{aligned}
\phi_1 &= \psi_1^\conj \\
\phi_2 &= \psi_2^\conj \\
\hatB &= \hatA^\dagger.
\end{aligned}
\end{equation}

We then have

\begin{equation}\label{eqn:liboff43:280}
\begin{aligned}
\int \psi_2^\conj (\hatA \psi_1)
&=
\int (\psi_2^\conj \hatA^\dagger) \psi_1 \\
&=
\int (\phi_2 \hatB) \phi_1^\conj \\
&=
\int \phi_1^\conj (\hatB \phi_2) \\
&=
\int (\phi_1^\conj \hatB^\dagger) \phi_2 \\
&=
\int \phi_2 (\hatB^\dagger \phi_1^\conj) \\
&=
\int \psi_2^\conj (\hatA^{\dagger \dagger} \psi_1) \\
\end{aligned}
\end{equation}

Since this is true for all \(\psi_k\), we have \(\hatA = \hatA^{\dagger \dagger}\) as expected.

Having figured out the problem in the simpleton way, it is now simple to go back and translate this into the Dirac inner product notation without getting muddled.  We have

\begin{equation}\label{eqn:liboff43:300}
\begin{aligned}
\braket{ \psi_2 }{ \hatA \psi_1 }
&=
\braket{ \hatA^\dagger \psi_2 }{ \psi_1 }  \\
&=
\braket{ \hatB \phi_2^\conj }{ \phi_1^\conj }  \\
&=
{\braket{ \phi_1 }{ \hatB^\conj \phi_2}}^\conj  \\
&=
{\braket{ (\hatB^\conj)^\dagger \phi_1 }{ \phi_2}}^\conj  \\
&=
\braket{\phi_2^\conj }{ \hatB^\dagger \phi_1^\conj } \\
&=
\braket{\psi_2 }{ \hatA^{\dagger \dagger} \psi_1 } \\
\end{aligned}
\end{equation}

\subsection{4.11 i}

\begin{equation}\label{eqn:liboff43:320}
\begin{aligned}
(\hatA \hatA^\dagger)^\dagger &= (\hatA^\dagger)^\dagger \hatA^\dagger
\end{aligned}
\end{equation}

since \((\hatA^\dagger) ^\dagger = \hatA\)

\begin{equation}\label{eqn:liboff43:340}
\begin{aligned}
(\hatA \hatA^\dagger)^\dagger &= \hatA \hatA^\dagger.
\end{aligned}
\end{equation}

\section{Problem 4.12 d}

If \(\hatA\) is not Hermitian, is the product \(\hatA^\dagger \hatA\) Hermitian?  To start we need to verify that \(\braket{\psi}{\hatA^\dagger \phi} = \braket{\hatA \psi}{\phi}\).

\begin{equation}\label{eqn:liboff43:360}
\begin{aligned}
\braket{ \psi }{ \hatA^\dagger \phi }
&=
{\braket{ (\hatA^\dagger)^\conj \phi^\conj }{ \psi^\conj }}^\conj \\
&=
{\braket{ \phi^\conj }{ \hatA^\conj \psi^\conj }}^\conj \\
&=
\braket{ \psi }{ \hatA \psi }.
\end{aligned}
\end{equation}

With that verified we have

\begin{equation}\label{eqn:liboff43:380}
\begin{aligned}
\braket{ \psi }{ \hatA^\dagger \hatA \phi }
&=
\braket{ \hatA \psi }{ \hatA \phi }  \\
&=
\braket{ \hatA^\dagger \hatA \psi }{ \phi },
\end{aligned}
\end{equation}

so, the answer is yes.  Provided the adjoint exists, that product will be Hermitian.

\section{Problem 4.14}

Show that \(\Expectation{\hatA} = \Expectation{\hatA}^\conj\) (that it is real), if \(\hatA\) is Hermitian.  This follows by expansion of that conjugate

\begin{equation}\label{eqn:liboff43:400}
\begin{aligned}
\Expectation{\hatA}^\conj
&= \left(\int \psi^\conj \hatA \psi \right)^\conj \\
&= \int \psi \hatA^\conj \psi^\conj \\
&= \int (\hatA \psi)^\conj \psi \\
&= \braket{ \hatA \psi }{ \psi } \\
&= \braket{ \psi }{ \hatA^\dagger \psi } \\
&= \braket{ \psi }{ \hatA \psi } \\
&= \Expectation{\hatA}
\end{aligned}
\end{equation}

\EndArticle
