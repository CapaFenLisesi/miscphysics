%
% Copyright � 2012 Peeter Joot.  All Rights Reserved.
% Licenced as described in the file LICENSE under the root directory of this GIT repository.
%

%
%
%\documentclass{article}

%\input{../peeters_macros.tex}
%\input{../peeters_macros2.tex}

%\usepackage[bookmarks=true]{hyperref}

%\usepackage{color,cite,graphicx}
   % use colour in the document, put your citations as [1-4]
   % rather than [1,2,3,4] (it looks nicer, and the extended LaTeX2e
   % graphics package.
%\usepackage{latexsym,amssymb,epsf} % do not remember if these are
   % needed, but their inclusion can not do any damage


\chapter{Applications of Fourier distribution theory to some PDEs}
\label{chap:distributions}
%\author{Peeter Joot \quad peeterjoot@protonmail.com }
\date{ March 4, 2009.  distributions.tex }

%\begin{document}
%\maketitle{}
%\tableofcontents

\section{Motivation}

I recently listened to Prof Brad Osgood's lectures on distributions in Fourier
transform theory and read the associated lecture notes
\citep{osgoodFourier}.  Here is an attempt to apply these ideas to solution
of some of the common PDEs of physics (wave and Poisson equations).

Some of these were tackled recently using ``classical'' Fourier methods
in \citep{gabookII:PJpoisson}.
This requires ad-hoc PV associations of integrals with delta and step
functions.  Such solutions do not inspire confidence.  Without
validation of the solutions by substitution back into the generating PDE
or comparison to a known solution one is left wondering
if all the right fudges were actually performed to get the answer.

\section{Conventions and definitions}

\subsection{Transform pairs}

The form of the Fourier transform pairs used here will be

\begin{equation}\label{eqn:distributions:20}
\begin{aligned}
\hat{f}(\Bk) &= \frac{1}{(\sqrt{2\pi})^n} \int f(\Bx) e^{-i \Bk \cdot \Bx } d^n x \\
{f}(\Bx) &= \frac{1}{(\sqrt{2\pi})^n} \int \hat{f}(\Bk) e^{i \Bk \cdot \Bx } d^n k \\
\end{aligned}
\end{equation}

Of particular interest will be forward and inverse Fourier transforms
of functions in \(\SchwartzSpace\), the space of Schwartz functions.  These are defined
more exactly in Osgood's notes, but in short have all the most desirable
properties one can imagine in a Fourier transformation context.  Namely
continuous, infinitely differentiable, with the function and all
derivatives rapidly decreasing (faster than any polynomial).

\section{1D first order homogeneous wave equation}

This is the simplest PDE that I can think of that one should be able
to apply Fourier techniques to.  We seek solutions \(f(x,t)\) of

\begin{equation}\label{eqn:distributions:firstOrderAdvancedWave}
\begin{aligned}
\inv{v} \PD{t}{f} - \PD{x}{f} = 0
\end{aligned}
\end{equation}

\subsection{Setup}

Taking Fourier transforms of \eqnref{eqn:distributions:firstOrderAdvancedWave}, and integration
by parts, we have

\begin{equation}\label{eqn:distributions:40}
\begin{aligned}
\calF\left(\inv{v} \PD{t}{f} - \PD{x}{f} \right) &=
\inv{v} \inv{\sqrt{2\pi}} \int \PD{t}{f} e^{-i k x} dx - \inv{\sqrt{2\pi}} \int \PD{x}{f} e^{-i k x } dx
\end{aligned}
\end{equation}

The second term can be integrated by parts

\begin{equation}\label{eqn:distributions:60}
\begin{aligned}
- \inv{\sqrt{2\pi}} \int \PD{x}{f} e^{-i k x } dx
=
- \inv{\sqrt{2\pi}} {\left. {f} e^{-i k x } \right\vert}_{x = -\infty}^{\infty}
+ \inv{\sqrt{2\pi}} \int {f} (-i k ) e^{-i k x } dx
\end{aligned}
\end{equation}

Already we can see we are in some trouble here since we require
\(f(x,t)\) vanish at \(\pm \infty\).
For now I close my eyes, and ignore the fact that the final solution will be seen to be more
general, and not necessarily have such a restriction.

FIXME: continue reading Osgood and figure out how to do this setup in the distribution formalism?
With a proper setup of this problem in the distribution formalism one should expect that this trouble will
go away, since the function \(f\) will be coupled with the Schwartz test function
that vanishes satisfactorily at the boundaries.

Again closing eyes, if one assumes that the order of integration and
differentiation in the time
derivative term above can be reversed, then
we have a nice single variable first order LDE to solve for \(\hat{f}\)

\begin{equation}\label{eqn:distributions:80}
\begin{aligned}
\inv{v} \PD{t}{\hat{f}}(k,t) + (-ik)\hat{f}(k,t) = 0
\end{aligned}
\end{equation}

Our solution in the wave number space is thus
\begin{equation}\label{eqn:distributions:startingPtFirstOrder}
\begin{aligned}
{\hat{f}}(k,t) = A(k) e^{i k v t} = \hat{f}(k,0) e^{i k v t}
\end{aligned}
\end{equation}

This will be the common starting point for both the classical and the distribution approach below.
Because of the integration by parts restriction and the reversion of order of integration and time differentiation, this probably ought not
be the starting point for the distribution case.  Let us compromise and work with what is known so far one step at a time.

\subsubsection{On reversal of the time derivative and integration order}

If one introduces a time variation
in the integral bounds, but still have them go to infinity based on some
other limiting process, as in

\begin{equation}\label{eqn:distributions:100}
\begin{aligned}
\PD{t}{} \int_{a(R,t)}^{b(R,t)} f(x,t) dx
&=
{\left.\PD{t}{f}(x,t)\right\vert}_{x = b(R,t)} \PD{t}{b}(R,t)
-{\left.\PD{t}{f}(x,t)\right\vert}_{x = a(R,t)} \PD{t}{a}(R,t)
+ \int_{a(R,t)}^{b(R,t)} \PD{t}{f}(x,t) dx
\end{aligned}
\end{equation}

then
it appears the time derivative of \(f\) must vanish faster than the time derivatives of the
integral bounds go to infinity in order to fully justify the switch of
the order of differentiation and integration above.  That is a rather loose statement, but probably a correct one.  I expect that
a complete formulation of the problem in terms of distributions would also avoid this issue, since the Schwartz functions and their
derivatives have the desirable property of vanishing ``rapidly''.

\subsection{Classical way}

To complete the solution in terms of initial conditions we can inverse Fourier transform
\eqnref{eqn:distributions:startingPtFirstOrder}

\begin{equation}\label{eqn:distributions:120}
\begin{aligned}
{{f}}(x,t) = \inv{\sqrt{2\pi}} \left( \inv{\sqrt{2\pi}}\int f(x',0) e^{-i k x'} dx' e^{i k v t} \right) e^{ i k x } dk
\end{aligned}
\end{equation}

Here is where things get fishy.  Reversing the order of integration, and performing
the \(k\) integration first we want to ``integrate''

\begin{equation}\label{eqn:distributions:140}
\begin{aligned}
I(x,t) = \inv{2\pi} \int e^{ i k ((x-x') + v t) } dk
\end{aligned}
\end{equation}

This we can fudge by saying taking the principle value integral (symmetric about the origin)

\begin{equation}\label{eqn:distributions:160}
\begin{aligned}
I(x,t) &= \lim_{R \rightarrow \infty} \inv{\pi} \frac{\sin( R (x-x' + v t) )}{ (x-x' + vt)}
\end{aligned}
\end{equation}

this is a nascent form of the delta function:

\begin{equation}\label{eqn:distributions:180}
\begin{aligned}
\delta(x) \sim \lim_{R \rightarrow \infty} \frac{\sinc( R x )}{\pi x}
\end{aligned}
\end{equation}

So we have
\begin{equation}\label{eqn:distributions:200}
\begin{aligned}
I(x,t) &= \delta(x - x' + v t)
\end{aligned}
\end{equation}

\begin{equation}\label{eqn:distributions:220}
\begin{aligned}
{{f}}(x,t)
&= \int f(x',0) \delta(x - x' + v t) dx' \\
&= f(x + v t,0)
\end{aligned}
\end{equation}

The first order advanced wave equation is thus solved by any function \(g(x + vt)\).  This
is familiar from
\citep{french1971vaw} where it was stated (then demonstrated correct) that solutions of the second
order homogeneous wave equation in one dimension can be expressed in terms of any functions
\(g(x \pm vt)\).

By invoking just the right incantations we can pull the delta function out of the magic hat and
produce this expected result.

\subsection{Using distributions}

Starting with \eqnref{eqn:distributions:startingPtFirstOrder} for now, we can form a functional pairing with a test function \(\psi\) in the Schwartz function space \(\SchwartzSpace\) by integration.  That is
assume that our function \(\hat{f}(k,t)\), or equivalently \(f(x,t)\) can be found for all test functions \(\psi(k) \in \SchwartzSpace\) as follows

\begin{equation}\label{eqn:distributions:240}
\begin{aligned}
\int {\hat{f}}(k,t) \psi(k) dk = \int \hat{f}(k,0) e^{i k v t} \psi(k) dk
\end{aligned}
\end{equation}

In terms of \(f\) this is

\begin{equation}\label{eqn:distributions:260}
\begin{aligned}
\int \left( \inv{2\pi} \int f(x,t) e^{-i k x} dx \right) \psi(k) dk = \int \left(\inv{2\pi} \int f(x,0) e^{-i k x} dx \right) e^{i k v t} \psi(k) dk
\end{aligned}
\end{equation}

Reversing order of integration we have

\begin{equation}\label{eqn:distributions:280}
\begin{aligned}
\int f(x,t) \left( \inv{2\pi} \int \psi(k) e^{-i k x} dk \right) dx = \int f(x,0) \left(\inv{2\pi} \int \psi(k) e^{-i k (x - v t)} dk \right) dx
\end{aligned}
\end{equation}

What we have left on both sides is a plain old Fourier transform, which exists since \(\psi(k) \in \SchwartzSpace\). We can write

\begin{equation}\label{eqn:distributions:300}
\begin{aligned}
\phi(x) = \inv{2\pi} \int \psi(k) e^{-i k x} dk
\end{aligned}
\end{equation}

which is

\begin{equation}\label{eqn:distributions:320}
\begin{aligned}
\int f(x,t) \phi(x) dx = \int f(x,0) \phi(x - vt) dx
\end{aligned}
\end{equation}

With a change of variables on the right hand side we have
%, and \(x = x'\) on the left we have
% x = x' + vt

\begin{equation}\label{eqn:distributions:340}
\begin{aligned}
\int f(x,t) \phi(x) dx = \int f(x + vt,0) \phi(x) dx
\end{aligned}
\end{equation}

Equivalently, for all \(\phi(x) \in \SchwartzSpace\), this is

\begin{equation}\label{eqn:distributions:360}
\begin{aligned}
\int (f(x,t) - f(x + vt, 0) \phi(x) dx = 0
\end{aligned}
\end{equation}

Now that is pretty slick, and we end up with the same result \(f(x,t) = f(x + vt, 0)\) for any function(al) f.
What was a delta function in the classical approach became nothing more than variable substitution and evaluation at a point.  The work required to get to the end result
is really no harder (at least in this case), with the distribution approach, but just requires a different viewpoint.

What is more, the only place where
any sort of unjustified step was made was the reversal of the order of integration.  There the rapid decrease property of the test function \(\psi\) can probably be used
to make that more rigorous (FIXME: have to think that through).

\subsection{Setup take II.  With distributions}

The pairing of the functional \(f\) with a function in \(\SchwartzSpace\) via integration was patched in mid stream above after having already Fourier transformed the original PDE.
How can this be done upfront?

Suppose instead of the differential \eqnref{eqn:distributions:firstOrderAdvancedWave}, we couple this entire equation using integration to a function in \(\SchwartzSpace\).  That is

\begin{equation}\label{eqn:distributions:380}
\begin{aligned}
\int \left( \inv{v} \PD{t}{f} - \PD{x}{f} \right) \phi(x) dx = 0
\end{aligned}
\end{equation}

and look for functional solutions \(f\) of this equation that hold for all \(\phi \in \SchwartzSpace\).  In particular we can write \(\phi\) as a Fourier transform

\begin{equation}\label{eqn:distributions:400}
\begin{aligned}
\phi(x) = \inv{\sqrt{2\pi}} \int \psi(k) e^{-i k x} dk
\end{aligned}
\end{equation}

which gives us

\begin{equation}\label{eqn:distributions:420}
\begin{aligned}
\int \left( \inv{v} \PD{t}{f} - \PD{x}{f} \right) \inv{\sqrt{2\pi}} \int \psi(k) e^{-i k x} dk = 0
\end{aligned}
\end{equation}

Now (again omitting the justification for switching order of integration) we have

\begin{equation}\label{eqn:distributions:440}
\begin{aligned}
\int \left( \inv{\sqrt{2\pi}} \int \left( \inv{v} \PD{t}{f} - \PD{x}{f} \right) e^{-i k x} dx \right) \psi(k) dk = 0
\end{aligned}
\end{equation}

Now we have a better justification for the integration by parts.  Even if \(f\) and \(\PDi{t}{f}\) do not vanish at infinity, we have a pairing with \(\psi(k)\) which does, and
in a handwaving fashion can feel a bit better with having done that in the first place.  Some additional thought about the rigor for these operations is definitely warranted, but
that also unfortunately likely involves a serious study about Lebesgue integration and sets of measure zero and all that jazz.  I had prefer not to go there, at least no time
soon.

Doing the final integration by parts and reversing the order of integration and differentiation takes us very close to the original starting point

\begin{equation}\label{eqn:distributions:460}
\begin{aligned}
\int \left( \inv{v} \PD{t}{\hat{f}}(k,t) + (-i k ) \hat{f}(k,t) \right) \psi(k) dk = 0
\end{aligned}
\end{equation}

We want this to be true for all \(\psi(k) \in \SchwartzSpace\), so we want functional solutions of the single variable LDE

\begin{equation}\label{eqn:distributions:480}
\begin{aligned}
\inv{v} \PD{t}{\hat{f}}(k,t) + (-i k ) \hat{f}(k,t) = 0
\end{aligned}
\end{equation}

This is exactly our original starting point.  Was much gained here by starting from a distribution point of view from the get go?  Not too much, at least
given the fact that only vague justification of the integration and differentiation order switching was given.  That was not even handwaving, but more
like a vague statement that it looks like handwaving is possible.

The fact
that this can be expressed entirely in terms of functionals it worth observing.  However, when it comes to practical matters, lets take a lazy Engineer's view ... use the parts
of the formalism that seem reasonable, and make life easier, but leave the rest for others to worry about.

\section{Second order wave equation}

Now, let us tackle the second order homogeneous equation

\begin{equation}\label{eqn:distributions:secondOrderWave}
\begin{aligned}
\inv{v^2} \PDSq{t}{f} - \sum_j \PDSq{x_j}{f} = 0
\end{aligned}
\end{equation}

Again Fourier transform the equation

\begin{equation}\label{eqn:distributions:500}
\begin{aligned}
\frac{1}{(\sqrt{2\pi})^n} \int \left( \inv{v^2} \PDSq{t}{f} - \sum_j \PDSq{x_j}{f} \right) e^{-i \Bk \cdot \Bx } d^n x \\
\end{aligned}
\end{equation}

From this we get

\begin{equation}\label{eqn:distributions:520}
\begin{aligned}
\inv{v^2} \PDSq{t}{\hat{f}}(\Bk,t) - \sum_j (-i k_j)^2 \hat{f}(\Bk,t) = 0
\end{aligned}
\end{equation}

Or
\begin{equation}\label{eqn:distributions:540}
\begin{aligned}
\PDSq{t}{\hat{f}}(\Bk,t) = - \Bk^2 v^2 \hat{f}(\Bk,t)
\end{aligned}
\end{equation}

With solutions of the form

\begin{equation}\label{eqn:distributions:560}
\begin{aligned}
\hat{f}(\Bk,t) = A(\Bk) e^{i \Abs{\Bk} v t} = \hat{f}(\Bk,0) e^{i \Abs{\Bk} v t}
\end{aligned}
\end{equation}

Here we can allow \(v\) to be positive or negative since solutions of either \(\pm i \Abs{v \Bk} t\) are allowed.

Now let us switch gears and make the functional nature of \(\hat{f}\) explicit.  For \(\psi(\Bk) \in \SchwartzSpace\), we want solutions
of

\begin{equation}\label{eqn:distributions:580}
\begin{aligned}
\int \hat{f}(\Bk,t) \psi(\Bk) d^n k = \int \hat{f}(\Bk,0) e^{i \Abs{\Bk} v t} \psi(\Bk) d^n k
\end{aligned}
\end{equation}

As in the first order wave equation, \(\psi\) can be written as a Fourier transform

\begin{equation}\label{eqn:distributions:fourierWaveTx}
\begin{aligned}
\psi(\Bk) = \frac{1}{(\sqrt{2\pi})^n} \int \phi(\Bx) e^{i \Bk \cdot \Bx } d^n x
\end{aligned}
\end{equation}

This gives us
\begin{equation}\label{eqn:distributions:600}
\begin{aligned}
\int {f}(\Bx,t) \phi(\Bx) d^n x
%&= \frac{1}{(\sqrt{2\pi})^n} \int \hat{f}(\Bk,0) e^{i \Abs{\Bk} v t + i \Bk \cdot \Bx } \psi(\Bx) d^n k d^n x \\
&= \frac{1}{(\sqrt{2\pi})^n} \int {f}(\Bx',0) e^{i \Abs{\Bk} v t - i \Bk \cdot \Bx' } \psi(\Bk) d^n k d^n x' \\
&=
\int {f}(\Bx',0) \left( \frac{1}{(\sqrt{2\pi})^n} \psi(\Bk) e^{i \Abs{\Bk} v t - i \Bk \cdot \Bx' } d^n k
\right) d^n x' \\
\end{aligned}
\end{equation}

In the classical Fourier treatment we end up with a Green's function of the form

\begin{equation}\label{eqn:distributions:620}
\begin{aligned}
G(\Bx, t) = \frac{1}{({2\pi})^n} \int e^{i \Abs{\Bk} v t - i \Bk \cdot \Bx' } d^n k
\end{aligned}
\end{equation}

And to solve the problem we have to somehow evaluate such an integral to get something delta function like.  In the distribution
formalism we see the equivalent problem is one of evaluating something that is structurally quite similar to
a Fourier transform

\begin{equation}\label{eqn:distributions:640}
\begin{aligned}
I(\Bx', t) = \frac{1}{(\sqrt{2\pi})^n} \int \psi(\Bk) e^{i \Abs{\Bk} v t - i \Bk \cdot \Bx' } d^n k
\end{aligned}
\end{equation}

This is a much more well defined looking problem.  It is not obvious however how exactly one can relate this to \(\phi(\Bx)\) as defined above in \eqnref{eqn:distributions:fourierWaveTx}?  One could also go around in circles here because attempting to reduce this integral could end up going straight back to the dubious ``integration'' methods for the original Green's function.

My conclusion is that I have to study more of Osgood's notes before going further.

\section{Poisson equation}

TODO.

\section{Non-homogeneous wave equation}

TODO.

%\bibliographystyle{plainnat}
%\bibliography{myrefs}

%\end{document}
